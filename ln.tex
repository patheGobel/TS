\documentclass[12pt]{article}
\usepackage{stmaryrd}
\usepackage{graphicx}
\usepackage[utf8]{inputenc}
\usepackage[french]{babel}
\usepackage[T1]{fontenc}
\usepackage{hyperref}
\usepackage{verbatim}
\usepackage{color,soul}
\usepackage{amsmath}
\usepackage{amsfonts}
\usepackage{amssymb}
\usepackage{systeme}
\usepackage{tkz-tab}
\author{Destiné à la TerminaleS2\\Au Lycée de Dindéferlo}
\title{\textbf{Fonction Exponentielle}}
\date{\today}
\usepackage{tikz}
\usetikzlibrary{arrows}
\usepackage[a4paper,left=20mm,right=20mm,top=15mm,bottom=15mm]{geometry}
\usepackage{mathtools}
\usepackage{systeme}

\usepackage{pgfplots}
\pgfplotsset{compat=1.15}
\usepackage{mathrsfs}
\usetikzlibrary{arrows}
\pagestyle{empty}

\DecimalMathComma

\begin{document}

\maketitle
\newpage
\subsection*{\underline{\textbf{\textcolor{red}{1.Définition :}}}}
La fonction \( f(x) = \ln(x) \) est continue et strictement croissante sur \( ]0; +\infty[ \) donc, c'est une bijection de \( ]0; +\infty[ \) vers \( \mathbb{R} \). Ainsi, \( f \) admet une bijection réciproque \( f^{-1} \) qui est continue et strictement croissante de \( \mathbb{R} \) vers \( ]0; +\infty[ \). \( f^{-1} \) est appelée fonction exponentielle, notée : $exp(x)=e^{x}$
\subsection*{\underline{\textbf{\textcolor{red}{2.Conséquences de la définition}}}}
\subsection*{\underline{\textbf{\textcolor{red}{3.Propriétés}}}}
\textbf{\textcolor{red}{Propriété fondamentale}}\\
Pour tout réel a et b, on a: $e^{a+b}=e^{a} \times e^{b}$.\\
\textbf{\textcolor{red}{Propriétés}}\\
\begin{itemize}
    \item \textbf{} \(e^{-a}=\frac{1}{e^{a}}\)
    \item \textbf{} \(e^{a-b}=\frac{e^{a}}{e^{b}}\)
    \item \textbf{} \(e^{ra}=(e^{a})^{r}\)
    \item \textbf{} \(e^{a}=e^{b} \Leftrightarrow a=b\)
    \item \textbf{} \(e^{a}<e^{b} \Leftrightarrow a<b\)
\end{itemize}
\subsection*{\underline{\textbf{\textcolor{red}{Exemples}}}}
\subsection*{\underline{\textbf{\textcolor{red}{Remarque}}}}
\subsection*{\underline{\textbf{\textcolor{red}{4.Limites Usuelles}}}}
\underline{\textbf{\textcolor{red}{Limites aux bornes de Df }}}\\
$\lim_{x \to -\infty}e^{x}=0$\\
$\lim_{x \to +\infty}e^{x}=+\infty$\\
\underline{\textbf{\textcolor{red}{Limites usuelles}}}\\
$\lim_{x \to +\infty}\frac{e^{x}}{x}=+\infty$\\
$\lim_{x \to -\infty}xe^{x}=0$\\
\underline{\textbf{\textcolor{red}{Preuve de quelques limites}}}\\

\underline{\textbf{\textcolor{red}{Exemple}}}\\
Déterminer les limites suivantes:\\
$\bullet$ La fonction $x \longmapsto e^{-x^{2}+x}$ est dérivable sur $\mathbb{R}$ et sa dérivée est la fonction\\
$\bullet$ La fonction $x \longmapsto e^{\cos x}$ est dérivable sur $\mathbb{R}$ et sa dérivée est  la fonction\\
$\bullet$ La fonction $x \longmapsto e^{\frac{1}{x}}$ est dérivable sur $\mathbb{R}^{*}$ et sa dérivée est  la fonction\\
\underline{\textbf{\textcolor{red}{Exercice d'application}}}\\
Déterminer les limites suivantes:\\
Calculer les limites suivantes.\\
a)\(\lim_{x \to +\infty}\frac{3e^{x}-2}{5e^{x}+3}\) ; b)\(\lim_{x \to -\infty}\frac{\ln(1+e^{x})}{e^{x}}\)
c)\( \lim_{x \to +\infty }(x-e^{x})\) ; d)\( \lim_{x \to +\infty }\frac{\sin2x}{1-e^{x}}\)
\subsection*{\underline{\textbf{\textcolor{red}{6.Limites des composées avec exp}}}}
\underline{\textbf{\textcolor{red}{Propriété}}}\\
Soit U une fonction dérivablesur un intervalle I de $\mathbb{R}$.\\
La fonction $\exp\circ u $ est dérivable sur I  et on a: $(\exp\circ u )'=u' \times \exp\circ u$\\
\underline{\textbf{\textcolor{red}{NB}}}\\
La fonction $\exp\circ u $ est généralement notée $e^{u}$ ; sa dérivée est alors $u'e^{u}$.\\
\underline{\textbf{\textcolor{red}{Exemple}}}\\
Calcule la limite suivante\\

\underline{\textbf{\textcolor{red}{Solution}}}\\
\subsection*{\underline{\textbf{\textcolor{red}{7.Dérivée}}}}
Soit u et v deux fonctions strictement positives \\

\underline{\textbf{\textcolor{red}{Exemple}}}\\
\subsection*{\underline{\textbf{\textcolor{red}{8.Croissance Comparée de $\ln x$ $e^{x}$ $x^{\alpha}$}}}}
$\lim_{x \to +\infty}\frac{e^{x}}{x^{\alpha}}=+\infty$\\
$\lim_{x \to +\infty}x^{\alpha}e^{-x}=0$\\
\underline{\textbf{\textcolor{red}{Remarque}}}\\
\underline{\textbf{\textcolor{red}{Exemple}}}\\
Détermine: $\lim_{x \to +\infty}\frac{e^{x}}{\ln(x^{2}+1)}$
\subsection*{\underline{\textbf{\textcolor{red}{9.Equation système et Inequation avec exp}}}}
\underline{\textbf{\textcolor{red}{a°)Equation}}}\\
\underline{\textbf{\textcolor{red}{Exemple}}}\\
Résoudre dans $\mathbb{R}$ les équations suivantes\\
a)$e^{x}=-1$;\\  b)$e^{x+1}=3$;\\ c)$e^{x^{2}}=e^{x+2}$;\\ d)$(e^{x}-2)(e^{-x}+1)$\\
\underline{\textbf{\textcolor{red}{b°)Système d'inéquations avec $\exp$:}}}\\
\[
\begin{cases}
4e^{x}-3e^{y} = 9 \\
2e^{x}+e^{y} = 7
\end{cases}
\]

\[
\begin{cases}
e^{x}e^{y} = 10 \\
e^{x-y} = \frac{2}{5}
\end{cases}
\]

\[
\begin{cases}
e^{2x}-7e^{y+1} = -10 \\
x-y = 1
\end{cases}
\]
\underline{\textbf{\textcolor{red}{c°)Inéquations avec $\exp$:}}}\\
a)$e^{-x}\geq2$\\
b)$e^{x^{2}-3}\leq e^{2x}$\\
c)$2e^{2x}-5e^{x}+2>0$
\section*{\underline{\textbf{\textcolor{red}{10.Etude le fonction exp}}}}
Soit $f(x)=exp(x)$ le domaine \\
Le Domaine $D_{f}$ \\
$D_{f}=\mathbb{R}$\\
$\bigotimes$ Limites aux bornes de $D_{f}$\\
\underline{En $-\infty $}\\
$\lim_{x \to -\infty} e^{x}=0$\\
\underline{En $+\infty$}\\
$\lim_{{x \to +\infty}} e^{x}=+\infty$\\
$\bigotimes$ La dérivée de f\\
$f'(x)=e^{x}$\\
$\forall x \in \mathbb{R}$, $f'(x)>0$, donc f est croissante sur $]0;+\infty[$\\
$\bigotimes$ Tableau de variation\\
%Tableau de Variation
\definecolor{cqcqcq}{rgb}{0.7529411764705882,0.7529411764705882,0.7529411764705882}
\begin{tikzpicture}[line cap=round,line join=round,>=triangle 45,x=1cm,y=1cm]
\draw [color=cqcqcq,, xstep=1cm,ystep=1cm] (-7,-10) grid (-22,17);
\clip(-22,-5) rectangle (12,10);
\draw [line width=2pt] (-23,8)-- (-7,8); %première ligne A(-22,8)---B(-7,8)
\draw [line width=2pt] (-22,6)-- (-7,6); %deuxième ligne
\draw [line width=2pt] (-22,4)-- (-7,4); %troisième ligne
\draw [line width=2pt] (-22,-2)-- (-7,-2);%dernière ligne
\draw [line width=2pt] (-22,-2)-- (-22,8); %première colonne
\draw [line width=2pt] (-19,8)-- (-19,-2); %deuxième colone
\draw [line width=2pt] (-7,8)-- (-7,-2); %troisième colonne
\draw (-21,1.5) node[anchor=north west] {$f(x)$};
\draw (-21,5.5) node[anchor=north west] {$f'(x)$};
\draw (-21,7) node[anchor=north west] {$x$};
\draw (-19,7) node[anchor=north west] {$-\infty$};
\draw (-8,7) node[anchor=north west] {$+\infty$};
%Asymptote verticale
%\draw [line width=2pt] (-18.7,6)-- (-18.7,-2);
%¨\draw [line width=2pt] (-18.79,6)-- (-18.79,-2);
%signe de la dérivé
\draw (-13.5,5.3) node[anchor=north west] {$+$};
%zéro de exp
\draw [line width=2pt] (-13,6)-- (-13,1);
\draw (-13.2,6.5) node[anchor=north west] {$0$};
\draw (-13.2,1.2) node[anchor=north west] {$1$};
\draw [->,line width=2pt] (-18,-1) -- (-8,3);
\draw (-18.5,-1) node[anchor=north west] {$0$};
\draw (-8,3.5) node[anchor=north west] {$+\infty$};
\end{tikzpicture}

\begin{tikzpicture}[scale=1.5]
  \draw[->] (-5,0) -- (2,0) node[right] {$x$};
  \draw[->] (0,-0.5) -- (0,2) node[above] {$y$};
  \draw[scale=1,domain=-5:1.5,smooth,variable=\x,blue] plot ({\x},{exp(\x)});
  %\node[right,blue] at (1.5,1.5) {$f(x) = e^x$};
  \draw[scale=1,domain=-1.5:1.5,smooth,variable=\x,blue] plot ({\x},{\x+1});
  
%  \draw[scale=1,domain=-5:1.5,smooth,variable=\x,blue] plot ({\x},{\ln(\x)});
  %\node[right,blue] at (1.5,1.5) {$f(x) = e^x$};
%  \draw[scale=1,domain=-1.5:1.5,smooth,variable=\x,blue] plot ({\x},{\x-1});
\end{tikzpicture}

$C_{f}$ est au-dessous de sa tangente en J; donc $\forall x \in \mathbb{R}, e^{x}>x+1$
\section*{\underline{\textbf{\textcolor{red}{11.Branche infinie de ln}}}}
On a  $\lim_{{x \to +\infty}} e^{x}=+\infty$ et $\lim_{{x \to +\infty}} \frac{e^{x}}{x}=+\infty$\\
Nous avons ainsi une branche parabolique de direction (Oy) au voisinage de +$\infty$.\\
Car $\lim_{{x \to +\infty}} \frac{ln(x)}{x}=0$\\
\section*{\underline{\textbf{\textcolor{red}{12.Application}}}}
\end{document}
