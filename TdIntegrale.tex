\documentclass[12pt]{article}
\usepackage{stmaryrd}
\usepackage{graphicx}
\usepackage[utf8]{inputenc}

\usepackage[french]{babel}
\usepackage[T1]{fontenc}
\usepackage{hyperref}
\usepackage{verbatim}

\usepackage{color, soul}

\usepackage{pgfplots}
\pgfplotsset{compat=1.15}
\usepackage{mathrsfs}

\usepackage{amsmath}
\usepackage{amsfonts}
\usepackage{amssymb}
\usepackage{tkz-tab}
\author{\\Lycée de Dindéfelo\\Mr BA}
\title{\textbf{Calcul Intégral Ts2}}
\date{\today}
\usepackage{tikz}
\usetikzlibrary{arrows, shapes.geometric, fit}

% Commande pour la couleur d'accentuation
\newcommand{\myul}[2][black]{\setulcolor{#1}\ul{#2}\setulcolor{black}}
\newcommand\tab[1][1cm]{\hspace*{#1}}

\begin{document}
\maketitle
\newpage

\section*{Calcul d'intégral par la recherche d'une primitive}

\subsection*{Exercice 1}
Calculer les intégrales suivantes :
\[
\text{I} = \int_{1}^{3}\frac{2x+1}{x^{2}+x+1}\,\mathrm{d}x\;;\qquad \text{J} = \int_{2}^{3}\frac{2}{\sqrt{u}}\,\mathrm{d}u\;;
\]
\[
\text{K} = \int_{-1}^{2}(x+1)(x^{2}+2x-7)\,\mathrm{d}x\;;\qquad \text{L} = \int_{1}^{3}\frac{\mathrm{d}t}{t+1}
\]
\[
\text{M} = \int_{0}^{\frac{\pi}{2}}\cos^{2}x\sin x\,\mathrm{d}x\;;\qquad \text{N} = \int_{-3}^{3}x\,\mathrm{e}^{2}\,\mathrm{d}x\;;
\]

\subsection*{Exercice 2}
Calculer les intégrales suivantes :
\[
1)\ \int_{0}^{5}(x^{4}-x^{2})\,\mathrm{d}x\qquad 2)\ \int_{1}^{2}(2x^{2}+5x-1)\,\mathrm{d}x
\]
\[
3)\ \int_{0}^{4}(2x^{5}+2x^{3}-1)\,\mathrm{d}x\qquad 4)\ \text{J} = \int_{2}^{5}\frac{\mathrm{d}x}{\sqrt{x-1}}
\]
\[
5)\ \int_{1}^{2}(x+5)^{4}\,\mathrm{d}x\qquad 6)\ \int_{1}^{5}\left(x^{2}+\frac{2}{x^{2}}\right)\,\mathrm{d}x
\]
\[
7)\ \int_{0}^{1}\left(2x^{2}-1-\frac{1}{(x+1)^{2}}\right)\,\mathrm{d}x\qquad 8)\ \int_{-1}^{2}\frac{2x+1}{(x^{2}+x+1)^{2}}\,\mathrm{d}x
\]
\[
9)\ \int_{-1}^{2}\frac{x+1}{\sqrt{x^{2}+2+5}}\,\mathrm{d}x\qquad 10)\ \int_{-1}^{2}(x^{3}-5^{2}+1)^{4}(3x^{2}-10)\,\mathrm{d}x
\]
\[
11)\ \int_{0}^{\frac{\pi}{2}}(\sin x-3\cos x+1)\,\mathrm{d}x\qquad 12)\ \int_{0}^{\frac{\pi}{2}}(\sin 2x+\cos 3x)\,\mathrm{d}x
\]
\[
13)\ \int_{0}^{\frac{\pi}{4}}\frac{\mathrm{d}x}{\cos^{2}x}\qquad 14)\ \int_{0}^{\frac{\pi}{2}}\tan^{2}x\,\mathrm{d}x
\]
\[
15)\ \int_{0}^{\frac{\pi}{4}}\frac{\mathrm{d}x}{\cos^{4}x}\qquad 16)\ \int_{0}^{\frac{\pi}{2}}(\cos^{2}x+\cos x-1)\sin x\,\mathrm{d}x
\]
\[
17)\ \int_{1}^{x}\frac{\ln t}{t}\,\mathrm{d}t\;,\ x > 0\qquad 18)\ \int_{0}^{\frac{\pi}{4}}(\tan x+\tan^{3}x)\,\mathrm{d}x
\]
\[
19)\ \int_{0}^{\frac{\pi}{4}}\tan x\,\mathrm{d}x\qquad 20)\ \int_{0}^{3}\frac{x^{2}+2x}{(x+1)^{3}}\,\mathrm{d}x
\]
\[
21)\ \int_{0}^{\frac{\pi}{3}}\frac{\sin x}{\cos^{4}x}\,\mathrm{d}x\qquad 22)\ \int_{\ln 3}^{\ln 4}\mathrm{e}^{2x}\,\mathrm{d}x
\]
\[
23)\ \int_{0}^{2}\sqrt{x}\,\mathrm{d}x\qquad 24)\ \int_{1}^{2}\sqrt{3x-1}\,\mathrm{d}x
\]
\[
25)\ \int_{1}^{3}\frac{1}{x^{2}}\mathrm{e}^{\frac{1}{x}}\,\mathrm{d}x\qquad 26)\ \int_{-1}^{4}\frac{x^{3}+1}{(x^{4}+4x+7)^{3}}\,\mathrm{d}x\qquad 27)\ \int_{\frac{\pi}{3}}^{\frac{\pi}{4}}\frac{\cos x+x\sin x}{x^{2}}\,\mathrm{d}x
\]

\subsection*{Exercice 3}
Déterminer les réels $a\;,\ b\text{ et }c$ puis calculer $I$.
\begin{align*}
1) & \quad f(x)=\dfrac{x+1}{x+2}=a+\dfrac{b}{x+2} \quad I=\int_{1}^{2}f(x)\mathrm{d}x \\
2) & \quad f(x)=\dfrac{x^{2}+x+1}{x-3}=ax+b+\dfrac{c}{x-3} \quad I=\int_{0}^{1}f(x)\mathrm{d}x \\
3) & \quad f(x)=\dfrac{2x+1}{x^{2}-5x+4}=\dfrac{a}{x-1}+\dfrac{b}{x-4} \quad I=\int_{2}^{3}f(x)\mathrm{d}x \\
4) & \quad f(x)=\dfrac{x-2}{(x-1)^{2}}=\dfrac{a}{x-1}+\dfrac{b}{(x-1)^{2}} \quad I=\int_{-1}^{0}f(x)\mathrm{d}x \\
5) & \quad f(x)=\dfrac{1}{x(x^{2}+1)}=\dfrac{a}{x}+\dfrac{bx+c}{x^{2}+1} \quad I=\int_{1}^{1}\dfrac{\mathrm{d}x}{x(x^{2}+1)} \\
6) & \quad f(x)=\dfrac{\mathrm{e}^{x}-2}{\mathrm{e}^{x}+1}=a+\dfrac{b\mathrm{e}^{-x}}{1+\mathrm{e}^{-x}} \quad I=\int_{\ln2}^{\ln3}f(x)\mathrm{d}x
\end{align*}

\subsection*{Exercice 4}
Calculer la fonction dérivée de la fonction $f: x \mapsto (ax^{2}+bx+c)\sqrt{x^{2}+2}\;,\ a\;,\ b\;,\ c$ étant des constantes.

En déduire le calcul de l'intégrale 
\[
\int_{-1}^{4}\frac{6x^{3}+2x^{2}+9x+2}{\sqrt{x^{2}+2}}\,\mathrm{d}x
\]

\subsection*{Exercice 5}
Soit :
\[
I = \int_{0}^{\frac{\pi}{2}}\frac{\cos x}{\cos x+\sin x}\,\mathrm{d}x\quad \text{et}\quad J = \int_{0}^{\frac{\pi}{2}}\frac{\sin x}{\cos x+\sin x}\,\mathrm{d}x
\]
Calculer $I + J$ et $I - J$.

En déduire $I$ et $J$.

\section*{Linéarisation}
\subsection*{Exercice 6}
Calculer les intégrales suivantes après linéarisation :
\begin{align*}
1) & \quad \int_{0}^{\frac{\pi}{2}}\sin^{3}x\cos^{2}x\mathrm{d}x \\
2) & \quad \int_{0}^{\frac{\pi}{4}}\sin^{4}x\mathrm{d}x \\
3) & \quad \int_{0}^{\frac{\pi}{2}}\cos^{3}t\mathrm{d}t \\
4) & \quad \int_{0}^{\pi}\sin^{2}u\cos^{2}u\mathrm{d}u \\
5) & \quad \int_{\frac{\pi}{4}}^{\frac{\pi}{3}}\cos x\cos 3x\cos 5x\mathrm{d}x \\
6) & \quad \int_{0}^{\frac{\pi}{3}}\cos 2x\sin^{3}x\mathrm{d}x
\end{align*}
\section*{Intégration par parties}
\subsection*{Exercice 8}
\subsubsection*{Calculer les intégrales suivantes à l'aide d'une ou plusieurs intégrations par parties}
\begin{align*}
1) & \quad \int_{\mathrm{e}}^{2\mathrm{e}}x^{3}\ln x\mathrm{d}x \\
2) & \quad \int_{1}^{\mathrm{e}}x^{2}\ln x^{2}\mathrm{d}x \\
3) & \quad \int_{\mathrm{e}}^{2\mathrm{e}}x(\ln x)^{2}\mathrm{d}x \quad (\text{double intégration par parties})\\
4) & \quad \int_{0}^{1}x\mathrm{e}^{x}\mathrm{d}x \\
5) & \quad \int_{1}^{2}x^{2}\mathrm{e}^{x}\mathrm{d}x \quad (\text{double intégration par parties})\\
6) & \quad \int_{0}^{1}(x-3)\mathrm{e}^{2x}\mathrm{d}x \\
7) & \quad \int_{0}^{2}(t^{2}+1)\mathrm{e}^{t}\mathrm{d}t \quad (\text{double intégration par parties})\\
8) & \quad \int_{0}^{\pi}(x+2)\sin x\mathrm{d}x \\
9) & \quad \int_{0}^{\frac{\pi}{2}}\mathrm{e}^{x}\sin x\mathrm{d}x \quad \text{et} \quad \int_{0}^{\frac{\pi}{4}}\mathrm{e}^{x}\cos x\mathrm{d}x \quad (\text{double intégration par parties})\\
\end{align*}

\subsection*{Exercice 9}
\subsubsection*{Calculer les intégrales suivantes à l'aide d'une ou plusieurs intégrations par parties}
\begin{align*}
1) & \quad \int_{1}^{a}x^{2}\ln x^{2}\mathrm{d}x \quad (a>0)\\
2) & \quad \int_{0}^{t}(3x^{2}+x+1)\cos x\mathrm{d}x \\
3) & \quad \int_{1}^{t}\dfrac{x\ln x}{(1+x^{2})^{2}}\mathrm{d}x \\
4) & \quad \int_{0}^{\frac{\pi}{2}}x^{2}\sin x\mathrm{d}x \\
5) & \quad \int_{1}^{x}t^{n}\ln t\mathrm{d}t \quad (x>0, \ n\in\mathbb{N}\setminus\{-1\})\\
6) & \quad \int_{1}^{1}(x^{2}+x+1)\sin 2x\mathrm{d}x \\
7) & \quad \int_{\frac{\pi}{4}}^{\frac{\pi}{2}}(x^{2}+1)\cos^{2}x\mathrm{d}x \\
8) & \quad \int_{-1}^{1}(1+x)^{2}\mathrm{e}^{-x}\mathrm{d}x \\
9) & \quad \int_{0}^{\frac{\pi}{3}}\dfrac{t\sin t}{\cos^{3}t}\mathrm{d}x \\
10) & \quad \int_{1}^{\lambda}\ln(x+\sqrt{x^{2}-1})\mathrm{d}x \quad (\lambda>1)\\
11) & \quad \int_{\frac{\pi}{4}}^{\frac{\pi}{3}}\cos x\ln(\cos x+1)\mathrm{d}x \\
12) & \quad \int_{0}^{2}x^{2}\mathrm{e}^{|x-1|}\mathrm{d}x \\
13) & \quad \int_{0}^{\frac{\pi}{3}}\mathrm{e}^{-x}\sin^{2}x(1+\cos^{2}x)\mathrm{d}x \\
\end{align*}

\subsection*{Exercice 10}
\subsubsection*{Soit l'intégrale}
\[I_{n}=\int_{1}^{\lambda}(\ln x)^{n}\mathrm{d}x, \quad \lambda>0, \quad n\in\mathbb{N}\]
1) Montrer, à l'aide d'une intégration par parties que l'on a :
\[I_{n}=\lambda(\ln\lambda)^{n}-n\,I_{n-1}.\]
2) Montrer alors que :
\[I_{n}=\lambda[(\ln\lambda)^{n}-n(\ln\lambda)^{n-1}+n(n-1)(\ln\lambda)^{n-2}+\cdots+(-1)^{n}\times n !]-(-1)^{n}\times n !\]
3) Calculer \(I_{0}, I_{1}, I_{2}\) et \(I_{3}.\)

\subsection*{Exercice 11}
\subsubsection*{Soit l'intégrale}
\[I=\int_{0}^{\frac{\pi}{4}}\dfrac{\mathrm{d}x}{\cos^{5}x}\]
Pour tout entier naturel \(n\), on pose :
\[I_{n}=\int_{0}^{\frac{\pi}{4}}\dfrac{\mathrm{d}x}{\cos^{2n+1}x}\]
1) Montrer qu'il existe deux réels \(a\) et \(b\) tels que :
\[\forall \;x\in\left[0\;;\ \dfrac{\pi}{4}\right]\;,\ \dfrac{1}{\cos x}=\dfrac{a\cos x}{1-\sin x}+\dfrac{b\cos x}{1+\sin x}\]
En déduire le calcul de \(I_{0}.\)
2) Montrer, par une intégration par parties que pour tout \(n\in\mathbb{N}^{\ast}\) :
\[2 nI_{n}=(2 n-1)\ln-1+\dfrac{2^{n}}{\sqrt{2}}.\]
3) En déduire le calcul de \(I.\)

\end{document}