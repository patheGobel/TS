\documentclass[12pt]{article}
\usepackage{lmodern} % Pour une police plus nette
\usepackage{stmaryrd}
\usepackage{graphicx} % Pour l'insertion d'images
\usepackage{float}    % Pour contrôler précisément le placement
\usepackage[utf8]{inputenc}
\usepackage[french]{babel}
\usepackage[T1]{fontenc}
\usepackage{hyperref}
\usepackage{verbatim}
\usepackage{color, soul}
\usepackage{pgfplots}
\pgfplotsset{compat=1.18} % Version plus récente de pgfplots
\usepackage{mathrsfs}
\usepackage{amsmath}
\usepackage{amsfonts}
\usepackage{amssymb}
\usepackage{tkz-tab}
%\author{Destiné aux élèves de Terminale S\\Lycée de Dindéfelo\\Présenté par M. BA}
%\title{\textbf{Rappels et compléments sur les fonctions numériques}}
%\date{\today}
\usepackage{tikz}
\usetikzlibrary{arrows, shapes.geometric, fit}
% Commande pour la couleur d'accentuation
\newcommand{\myul}[2][black]{\setulcolor{#1}\ul{#2}\setulcolor{black}}
\newcommand\tab[1][1cm]{\hspace*{#1}}
\usepackage[margin=2.5cm]{geometry} % Ajustement des marges
\usepackage{eso-pic} % Pour ajouter des éléments en arrière-plan

% Commande pour ajouter du texte en arrière-plan, centré au milieu de chaque page
\AddToShipoutPicture{
    \AtPageCenter{%
        \makebox(0,0)[c]{\rotatebox{60}{\textcolor[gray]{0.9}{\fontsize{2cm}{2cm}\selectfont Pathé Gobel BA}}}
    }
}

\begin{document}

\noindent
\begin{minipage}[t]{0.48\textwidth}
\raggedright
\textbf{Ministère de l'Éducation Nationale}\\
Inspection Académique de Kédougou\\
Lycée Dindéfelo\\
Cellule de Mathématiques
\end{minipage}
\hfill
\begin{minipage}[t]{0.48\textwidth}
\raggedleft
\textbf{Année scolaire 2024-2025}\\
Date : 02/12/2024\\
Classe : Terminale S2\\
Professeur : M. BA
\end{minipage}
\begin{center}
\underline{\textbf{Suites numériques}}
\end{center}
\section*{Exercice 1 \quad Échauffement}

Déterminer toutes les primitives des fonctions suivantes sur \( I \) :

\begin{enumerate}
    \item[\textbf{a)}] \( f(x) = \cos x \sin^5 x, \quad I = \mathbb{R}; \)
    \item[\textbf{b)}] \( f(x) = \frac{7}{(x+2)^{5}}, \quad I = ]-\infty; -2[; \)
    \item[\textbf{c)}] \( f(x) = \frac{2x^{2}+3x-1}{x}, \quad I = ]0; +\infty[; \)
    \item[\textbf{d)}] \( f(x) = \frac{x^{2}}{1+x^{3}}, \quad I = ]0; +\infty[; \)
    \item[\textbf{e)}] \( f(x) = \frac{x}{\sqrt{3x^2 + 2}}, \quad I = \mathbb{R}; \)
    \item[\textbf{f)}] \( f(x) = \left( \frac{1}{2}x-1 \right)^6, \quad I = \mathbb{R}. \)
\end{enumerate}
\section*{Exercice 2 \quad Recherche de primitives}

Déterminer une primitive, sur un intervalle à préciser (le plus grand possible), des fonctions suivantes :

\begin{enumerate}
    \item $f(x) = 2x^2 -7x + 3;$
    \item $f(x) = x^5 - 3x^2 + 5x;$
    \item $f(x) = 6x^3 - \frac{4}{3}x^2 + \frac{1}{2}x-1;$
    \item $f(x) = x^5 -3x^{3}+ 5x;$
    \item $f(x) = 3x^{3}-\frac{1}{2x^2} + \frac{3}{x^3};$
    \item $f(x) = \frac{3}{x^2};$
    \item $f(x) = -\frac{2}{x^6};$
    \item $f(x) = \frac{1}{x^2} + \frac{2}{\sqrt{x}};$
    \item $f(x) = -\frac{3}{x^2} + \frac{1}{2\sqrt{x}} + 2\sqrt{x};$
    \item $f(x) = 3x + \frac{2}{x^2} - \frac{5}{x^3};$
    \item $f(x) = \frac{7x^3 - 4x^2 + 3x - 1}{5};$
    \item $f(x) = -\frac{5x^4 + 2x^3 - 3x + 6}{x^3};$
    \item $f(x)=\frac{4(x-1)^{3}-1}{(x-1)^{2}}.$
\end{enumerate}

\section*{Exercice 3 : Recherche de primitives}

\textit{Déterminer une primitive de la fonction $f$ sur l’intervalle $I$.}

\begin{enumerate}
    \item 
    \begin{enumerate}
        \item $f(x) = \dfrac{2x + 5}{(x^2 + 5x)^4}; \quad I = \left]0; +\infty\right[$
        \item $f(x) = \dfrac{3x^2 + 2}{(x^2 + 2x)^2}; \quad I = \left]0; +\infty\right[$
        \item $f(x) = \dfrac{x}{(x^2 + 1)^3}; \quad I = \mathbb{R}$
    \end{enumerate}

    \item 
    \begin{enumerate}
        \item $f(x) = \dfrac{7}{(x - 1)^4}; \quad I = \left]-\infty; 1\right[$
        \item $f(x) = \dfrac{9}{(4x + 1)^3}; \quad I = \mathbb{R}_+$
    \end{enumerate}
    \item
    \begin{enumerate}
        \item $f(x) = \dfrac{1 - x^2}{x^3 - 3x + 1}; \quad I = [2; +\infty[$
        \item $f(x) = \dfrac{3x^2 + 2}{x^3 + 2x}; \quad I = ]-\infty; 0[$
        \item $f(x) = \dfrac{-\dfrac{1}{6}x + \dfrac{1}{3}}{x^2 - 4x + 9}; \quad I = \mathbb{R}$
    \end{enumerate}

    \item 
    \begin{enumerate}
        \item $f(x) = \dfrac{\sin x}{\cos^3 x}; \quad I = \left]-\dfrac{\pi}{2}; \dfrac{\pi}{2}\right[$
        \item $f(x) = \dfrac{1}{1 + \tan^2 x}; \quad I = \left]0; \dfrac{\pi}{2}\right[$
        \item $f(x) = \dfrac{\tan x}{\cos x}; \quad I = \left]0; \dfrac{\pi}{2}\right[$
        \item $f(x) = \dfrac{\cos x}{\sin^3 x (1 - \cos^2 x)^6}; \quad I = \left]0; \frac{\pi}{2}\right[$
    \end{enumerate}
    \item 
    \begin{enumerate}
        \item $f(x) = \dfrac{x - 1}{(3x^2 - 6x + 11)^7}; \quad I = \mathbb{R}$
        \item \( f(x) = \frac{6 - 4x}{(x^2 - 3x - 4)^5}; \quad I = ]-\infty; -1[ \)
        \item \( f(x) = \frac{1}{\cos^2 x} \tan^3 x; \quad I = \left] - \frac{\pi}{2}; \frac{\pi}{2} \right[ \)
    \end{enumerate}
    \item
    \begin{enumerate}
    		\item \( f(x) = (3x^2 - 2) (x^3 - 2x + 3)^3; \quad I = \mathbb{R} \)
				\item \( f(x) = (2x + 1) (x^2 + x + 1)^5; \quad I = \mathbb{R} \)
    		\item \( f(x) = \frac{\sin^4 x}{\cos^6 x}; \quad I = \left[ - \frac{\pi}{2}; \frac{\pi}{2} \right[ \)
				\item \( f(x) = (x - 1)^2 (x + 2); \quad I = \mathbb{R} \)
    \end{enumerate}
    \item
    \begin{enumerate}
				\item \( f(x) = \frac{1 - x^2}{(x^3 - 3x + 1)^3}; \quad I = ]1; +\infty[ \)
 				\item \( f(x) = \frac{-15x^2 - 10}{(x^3 + 2x)^4}; \quad I = \mathbb{R}^* \).
		\end{enumerate}
		\item
		\begin{enumerate}
			\item \( f(x) = \sin x \cos^4 x; \quad I = \mathbb{R} \)
			\item \( f(x) = \sin(3x) \cos^3(3x); \quad I = \mathbb{R} \)
			\item \( f(x) = \frac{\sin^2 x}{\cos^4 x}; \quad I = \left[ 0; \frac{\pi}{4} \right] \).
		\end{enumerate}
		\item
		\begin{enumerate}
			\item Montrer que, pour tout \( x \) réel :
			\[\sin^3 x = \sin x - \sin x \cos^2 x.\]
			\item En déduire une primitive de la fonction définie par :
			\item \[f(x) = \sin^3 x, \quad x \in \mathbb{R}.\]
		\end{enumerate}
		\item
	\begin{enumerate}
		\item \( f(x) = 1 + \tan^2 x, \quad I = \left] - \frac{\pi}{2}; \frac{\pi}{2} \right[ \)
  	\item \( f(x) = \tan^2 x, \quad I = \left] - \frac{\pi}{2}; \frac{\pi}{2} \right[ \)
		\item \( f(x) = \tan^4 x + \tan^6 x, \quad I = \left] - \frac{\pi}{2}; \frac{\pi}{2} \right[ \).
	\end{enumerate}
	\item
	\begin{enumerate}
	\item \( f(x) = \frac{1}{\sqrt{3x + 1}}, \quad I = \left] -\frac{1}{3}; +\infty \right[ \).
	\item \( f(x) = \frac{1}{\sqrt{4 - 5x}}, \quad I = \left] -\infty; \frac{4}{5} \right[ \).
	\end{enumerate}
	\item
	\begin{enumerate}
	\item \( f(x) = \frac{2x^2 + \sin x}{\sqrt{2x^3 - 3 \cos x + 3}}, \quad I = [1; +\infty[ \);\\
	\item \( f(x) = \frac{\sin(2x)}{\sqrt{1 + \sin^2 x}}, \quad I = \mathbb{R} \).
	\end{enumerate}
\end{enumerate}

\section*{Exercice 4 \quad Recherche de primitives}

\emph{Déterminer la primitive de la fonction $f$ sur l'intervalle $I$ qui prend la valeur $y_0$ en $x_0 \in I$.}

\begin{enumerate}
    \item $f(x) = x - \frac{1}{x^2} + \frac{1}{\sqrt{x}}, \quad I = \mathbb{R}^*, \quad x_0 = 1 \text{ et } y_0 = 0.$
    
    \item $f(x) = \frac{1}{(2x+5)^4}, \quad I = \left]-\infty; -\frac{5}{2}\right[ \quad x_0 = -4 \text{ et } y_0 = 2.$
    
    \item $f(x) = \frac{x^4 + 3x^2 - 2}{x^{2}}, \quad I = \mathbb{R}_+^{*}, \quad x_0 = 2 \text{ et } y_0 = -3.$
    
    \item $f(x) = \frac{1}{\sqrt{x}} (x^2 - 1) + 4x\sqrt{x}, \quad I = \mathbb{R}_+^{*}, \quad x_0 = 1 \text{ et } y_0 = 10.$
    
    \item $f(x) = \sin x \cos^3 x, \quad I = \mathbb{R}, \quad x_0 = \frac{\pi}{2} \text{ et } y_0 = 0.$
\end{enumerate}

\end{document}
