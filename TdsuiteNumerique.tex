\documentclass[a4paper,11pt]{article}
\usepackage{amsmath}
\usepackage{amssymb}

\begin{document}
\section*{Exercice 1}

Soit \((u_n)\) une suite définie par :
\[
u_0 = 1 \quad \text{et} \quad u_{n+1} = \frac{u_n - 1}{3u_n + 1}.
\]

\begin{enumerate}
    \item Calculer les 4 premiers termes de la suite \((u_n)\).
    \item Représenter graphiquement la suite \((u_n)\).
    \item Montrer que \((u_n)\) est périodique de période 3.
\end{enumerate}

\section*{Exercice 2}

On considère la suite \((u_n)_{n \in \mathbb{N}}\) définie par :
\[
\begin{cases}
u_0 = 13, \\
u_{n+1} = \frac{1}{2} u_n + 7.
\end{cases}
\]

\begin{enumerate}
    \item Montrer que : \((\forall n \in \mathbb{N}), u_n < 14\).
    \item Soit \((v_n)_{n \in \mathbb{N}}\) la suite définie par : \(v_n = 14 - u_n\) pour tout \(n \in \mathbb{N}\).
    \begin{enumerate}
        \item Montrer que la suite \((v_n)_{n \in \mathbb{N}}\) est géométrique de raison \(\frac{1}{2}\), puis écrire \(v_n\) en fonction de \(n\).
        \item En déduire que : \((\forall n \in \mathbb{N}), u_n = 14 - \left( \frac{1}{2} \right)^n\).
    \end{enumerate}
\end{enumerate}

\section*{Exercice 3}

Soit \((u_n)_{n \in \mathbb{N}}\) la suite numérique définie par :
\[
\begin{cases}
u_0 = 2, \\
(\forall n \in \mathbb{N}), \ u_{n+1} = \frac{7u_n - 25}{u_n - 3}.
\end{cases}
\]

\begin{enumerate}
    \item Montrer que : \((\forall n \in \mathbb{N}), u_n \neq 5\).

    \item On considère la suite \((v_n)_{n \in \mathbb{N}}\) définie par : \(v_n = \frac{1}{u_n - 5}\) pour tout \(n \in \mathbb{N}\).
    \begin{enumerate}
        \item Montrer que \((v_n)_{n \in \mathbb{N}}\) est une suite arithmétique.
        \item Déterminer \(v_n\), puis \(u_n\) en fonction de \(n\).
    \end{enumerate}

    \item 
    \begin{enumerate}
        \item Calculer : \(S_n = v_0 + v_1 + \cdots + v_n\) en fonction de \(n\).
        \item On pose : \(P_n = 2^{v_0} \times 2^{v_1} \times \cdots \times 2^{v_n}\). Déterminer \(P_n\) en fonction de \(n\).
    \end{enumerate}
\end{enumerate}

\section*{Exercice 4}

Soit \((u_n)_{n \in \mathbb{N}^*}\) la suite numérique définie par :
\[
\begin{cases}
u_1 = \frac{1}{3}, \\
(\forall n \in \mathbb{N}^*), \ u_{n+1} = \frac{2u_n}{1 + (n+2)u_n}.
\end{cases}
\]

Soit \((v_n)_{n \in \mathbb{N}^*}\) la suite numérique définie par : \(v_n = \frac{1}{u_n} - n\).

\begin{enumerate}
    \item Montrer que la suite \((v_n)_{n \in \mathbb{N}^*}\) est géométrique.

    \item 
    \begin{enumerate}
        \item Déterminer \(v_n\) et \(u_n\) en fonction de \(n\).
        \item Calculer en fonction de \(n\) la somme : \(S_n = v_1 + v_2 + \cdots + v_n\).
    \end{enumerate}
\end{enumerate}

\section*{Exercice 5}

Soit \((u_n)_{n \in \mathbb{N}}\) la suite numérique définie par :
\[
\begin{cases}
u_0 = 1, \\
(\forall n \in \mathbb{N}), \ u_{n+1} = \frac{2u_n^3}{3u_n^2 + 1}.
\end{cases}
\]

\begin{enumerate}
    \item 
    \begin{enumerate}
        \item Montrer que : \((\forall n \in \mathbb{N}), u_n > 0\).
        \item Étudier la monotonie de la suite \((u_n)_{n \in \mathbb{N}}\).
    \end{enumerate}

    \item 
    \begin{enumerate}
        \item Montrer que : \((\forall n \in \mathbb{N}), u_{n+1} \leq \frac{1}{2} u_n\).
        \item En déduire que : \((\forall n \in \mathbb{N}), u_n \leq \left(\frac{1}{2}\right)^n\).
    \end{enumerate}
\end{enumerate}

\section*{Exercice 6}

Soit \((u_n)_{n \in \mathbb{N}}\) la suite numérique définie par :
\[
\begin{cases}
u_0 = 0, \\
u_{n+1} = \sqrt{3u_n + 4}, \quad \forall n \in \mathbb{N}.
\end{cases}
\]

\begin{enumerate}
    \item Montrer que : \(\forall n \in \mathbb{N}, \ 0 \leq u_n \leq 4\).

    \item Montrer que la suite \((u_n)_{n \in \mathbb{N}}\) est croissante.

    \item 
    \begin{enumerate}
        \item Montrer que : \(\forall n \in \mathbb{N}, \ 4 - u_{n+1} \leq \frac{1}{2}(4 - u_n)\).
        \item En déduire que : \(\forall n \in \mathbb{N}, \ 4 - u_n \leq 4 \left(\frac{1}{2}\right)^n\).
        \item Calculer \(\lim_{n \to +\infty} u_n\).
    \end{enumerate}
\end{enumerate}

\section*{Exercice 7}

\begin{enumerate}

\item Soit \((u_n)_{n \in \mathbb{N}}\) la suite numérique définie par :
\[
\begin{cases}
u_0 = \frac{10}{3}, \\
(\forall n \in \mathbb{N}), \ u_{n+1} = \frac{u_n^2 - 3u_n + 9}{u_n}.
\end{cases}
\]

\begin{enumerate}
    \item Montrer que : \((\forall n \in \mathbb{N}), u_n \geq 3\).
    \item En déduire que la suite \((u_n)_{n \in \mathbb{N}}\) est décroissante et que : \((\forall n \in \mathbb{N}), 3 \leq u_n \leq \frac{10}{3}\).
\end{enumerate}

\item On considère la suite \((v_n)_{n \in \mathbb{N}}\) définie par :
\[
\begin{cases}
v_0 = \frac{1}{2}, \\
(\forall n \in \mathbb{N}), \ v_{n+1} = \frac{2v_n^2}{1 + v_n^2}.
\end{cases}
\]

\begin{enumerate}
    \item Montrer que : \((\forall n \in \mathbb{N}), v_n > 0 \text{ et } v_n \leq \frac{1}{2}\).
    \item Montrer que : \((\forall n \in \mathbb{N}), \frac{v_{n+1}}{v_n}  \leq 1\). En déduire que \((v_n)_{n \in \mathbb{N}}\) est décroissante.
\end{enumerate}

\item Soit \((t_n)_{n \in \mathbb{N}}\) la suite définie par :
\[
\begin{cases}
t_0 = 1, \\
(\forall n \in \mathbb{N}), \ t_{n+1} = \sqrt{\frac{1}{2} t_n + \frac{3}{2}}.
\end{cases}
\]

Montrer que la suite \((t_n)_{n \in \mathbb{N}}\) est strictement croissante.

\item Soit \((w_n)_{n \in \mathbb{N}}\) la suite définie par :
\[
\begin{cases}
w_0 = 1, \\
(\forall n \in \mathbb{N}), \ w_{n+1} = f(w_n).
\end{cases}
\]
où \(f(x) = x^2 - 2x\).

\item Montrer que : \((\forall n \in \mathbb{N}), w_n \geq 3\).
\item Montrer que la suite \((w_n)_{n \in \mathbb{N}}\) est croissante.
\end{enumerate}

\section*{Exercice 8}

On considère la fonction \(f\) définie sur \(\mathbb{R}\) par :
\[
f(x) = \frac{x}{1 + x + x^2}.
\]

\begin{enumerate}
    \item 
    \begin{enumerate}
        \item Étudier les variations de \(f\).
        \item Déterminer \(f([0,1])\).
    \end{enumerate}
    \item On considère la suite \((u_n)_{n \in \mathbb{N}}\) définie par :
    \[
    \begin{cases}
    u_0 = 1, \\
    u_{n+1} = f(u_n), \quad \forall n \in \mathbb{N}.
    \end{cases}
    \]
    \begin{enumerate}
        \item Montrer que : \(\forall n \in \mathbb{N}, \ 0 \leq u_n \leq 1\), puis \(\forall n \in \mathbb{N}, \ u_{n+1} < u_n\).
        \item Montrer que : \(\forall n \in \mathbb{N}^*, \ f\left(\frac{1}{n}\right) \leq \frac{1}{n+1}\), en déduire que : \(\forall n \in \mathbb{N}^*, \ 0 \leq u_n \leq \frac{1}{n}\).
        \item Montrer que la suite \((u_n)_{n \in \mathbb{N}}\) est convergente, puis calculer \(\lim_{n \to +\infty} u_n\).
    \end{enumerate}
\end{enumerate}

\section*{Exercice 9}

Démontrer par récurrence les propriétés suivantes :

\begin{enumerate}
    \item \(3^{2n+1} + 2^{n+2}\) est divisible par 7.
    \item \(\forall n \in \mathbb{N}, \ (a+b)^n = \sum_{p=0}^n C_n^p a^p b^{n-p}.\)
    \item \(\forall n \in \mathbb{N}, \ 0 \leq u_n \leq 2\) où
    \[
    \begin{cases}
    u_0 = 1, \\
    u_{n+1} = \frac{2u_n + 1}{u_n + 1}.
    \end{cases}
    \]

    \item Soit \((u_n)\) la suite définie par :
    \[
    u_0 = 2 \quad \text{et} \quad u_{n+1} = 3u_n - 5, \quad \text{pour tout } n \in \mathbb{N}.
    \]
    Démontrer sur \(\mathbb{N}\) que :
    \[
    u_n = \frac{1}{2}(3^n - 5).
    \]

    \item Soit la suite définie par \(u_1 = 5\) et, pour tout \(n \in \mathbb{N}^*\),
    \[
    u_{n+1} = \sqrt{2 + u_n}.
    \]
    Démontrer sur \(\mathbb{N}^*\) que \(u_n \geq 2\).

    \item Démontrer que \(\forall n \geq 4, \ 2^n \geq n^2.\)
\end{enumerate}

\section*{Exercice 10}

Soit \((u_n)_{n \in \mathbb{N}}\) la suite numérique définie par :
\[
\begin{cases}
u_0 = 0, \\
u_{n+1} = \sqrt{3u_n + 4}, \quad \forall n \in \mathbb{N}.
\end{cases}
\]

\begin{enumerate}
    \item Montrer que : \((\forall n \in \mathbb{N}), \ 0 \leq u_n \leq 4\).
    \item Montrer que la suite \((u_n)_{n \in \mathbb{N}}\) est croissante.
    \item 
    \begin{enumerate}
        \item Montrer que : \((\forall n \in \mathbb{N}), \ 4 - u_{n+1} \leq \frac{1}{2}(4 - u_n)\).
        \item En déduire que : \((\forall n \in \mathbb{N}), \ 4 - u_n \leq 4 \left(\frac{1}{2}\right)^n\).
    \end{enumerate}
\end{enumerate}

\section*{Exercice 11}

Soit \((u_n)_{n \in \mathbb{N}}\) la suite numérique définie par :
\[
\begin{cases}
u_0 = 1, \\
u_{n+1} = \frac{2u_n + 3}{u_n + 2}, \quad \forall n \in \mathbb{N}.
\end{cases}
\]

\begin{enumerate}
    \item 
    \begin{enumerate}
        \item Montrer par récurrence que la suite \((u_n)_{n \in \mathbb{N}}\) est strictement croissante.
        \item Montrer que la suite \((u_n)_{n \in \mathbb{N}}\) est majorée par 2.
    \end{enumerate}

    \item On pose : \(\forall n \in \mathbb{N}, \ v_n = \frac{u_n - \sqrt{3}}{u_n + \sqrt{3}}\).
    \begin{enumerate}
        \item Montrer que \((v_n)_{n \in \mathbb{N}}\) est une suite géométrique en déterminant sa raison et son premier terme.
        \item Déterminer \(u_n\) en fonction de \(n\).
    \end{enumerate}
\end{enumerate}

\section*{Exercice 12}

Soit \((U_n)_{n \geq 1}\) la suite définie par :
\[
U_1 = 1, \quad (U_{n+1})^2 = 4U_n, \quad \forall n \geq 1.
\]

\begin{enumerate}
    \item Calculer \(U_2\), \(U_3\), \(U_4\) et \(U_5\) sous la forme \(2^\alpha\).
    \item Soit \((V_n)\) la suite définie par : \(V_n = \ln(U_n) - \ln 4, \ \forall n \geq 1\).
    \begin{enumerate}
        \item Montrer que \((V_n)\) est une suite géométrique ; préciser la raison et le premier terme.
        \item En déduire l'expression de \(V_n\), puis celle de \(U_n\) en fonction de \(n\).
    \end{enumerate}
    \item Pour quelles valeurs de \(n\) a-t-on \(U_n > 3,96\) ?
\end{enumerate}

\section*{Exercice 13}

On considère les suites \((U_n)\) et \((V_n)\) définies par :
\[
U_1 = 2, \quad U_{n+1} = \frac{U_n + 2V_n}{3}, \quad
V_1 = 8, \quad V_{n+1} = \frac{U_n + 5V_n}{6}, \quad \text{avec } n \in \mathbb{N}^*.
\]

\begin{enumerate}
    \item Soit \(a_n = V_n - U_n\) pour tout entier naturel \(n\) non nul.
    \begin{enumerate}
        \item Démontrer que \((a_n)\) est une suite géométrique.
        \item Exprimer \(a_n\) en fonction de \(n\).
    \end{enumerate}

    \item 
    \begin{enumerate}
        \item Démontrer que \((U_n)\) est croissante et \((V_n)\) décroissante.
        \item Démontrer que \((U_n)\) est majorée et \((V_n)\) minorée.
        \item En déduire que \((U_n)\) et \((V_n)\) ont une limite commune.
    \end{enumerate}

    \item Soit \(w_n = \frac{1}{4}U_n + V_n\) pour tout entier naturel \(n\) non nul.
    \begin{enumerate}
        \item Montrer que \((w_n)\) est une suite constante.
        \item En déduire les limites de \((U_n)\) et \((V_n)\).
    \end{enumerate}
\end{enumerate}

\section*{Exercice 14}

Soit la suite \((u_n)\) définie par :
\[
u_0 = 0, \quad u_{n+1} = \sqrt{2 + u_n}.
\]

\begin{enumerate}
    \item Montrer que pour tout entier naturel \(n\), \(0 \leq u_n \leq 2\).
    \item Montrer que \((u_n)\) est croissante.
    \item En déduire que \((u_n)\) converge vers un réel \(l\) à déterminer.
    \item Démontrer que \(\forall n \in \mathbb{N}, \ |u_{n+1} - 2| \leq \frac{1}{2} |u_n - 2|\), puis \(|u_n - 2| \leq \left(\frac{1}{2}\right)^n |u_0 - 2|\).
    \item Retrouver les résultats de la troisième question.
\end{enumerate}

\section*{Exercice 15}

On considère les suites \((U_n)\) et \((V_n)\) définies par :
\[
\begin{cases}
U_1 = 12, \\
U_{n+1} = \frac{U_n + 2V_n}{3},
\end{cases}
\quad \text{et} \quad
\begin{cases}
V_1 = 1, \\
V_{n+1} = \frac{U_n + 3V_n}{4},
\end{cases}
\quad \text{avec } n \in \mathbb{N}^*.
\]

\begin{enumerate}
    \item \(W_n = U_n - V_n\) pour tout \(n\) non nul. Montrer que \((W_n)\) est une suite géométrique convergente.

    \item Exprimer \(W_n\) en fonction de \(n\).

    \item Démontrer que \((U_n)\) est décroissante et \((V_n)\) est croissante.

    \item Démontrer que pour tout entier non nul \(n\), \(U_n \geq V_n\), et en déduire que \(V_1 \leq V_n \leq U_n \leq U_1\). Conclure.

    \item \(T_n = 3U_n + 8V_n\) avec \(n\) entier naturel non nul. Montrer que \((T_n)\) est une suite constante, puis déterminer les limites de \((U_n)\) et \((V_n)\).
\end{enumerate}

\section*{Exercice 16}

On considère la fonction \(f\) définie par \(f(x) = x - \ln|x|\). Soit \((C_f)\) sa courbe représentative dans le plan muni d’un repère orthonormal \((O, \vec{i}, \vec{j})\).

\begin{enumerate}
    \item Déterminer son ensemble de définition \(D_f\), puis calculer les limites aux bornes de \(D_f\).
    \item Calculer sa dérivée, étudier son signe puis dresser son tableau de variation.
    \item Montrer que \((C_f)\) coupe l’axe des abscisses en un seul point \(A\) d’abscisse \(\alpha\).
    \item Établir les inégalités suivantes :
    \begin{enumerate}
        \item \(\forall x \geq 1, \ x - \ln x \geq 1\) (on pourra utiliser l’étude de la fonction \(f\) de la partie I).
        \item \(0 \leq x \leq 1, \ x^2 \leq x\).
        \item \(0 \leq x \leq 1, \ \ln(1 + x) - x - \frac{x^2}{2} \geq 0\) (on pourra étudier les variations de \(g(x) = \ln(1 + x) - x + \frac{x^2}{2}\)).
    \end{enumerate}

    \item On considère la suite \((U_n)\) définie par :
    \[
    U_0 = 2, \quad U_{n+1} = U_n - \ln U_n, \quad \forall n \in \mathbb{N}.
    \]
    \begin{enumerate}
        \item Donner une valeur approchée de \(U_1\) à \(10^{-2}\) près et montrer que \(\forall n \in \mathbb{N}, U_n \geq 1\).
        \item Montrer que la suite \((U_n)\) est décroissante.
    \end{enumerate}

    \item On pose \(\forall n \in \mathbb{N}, V_n = U_n - 1\).
    \begin{enumerate}
        \item Calculer \(V_0\), puis exprimer \(V_{n+1}\) en fonction de \(V_n\).
        \item Préciser le sens de variation de la suite \((V_n)\).
        \item Montrer que \(\forall n \in \mathbb{N}, 0 \leq V_n \leq 1\). En déduire que \((V_n)\) converge.
        \item Montrer que \(\forall n \in \mathbb{N}, V_{n+1} \leq \frac{1}{2} V_n\) et \(0 \leq V_n \leq \left(\frac{1}{2}\right)^n\). En déduire la limite de \((V_n)\).
    \end{enumerate}
\end{enumerate}

\section*{Exercice 17}

\textit{Les parties A et B sont indépendantes.}

\subsection*{Partie A}

Un individu décide d’investir dans un plan d’épargne où le capital initial est de 5000 F CFA. Chaque année, le montant investi croît de 5\% par rapport à l’année précédente, grâce aux intérêts composés.

\begin{enumerate}
    \item Déterminer la nature de la suite représentant le capital total après \(n\) années.
    \item Calculer le premier terme de cette suite.
    \item Exprimer le terme général \(u_n\) de la suite en fonction de \(n\).
    \item Calculer la somme \(S_n = u_1 + u_2 + \cdots + u_n\), représentant le capital total accumulé après \(n\) années.
    \item Calculer la limite de \(S_n\) lorsque \(n \to +\infty\), si elle existe.
\end{enumerate}

\subsection*{Partie B}

Un individu décide de rembourser une dette de 10 000 € par versements annuels. Chaque année, il augmente son remboursement de 200 € par rapport à l’année précédente.

\begin{enumerate}
    \item Déterminer la nature de la suite représentant le montant remboursé chaque année.
    \item Calculer le premier terme de cette suite.
    \item Exprimer le terme général \(u_n\) de la suite en fonction de \(n\).
    \item Calculer la somme \(S_n = u_1 + u_2 + \cdots + u_n\), représentant le total remboursé après \(n\) années.
    \item Calculer la limite de \(S_n\) lorsque \(n \to +\infty\), si elle existe.
\end{enumerate}

\end{document}