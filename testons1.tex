\documentclass{article}
\usepackage{graphicx}
\usepackage{geometry}
\usepackage{fancyhdr}
\usepackage{amsmath}
\usepackage{amsfonts}
\usepackage{amssymb}
\usepackage{systeme}
\usepackage{color,soul}
\geometry{top=2cm, bottom=2cm, left=2cm, right=2cm}

\setlength{\headheight}{25pt} % Ajustez la hauteur de l'en-tête si nécessaire

\pagestyle{fancy}
\fancyhead[L]{\includegraphics[width=1.5cm]{flag.png}} % Remplacer 'flag.png' par le chemin de l'image du drapeau
\fancyhead[C]{\hspace{-1.5cm} \textbf{IA Kédougou - Cellule Mixte des Sciences Physiques - Première S}} % Décalage à gauche
\fancyhead[R]{\raisebox{-0.3cm}{\includegraphics[width=1.5cm]{logo.png}}\hspace{0.1cm}2024/2025} % Descendre le logo pour aligner avec le drapeau

\begin{document}

\vspace*{2cm} % Décalage vers le haut
\begin{center}
    \huge \textbf{TD : chaînes carbonées insaturées : alcènes et alcynes}
\end{center}

\noindent\rule{\textwidth}{0.4pt}

\begin{flushleft}
    \textbf{Exercice 1 :}
\end{flushleft}
++++++++++++++++++++++++++++++++++++++
 \subsection*{\underline{\textbf{\textcolor{red}{2. Propriétés}}}}

Le module d'un nombre complexe possède plusieurs propriétés importantes, qui sont utiles pour les manipulations algébriques et géométriques :

\begin{itemize}
  \item[$\bullet$] \textbf{Le module d'un conjugué :} Pour tout \(z \in \mathbb{C}\),
  \[
  |\overline{z}| = |z|.
  \]

  \item[$\bullet$] \textbf{Le module d'un produit :} Pour tous \(z_1, z_2 \in \mathbb{C}\),
  \[
  |z_1 \cdot z_2| = |z_1| \cdot |z_2|.
  \]

  \item[$\bullet$] \textbf{Le module d'une puissance :} Pour tout \(z \in \mathbb{C}\) et tout entier \(n \geq 0\),
  \[
  |z^n| = |z|^n.
  \]

  \item[$\bullet$] \textbf{Relation avec le conjugué :} Pour tout \(z = a + ib \in \mathbb{C}\),
  \[
  z \cdot \overline{z} = |z|^2 = a^2 + b^2.
  \]

  \item[$\bullet$] \textbf{Le module d'un quotient :} Pour tous \(z_1, z_2 \in \mathbb{C}\), avec \(z_2 \neq 0\),
  \[
  \left|\frac{z_1}{z_2}\right| = \frac{|z_1|}{|z_2|}.
  \]

  \item[$\bullet$] \textbf{Inégalité triangulaire :} Pour tous \(z_1, z_2 \in \mathbb{C}\),
  \[
  |z_1 + z_2| \leq |z_1| + |z_2|.
  \]

  \item[$\bullet$] \textbf{Distance dans le plan complexe :} Pour deux points \(A\) et \(B\) d’affixes respectives \(z_A\) et \(z_B\), la distance \(AB\) entre ces points est donnée par :
  \[
  |z_A - z_B| = AB.
  \]
\end{itemize}

++++++++++++++++++++++++++++++++++++++
\end{document}
