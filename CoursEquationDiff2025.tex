\documentclass[12pt]{article}
\usepackage[utf8]{inputenc}
\usepackage[french]{babel}
\usepackage{amsmath, amssymb}
\usepackage{xcolor}
\usepackage{enumitem}
\usepackage{tcolorbox}
\usepackage{geometry}
\geometry{a4paper, margin=2.5cm}

\begin{document}

\begin{center}
    \fcolorbox{red}{white}{\textbf{\textcolor{red}{CHAP 9 : ÉQUATIONS DIFFÉRENTIELLES}}}
\end{center}

\section*{\textcolor{red}{I. \underline{Activité} (Oscillateur harmonique)}}
On dispose d’un pendule élastique horizontal non amorti composé d’un ressort de raideur $k$ et d’un solide $(S)$ de masse $m$ fixé à l’extrémité mobile du ressort.\\
À l’équilibre, l’abscisse $x$ du centre d’inertie $G$ de $(S)$ est repérée par rapport à $O$ dans le repère $(O,\overrightarrow{u})$. On écarte $(S)$ de sa position d’équilibre puis on le relâche avec une vitesse initiale $v$. Les forces de frottements sont supposées nulles.
\begin{enumerate}
    \item Faire le bilan des forces appliquées à $(S)$.
    \item Établir l’équation du mouvement du solide $(S)$.
\end{enumerate}

\vspace{0.5cm}

\subsection*{\textcolor{red}{Correction de l'activité}}

\begin{enumerate}
    \item \textbf{Bilan des forces appliquées à $(S)$} :\\
    On note $x(t)$ l'abscisse du centre d'inertie $G$ du solide $(S)$ à l'instant $t$, mesurée par rapport à la position d'équilibre.

    \begin{itemize}
        \item Le solide $(S)$ est soumis à la force de rappel du ressort, proportionnelle à l’allongement $x(t)$ et dirigée vers la position d'équilibre.\\
        \textcolor{blue}{Donc :} \quad $\vec{F}_{\text{ressort}} = -k x(t) \, \vec{u}$

        \item Il n'y a pas de frottement (hypothèse de l'énoncé).
    \end{itemize}

    Par le principe fondamental de la dynamique (PFD) appliqué au solide $(S)$ dans le repère $(O,\vec{u})$ :
    \[
    m \, \ddot{x}(t) = -k x(t)
    \]

    \item \textbf{Équation du mouvement du solide $(S)$} :\\
    On obtient l'équation différentielle :
    \[
    m \, \ddot{x}(t) + k x(t) = 0
    \quad \text{ou encore} \quad
    \ddot{x}(t) + \omega^2 x(t) = 0 \quad \text{où } \omega^2 = \dfrac{k}{m}
    \]

    Il s’agit d’une EDL linéaire d’ordre 2 à coefficients constants sans second membre, modélisant un oscillateur harmonique non amorti.

    \textbf{Solution générale} :
    \[
    x(t) = A \cos(\omega t) + B \sin(\omega t) \quad \text{où } A, B \in \mathbb{R}
    \]

    \textbf{Conditions initiales} :\\
    Si le solide est relâché avec une vitesse initiale $v$ et une position initiale $x(0) = x_0$, alors :
    \[
    x(0) = A = x_0 \quad \text{et} \quad \dot{x}(0) = B \omega = v \Rightarrow B = \dfrac{v}{\omega}
    \]

    D’où la solution particulière :
    \[
    \boxed{x(t) = x_0 \cos(\omega t) + \dfrac{v}{\omega} \sin(\omega t)}
    \]
\end{enumerate}


\section*{\textcolor{red}{II. \underline{Définitions et exemples}}}

\begin{itemize}
    \item On appelle \textbf{équation différentielle linéaire (EDL)}, toute équation qui met en relation une fonction $f$ avec une ou plusieurs de ses dérivées. \textcolor{blue}{\underline{Résoudre}} ou \textcolor{blue}{\underline{intégrer}} une telle équation c’est donc déterminer l’expression de $f(x)$ qui constitue l’inconnue à rechercher. Une EDL admet toujours une infinité de solutions mais une seule des solutions vérifie $f(x_0)=y_0$
    
    \item \textbf{\textcolor{red}{Exemple :}} $(E_1) : f''(x) - f'(x) + 2f(x) = 0$ ; $(E_2) : f'(x) + 2f(x) = x^2 + 1$ etc.
    
    \item Comme $y = f(x)$ ; on notera plus souvent $y' = f'(x)$ et $y'' = f''(x)$.\\
    Ainsi les équations précédentes s’écrivent : 
    \[
        (E_1) : y'' - y' + 2y = 0 \quad \text{et} \quad (E_2) : y' + 2y = x^2 + 1
    \]
    
    \item Si la dérivée $n^{\text{ième}}$ figure dans l’équation, on dit que l’EDL est d’ordre $n$.\\
    L’équation $(E_1)$ de l’exemple précédent est d’ordre 2 et $(E_2)$ de l’exemple précédent est d’ordre 1.
\end{itemize}

\section*{\textcolor{red}{III. \underline{Résolution}}}
\subsection*{1. \textcolor{red}{EDL sans second membre}}

\subsubsection*{\textcolor{red}{a. EDL d’ordre 1 sans second membre}}
Une EDL d’ordre 1 sans second membre est une équation de la forme : \(\textcolor{red}{ay' + by = 0} ;\)

$a$ et $b$ sont des réels avec $a \neq 0$. Les solutions d’une telle équation sont de la forme : 
\[
\boxed{f(x) = Ke^{-\frac{b}{a}x}} \quad \text{où } K \in \mathbb{R}.
\]

\textbf{\textcolor{red}{Exemple}}\\
Les solutions de $y' + 2y = 0$ sont de la forme \fbox{$f(x) = Ae^{-2x}$} où $A \in \mathbb{R}$.

\vspace{0.5cm}

\subsubsection*{\textcolor{red}{b. EDL d’ordre 2 sans second membre}}
Une EDL d’ordre 2 sans second membre est une équation de la forme :
\[
\textcolor{red}{ay'' + by' + cy = 0}
\]
où $a, b$ et $c$ sont des réels avec $a \neq 0$.\\
Pour résoudre une telle équation, on considère l’équation du second degré suivante : 
\[
\boxed{(EC) : ar^2 + br + c = 0}
\]
appelée \textit{Équation caractéristique} puis on détermine son discriminant.

\vspace{0.5cm}

Si $\Delta = 0$ ; alors $(EC)$ admet une solution double $r_0$ et les solutions de l’EDL sont de la forme :
\[
\boxed{f(x) = (Ax + B)e^{r_0 x}} \quad \text{où } A, B \in \mathbb{R}.
\]

Si $\Delta > 0$ ; alors $(EC)$ admet deux solutions réelles $r_1$ et $r_2$ et les solutions de l’EDL sont de la forme :
\[
\boxed{f(x) = Ae^{r_1 x} + Be^{r_2 x}} \quad \text{où } A, B \in \mathbb{R}.
\]

Si $\Delta < 0$ ; alors $(EC)$ admet deux solutions complexes $z_1 = \alpha + i\beta$ et $z_2 = \alpha - i\beta$ et les solutions de l’EDL sont de la forme :
\[
\boxed{f(x) = [A\cos(\beta x) + B\sin(\beta x)]e^{\alpha x}} \quad \text{où } A, B \in \mathbb{R}.
\]

\vspace{0.5cm}

\textbf{\textcolor{red}{Exercice d'application}}\\
Soit l’équation $(E) : y'' + y' - 2y = 0$.
\begin{enumerate}
    \item Déterminer les solutions de $(E)$.
    \item Déterminer la solution de $(E)$ qui vérifie $f(0) = 3$ et $f'(0) = 0$.
\end{enumerate}

\textbf{\textcolor{red}{Solution}}

\begin{enumerate}
    \item L’équation $(E) : y'' + y' - 2y = 0$ est une EDL d’ordre 2 sans second membre. Son équation caractéristique est :
    \[
    (EC) : r^2 + r - 2 = 0
    \]
    Elle admet deux solutions réelles $r_1 = 1$ et $r_2 = -2$. Ainsi les solutions de $(E)$ sont de la forme :
    \[
    \boxed{f(x) = Ae^{x} + Be^{-2x}} \quad \text{où } A, B \in \mathbb{R}.
    \]

    \item On a $f(0) = A + B = 3$ et $f'(0) = A - 2B = 0$\\
    De là, on tire $A = 2B$ et $2B + B = 3$ donc $B = 1$, $A = 2$\\
    Donc \fbox{$f(x) = 2e^{x} + e^{-2x}$}
\end{enumerate}

\subsection*{2. \textcolor{red}{EDL avec second membre}}

Il existe deux types d’EDL avec second membre :

\begin{itemize}
    \item EDL d’ordre 1 avec second membre : \textcolor{yellow}{$ay' + by = \varphi(x)$} où $a ; b \in \mathbb{R}$ avec $a \neq 0$.
    \item EDL d’ordre 2 avec second membre : \textcolor{yellow}{$ay'' + by' + cy = \varphi(x)$} où $a ; b ; c \in \mathbb{R}$ avec $a \neq 0$.
\end{itemize}

Dans les deux cas, $\varphi$ est une fonction donnée.\\

Pour résoudre de telles équations, on détermine la solution $h$ de l’EDL sans second membre obtenue en enlevant $\varphi$. On cherche par suite la solution particulière $g$ obtenue grâce à l’information donnée dans l’exercice.\\

Les solutions générales sont alors de la forme : 
\[
\boxed{f(x) = g(x) + h(x)}.
\]

\vspace{0.4cm}
\textbf{\textcolor{red}{Exercice d'application}}\\
Soit l’équation différentielle $(E) : y' - 2y = xe^x$
\begin{enumerate}
    \item Déterminer les réels $a$ et $b$ pour que la fonction $u(x) = (ax + b)e^x$ soit une solution de $(E)$.
    \item En déduire toutes les solutions de $(E)$.
    \item Déterminer la solution de $(E)$ qui s’annule en $0$.
\end{enumerate}

\end{document}
