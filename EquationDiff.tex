\documentclass{article}
\usepackage{stmaryrd}
\usepackage{graphicx}
\usepackage[utf8]{inputenc}
\usepackage[french]{babel}
\usepackage[T1]{fontenc}
\usepackage{hyperref}
\usepackage{verbatim}
\usepackage{color, soul}
\usepackage{pgfplots}
\pgfplotsset{compat=1.18}
\usepackage{mathrsfs}
\usepackage{amsmath}
\usepackage{amsfonts}
\usepackage{amssymb}
\usepackage{tkz-tab}

\author{Lycée de Dindéfelo\\Mr BA}
\title{\textbf{Equations Differnetielles Lineaires}}
\date{\today}

\usepackage{tikz}
\usetikzlibrary{arrows, shapes.geometric, fit}

% Commande pour la couleur d'accentuation
\newcommand{\myul}[2][black]{\setulcolor{#1}\ul{#2}\setulcolor{black}}
\newcommand\tab[1][1cm]{\hspace*{#1}}

\begin{document}
\maketitle
\newpage
\section*{\underline{\textbf{\textcolor{red}{I. Définitions}}}}
Soient $a_{0},a_{1},...,a_{n}$ des constantes réelles, $y=y(x)$ une fonction de 
$x$. On appelle équation différentielle linéaire d'ordre $n$ à coefficients constants, une équation liant une fonction $y$ et ses dérivées successives $y',y'',\cdots , y^{n}$. On a $a_{n}y^{n}(x)+a_{n-1}y^{n-1}(x)+\cdots +a_{2}y''(x)+a_{1}y'(x)+a_{0}y(x)=g(x)$ où $g(x)$ est une fonction et $a_{n}\neq 0$
\subsection*{\underline{\textbf{\textcolor{red}{Exemple}}}}
$2y''-3y'+4y=\sin 2x$ est une équation différentielle linéaire du $2^{nd}$ ordre, à coefficients constants.
\begin{itemize}
    \item pour $a_{n}\neq 0$, \textbf{(E)}$:a_{n}y^{n}(x)+a_{n-1}y^{n-1}(x)+\cdots+a_{2}y''(x)+a_{1}y'(x)+a_{0}y(x)=g(x)$ est une équation différentielle Linéaire d'ordre $n$, à coefficients constants, non homogène ou avec second membre. 
    \item pour $a_{n}\neq 0$, \textbf{(E')}$:a_{n}y^{n}+a_{n-1}y^{n-1}+\cdots+a_{2}y''+a_{1}y'+a_{0}y=0$ est une équation différentielle linéaire d'ordre $n$, à coefficients constants, non homogène ou sans second membre ou homogène. 	   
    \item \textbf{(E')} est associée à \textbf{(E)}
    \item  Les solutions d'une équation différentielle sont des fonctions.
    \item $f$ est une solution d'une équation différentielle d'ordre $n$, à coefficients constants sur \textbf{I}, si $f$ est $n$ fois dérivables sur \textbf{I} et $f$ vérifie l'équation.
    \item L'ensemble des solutions d'une équation différentielle est appelé solution générale de l'équation différentielle.
    \item Toute fonction vérifiant l'équation différentielle est appelée solution particulière.
    \item Une solution qui vérifie des conditions initiales est appelée solution singulière.
\end{itemize}
\section*{\underline{\textbf{\textcolor{red}{II. Théorèmes}}}}
Soit l'équation différentielle linéaire d'ordre $n$ à coefficients constants avec $2^{nd}$ membre.
 
\textbf{(E)}$:a_{n}y^{(n)}(x)+a_{n-1}y^{(n-1)}(x)+\cdot+a_{2}y''(x)+a_{1}y'(x)+a_{0}y(x)=g(x)$ et l'équation différentielle linéaire d'ordre $n$ à coefficients constants sans $2^{nd}$ membre.

\textbf{(E')}$:a_{n}y^{(n)}+a_{n-1}y^{(n-1)}+\cdot+a_{2}y''+a_{1}y'+a_{0}y=0$.

Si $f_{1}$ est une solution particulière de \textbf{(E)} et $f_{2}$ une solution générale de \textbf{(E')} alors, $f(x)=f_{1}(x)+f_{2}(x)$ est une solution générale de \textbf{(E)}.

\subsection*{\underline{\textbf{\textcolor{green}{Preuve}}}}
Soit $f_1$ une solution particulière de l'équation \textbf{(E)} avec second membre, et $f_2$ une solution générale de l'équation homogène associée \textbf{(E')}. Nous devons montrer que $f(x) = f_1(x) + f_2(x)$ est une solution générale de l'équation \textbf{(E)}.

Puisque $f_1$ est une solution particulière de \textbf{(E)}, cela signifie que lorsque nous remplaçons $y$ par $f_1$ dans \textbf{(E)}, nous obtenons une équation vraie. 

C'est-à-dire: \textcolor{green}{\[a_{n}f_1^{(n)}(x) + a_{n-1}f_1^{(n-1)}(x) + \ldots + a_{2}f_{1}''(x) + a_{1}f_{1}'(x) + a_{0}f_{1}(x)=0\]}

De même, puisque $f_2$ est une solution générale de \textbf{(E')}, lorsque nous remplaçons $y$ par $f_2$ dans \textbf{(E')}, nous obtenons une équation vraie.

C'est-à-dire: \textcolor{blue}{\[a_{n}f_{2}^{(n)}(x)+a_{n-1}f_{2}^{(n-1)}(x)+\cdots+a_{2}f_{2}^{''}(x)+a_{1}f_{2}^{'}(x)+a_{0}f_{2}(x)=g(x)\]}

En ajoutant les deux équations, nous obtenons l'équation :

$\textcolor{green}{a_{n}f_1^{(n)}(x) + a_{n-1}f_1^{(n-1)}(x) + \ldots + a_{2}f_{1}''(x) + a_{1}f_{1}'(x) + a_{0}f_{1}(x)}$ + $\textcolor{blue}{a_{n}f_2^{(n)}(x) + a_{n-1}f_2^{(n-1)}(x) +}$ 
\[\textcolor{blue}{\ldots + a_{2}f_{2}''(x) + a_{1}f_{2}'(x) + a_{0}f_{2}(x)}\]
\[=\textcolor{green}{0}+\textcolor{blue}{g(x)}\]

\[a_{n}(f_{1}+f{2})^{(n)}(x)+a_{n-1}(f_{1}+f{2})^{(n-1)}(x)+\cdots+a_{2}(f_{1}+f{2})^{''}(x)+a_{1}(f_{1}+f{2})^{(')}(x)+a_{0}(f_{1}+f{2})(x)=g(x)\]
Puisque $f_1$ est une solution de l'équation avec second membre \textbf{(E)} et $f_2$ est une solution de l'équation homogène \textbf{(E')}, la partie gauche de l'équation ci-dessus est égale à $g(x) + 0$, ce qui est simplement $g(x)$. Ainsi,\\ $f(x) = f_1(x) + f_2(x)$ satisfait l'équation différentielle \textbf{(E)}, ce qui prouve que c'est une solution générale de \textbf{(E)}.

\section*{\underline{\textbf{\textcolor{red}{
\begin{minipage}[t]{\textwidth}III. Équations différentielles linéaires d'ordre 1, à coefficients constants
\end{minipage}}}}}
Une équation différentielle linéaire d'ordre 1, à coefficients constants est une équation de la forme
 \[\textbf{(E)}: ay'+by=g(x)\textbf{(avec second membre)} \]
 \[\textbf{(E')}: ay'+by=0\textbf{(sans second membre ou équation homogène)}\]
\subsection*{\underline{\textbf{\textcolor{red}{1. Recherche de la solution générale de l'équation homogène}}}}
Soit \textbf{(E')}$:ay'+by=0$ alors,

\[ay'=-by \Longrightarrow \frac{y'}{y}=-\frac{b}{a}\]
\[\Longrightarrow \ln|y|=-\frac{b}{a}x+c\]
\[\Longrightarrow |y|=e^{-\frac{b}{a}x+c}=e^{c}.e^{-\frac{b}{a}x}\]
\[\Longrightarrow y=\pm e^{c}.e^{-\frac{b}{a}x}\]

Soit \textbf{(E')}:$ay'+by=0$ alors $ay'=-by$ 

Donc, $f_{2}$ est de la forme $f_{2}(x)=y_{2}(x)=ke^{-\frac{b}{a}x}$ où $K$ est une constante.
\subsection*{\textbf{\textcolor{red}{
\begin{minipage}[t]{\textwidth}
\underline{2. Recherche d'une solution particulière de l'équation}\\
\underline{différentielle linéaire d'ordre 1 avec 2\textsuperscript{nd} membre}
\end{minipage}}}}
\begin{itemize}
    \item Une fonction $f_{1}$ est solution de \textbf{(E)} si elle est dérivable et si elle vérifie \textbf{(E)} \[af'_{1}(x)+bf_{1}(x)=g(x)\]
    \item Si $g(x)$ est un polynôme de degré $n$, alors $f_{1}$ est aussi un polynôme de degré $n$.
    \item Si \( g(x) \) est de la forme \( a\cos \beta x , a\sin \beta x \) , \( a\cos \beta x + b\sin \beta x \), alors \( f_1 \) sera de la forme \( A\cos \beta x + B\sin \beta x \).
\end{itemize}
\subsection*{\underline{\textbf{\textcolor{red}{3. Solution générale de \textbf{(E)}}}}}
Une solution générale $y(x)$ de \textbf{(E)} est donnée par \[y(x)=f_{2}(x)+f_{1}(x)\] où $f_{2}$ est une solution générale de \textbf{(E')} et $f_{1}(x)$ une solution particulière de \textbf{(E).}
\subsection*{\underline{\textbf{\textcolor{red}{Exercice d'application}}}}
Soit $\phi$ la fonction dérivable sur $\mathbb{R}$ solution de l'équation différentielle\\
\textbf{(E)}:$2y'-3y=x^{2}+5$ dont la dérivée s'annule en 0.
\begin{enumerate}
    \item Montrer qu'il existe un polynôme de degré 2, notée $f_{1}$, solution de \textbf{(E).}
    \item Résoudre l'équation différentielle \textbf{(E):}$2y'=3y$ et en déduire l'ensemble des solutions de \textbf{(E).}
    \item Déterminer $\phi$ puis construire $C_{\phi}$ sa courbe représentative. 
\end{enumerate}
\subsection*{\underline{\textbf{\textcolor{green}{Résolution }}}}
\begin{enumerate}
    \item 
    Cherchons une solution particulière $y_p$ de l'équation \textbf{(E)} sous la forme d'un polynôme de degré 2 : $y_p = ax^2 + bx + c$.\\
    Calculons $y_p'$ :
    \[
    y_p' = 2ax + b
    \]
    Substituons $y_p$ et $y_p'$ dans l'équation \textbf{(E)} :
    \[
    2(2ax + b) - 3(ax^2 + bx + c) = x^2 + 5
    \]
    \[
    4ax + 2b - 3ax^2 - 3bx - 3c = x^2 + 5
    \]
    Regroupons les termes semblables :
    \[
    -3ax^2 + (4a - 3b)x + (2b - 3c) = x^2 + 5
    \]
    Égalons les coefficients :
    \[
    \begin{cases}
    -3a = 1 \\
    4a - 3b = 0 \\
    2b - 3c = 5
    \end{cases}
    \]
    Résolvons ce système :
    \[
    a = -\frac{1}{3}
    \]
    \[
    4 \left(-\frac{1}{3}\right) - 3b = 0 \implies -\frac{4}{3} - 3b = 0 \implies b = -\frac{4}{9}
    \]
    \[
    2 \left(-\frac{4}{9}\right) - 3c = 5 \implies -\frac{8}{9} - 3c = 5 \implies -3c = 5 + \frac{8}{9} \implies -3c = \frac{45 + 8}{9} \implies -3c = \frac{53}{9} \implies c = -\frac{53}{27}
    \]
    La solution particulière est donc :
    \[
    y_p = -\frac{1}{3}x^2 - \frac{4}{9}x - \frac{53}{27}
    \]

    \item
    Résolvons maintenant l'équation homogène associée \textbf{(E')}: $2y' - 3y = 0$.
    \[
    2y' = 3y \implies \frac{y'}{y} = \frac{3}{2} \implies \int \frac{dy}{y} = \int \frac{3}{2} dx \implies \ln|y| = \frac{3}{2}x + C
    \]
    \[
    y = Ce^{\frac{3}{2}x}
    \]
    L'ensemble des solutions de \textbf{(E)} est donc :
    \[
    y = Ce^{\frac{3}{2}x} + y_p
    \]

    \item
    Puisque la dérivée de $\phi$ s'annule en 0, nous devons avoir $\phi'(0) = 0$.
    \[
    \phi(x) = Ce^{\frac{3}{2}x} - \frac{1}{3}x^2 - \frac{4}{9}x - \frac{53}{27}
    \]
    Calculons $\phi'(x)$ :
    \[
    \phi'(x) = \frac{3}{2}Ce^{\frac{3}{2}x} - \frac{2}{3}x - \frac{4}{9}
    \]
    \[
    \phi'(0) = \frac{3}{2}C - \frac{4}{9} = 0 \implies \frac{3}{2}C = \frac{4}{9} \implies C = \frac{8}{27}
    \]
    La solution est donc :
    \[
    \phi(x) = \frac{8}{27}e^{\frac{3}{2}x} - \frac{1}{3}x^2 - \frac{4}{9}x - \frac{53}{27}
    \]
    
    \item
    Pour construire la courbe représentative $C_{\phi}$, nous traçons la fonction $\phi$ sur un graphique.

    \begin{figure}[h!]
        \centering
        \begin{tikzpicture}
        \begin{axis}[
            axis lines = middle,
            xlabel = $x$,
            ylabel = {$\phi(x)$},
            domain=-2:2,
            samples=100,
            width=12cm,
            height=8cm,
            grid=both,
        ]
        \addplot [
            color=blue,
            thick
        ]
        { (8/27) * exp(3*x/2) - (1/3) * x^2 - (4/9) * x - 53/27 };
        \end{axis}
        \end{tikzpicture}
        \caption{Courbe représentative de $\phi(x)$.}
    \end{figure}
    
\end{enumerate}
\section*{\underline{\textbf{\textcolor{red}{IV. Équations différentielles du 2nd ordre, à coefficients constants}}}}
C'est une équation de la forme

\textbf{(E)}: $ay''+by'+cy=g(x)$ ; $\neq 0$ (équation avec $2^{nd}$ membre)

\textbf{(E')}: $ay''+by'+cy=0$ ; $\neq 0$ (équation homogène)

$f$ est une solution de \textbf{(E):}(respectivement de \textbf{(E')}) sur \textbf{I} si $f$ est deux fois dérivables sur \textbf{I}  et $f$ vérifie \textbf{(E)}(respectivement $f$ vérifie \textbf{(E')})
\subsection*{\underline{\textbf{\textcolor{red}{1. Recherche de la solution générale de (E')}}}}
Soit $\textbf{(E':)}ay''+by'+cy=0$ avec $a\neq 0$, nous appelons équation\\\ caractéristique associée à \textbf{(E')} l'équation $ar^{2}+br+c=0$
\subsection*{\underline{\textbf{\textcolor{red}{Théorème}}}}
Soit \textbf{(E')}:$ay''+by'+cy=0$ et $ar^{2}+br+c=0$ équation caractéristique associée de discriminant $\Delta=b^{2}-4ac$, alors:
\begin{itemize}
    \item[$\bullet$] Si $\Delta >0$ alors, l'équation $ar^{2}+br+c=0$ admet deux solutions réelles distinctes \[r_{1}=\frac{-b-\sqrt{\Delta}}{2a}, r_{2}=\frac{-b+\sqrt{\Delta}}{2a}\]
    Dans ce cas, \textbf{(E')}:$ay''+by'+cy=0$ admet comme solution générale $y_{2}$ de la forme\\
    \[\textcolor{red}{ \boxed{ y_{2}(x)=Ae^{r_{1}x}+Be^{r_{2}x} } }\text{ Avec A et B des Constants déterminées par les conditions initales}\]
    \item[$\bullet$] Si $\Delta = 0$ alors, l'équation $ar^{2}+br+c=0$ admet une solutions double réelle \[r_{0}=\frac{-b}{2a}\]
    Et l'équation $(E')$ admet une solution générale de la forme 
    \[\textcolor{red}{ \boxed{ y_{2}(x)=(Ax+B)e^{r_{0}x} } }\]
    \item[$\bullet$] Si $\Delta <0$ alors, l'équation $ar^{2}+br+c=0$ admet deux solutions complexes conjuguées \[\alpha + i\beta, \alpha - i\beta\]
    Ainsi, \textbf{(E')}:$ay''+by'+cy=0$ admet une solution générale de la forme 
     \[\textcolor{red}{ \boxed{ y_{2}(x)=Ae^{\alpha x}(A\cos \beta x+B\sin \beta x) } }\]
\end{itemize}
\subsection*{\underline{\textbf{\textcolor{red}{Remarques}}}}
Dans tous les cas, les constantes $A$ et $B$ sont dépendantes des conditions initiales.
\subsection*{\underline{\textbf{\textcolor{red}{2. Recherche d'une solution particulière}}}}
\begin{itemize}
    \item[$\bullet$ ] Si $g$ est un polynôme de degré $n$, alors $f_{1}$ est aussi un polynôme de degré
    \item $n+1$ si $c=0$.
    \item $n$ si $c\neq 0$
    \item[$\bullet$ ] Si $g(x)$ est de la forme \( a\cos \beta x / a\sin \beta x \),\( a\cos \beta x + b\sin \beta x \) alors, $f_{1}(x)$ sera de la forme \( A\cos \beta x + B\sin \beta x \).
\end{itemize}
\subsection*{\underline{\textbf{\textcolor{red}{3. Solution générale de (E)}}}}
Soit $f_{2}(x)$ solution générale de \textbf{(E')}, si $f_{1}(x)$ donnée, est solution particulière de \textbf{(E)}, alors la solution générale de \textbf{(E)}; $y(x)$,sera de la forme \[y(x)=f_{2}(x)+f_{1}(x) \]
\subsection*{\underline{\textbf{\textcolor{red}{Exemple}}}}
Résoudre l'équation différentielle \textbf{(E:)}$y''-4y'+3y=0$ puis, déterminer et représenter graphiquement la solution qui passe par l'origine $O(0,0)$ du repère et dont la dérivée vaut 1 en 0.
\subsection*{\underline{\textbf{\textcolor{green}{Solution}}}}
\subsection*{\underline{\textbf{\textcolor{green}{1. Résolution de l'équation différentielle homogène}}}}
L'équation différentielle donnée est une équation linéaire homogène d'ordre 2 à coefficients constants :
\[
y'' - 4y' + 3y = 0
\]
Pour résoudre cette équation, nous cherchons les racines de l'équation caractéristique associée :
\[
r^2 - 4r + 3 = 0
\]
Cette équation se factorise facilement :
\[
(r - 3)(r - 1) = 0
\]
Les racines sont :
\[
r_1 = 3 \quad \text{et} \quad r_2 = 1
\]
Ainsi, la solution générale de l'équation différentielle est :
\[
y(x) = C_1 e^{3x} + C_2 e^x
\]

\subsection*{\underline{\textbf{\textcolor{green}{2. Détermination de la solution particulière}}}}
Nous devons maintenant déterminer les constantes $C_1$ et $C_2$ en utilisant les conditions initiales données :
\[
y(0) = 0 \quad \text{et} \quad y'(0) = 1
\]
Calculons $y'(x)$ à partir de la solution générale :
\[
y'(x) = 3C_1 e^{3x} + C_2 e^x
\]
En utilisant les conditions initiales :
\[
y(0) = C_1 e^0 + C_2 e^0 = C_1 + C_2 = 0 \quad \Rightarrow \quad C_2 = -C_1
\]
\[
y'(0) = 3C_1 e^0 + C_2 e^0 = 3C_1 + C_2 = 1
\]
Substituons $C_2 = -C_1$ dans la deuxième équation :
\[
3C_1 - C_1 = 1 \quad \Rightarrow \quad 2C_1 = 1 \quad \Rightarrow \quad C_1 = \frac{1}{2}
\]
\[
C_2 = -\frac{1}{2}
\]
La solution particulière est donc :
\[
y(x) = \frac{1}{2} e^{3x} - \frac{1}{2} e^x
\]

\subsection*{\underline{\textbf{\textcolor{green}{3. Représentation graphique}}}}

Pour représenter graphiquement la solution $y(x) = \frac{1}{2} e^{3x} - \frac{1}{2} e^x$, nous traçons la courbe dans un repère.

\begin{figure}[h!]
    \centering
    \begin{tikzpicture}
    \begin{axis}[
        axis lines = middle,
        xlabel = $x$,
        ylabel = {$y(x)$},
        domain=-8:1,
        xmin=-8, xmax=8,
        ymin=-8, ymax=8,
        samples=100,
        width=12cm,
        height=8cm,
        grid=both,
    ]
    \addplot [
        color=blue,
        thick
    ]
    { (1/2) * exp(3*x) - (1/2) * exp(x) };
    \end{axis}
    \end{tikzpicture}
    \caption{Courbe représentative de $y(x) = \frac{1}{2} e^{3x} - \frac{1}{2} e^x$.}
\end{figure}
\end{document}