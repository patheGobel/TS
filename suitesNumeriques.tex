\documentclass{article}
\usepackage{stmaryrd}
\usepackage{graphicx} % Pour l'insertion d'images
\usepackage{float}    % Pour contrôler précisément le placement
\usepackage[utf8]{inputenc}

\usepackage[french]{babel}
\usepackage[T1]{fontenc}
\usepackage{hyperref}
\usepackage{verbatim}

\usepackage{color, soul}

\usepackage{pgfplots}
\pgfplotsset{compat=1.15}
\usepackage{mathrsfs}

\usepackage{array}
\usepackage{geometry}
\usepackage{multirow}
\geometry{a4paper, margin=1in}

\usepackage{amsmath}
\usepackage{amsfonts}
\usepackage{amssymb}
\usepackage{tkz-tab}
\usepackage{mdframed}
\author{Destiné aux élèves de Terminale S\\Lycée de Dindéfelo\\Présenté par M. BA}
\title{\textbf{Suites numériques}}
\date{\today}
\usepackage{tikz}
\usetikzlibrary{arrows, shapes.geometric, fit}

% Commande pour la couleur d'accentuation
\newcommand{\myul}[2][black]{\setulcolor{#1}\ul{#2}\setulcolor{black}}
\newcommand\tab[1][1cm]{\hspace*{#1}}

%\usepackage[margin=2cm]{geometry}
\usepackage{eso-pic}         % Pour ajouter des éléments en arrière-plan

\usepackage{enumitem}
%---------------------------------------
% Définir un compteur pour les exemples
\newcounter{exemple}

% Définir la commande \exemple pour afficher un exemple numéroté
\newcommand{\exemple}{%
  \refstepcounter{exemple}% Incrémenter le compteur
  \textbf{\textcolor{orange}{Exemple \theexemple : }} \ignorespaces
}
%---------------------------------------
\newcounter{solution}

% Définir la commande \solutione pour afficher un solution numéroté
\newcommand{\solution}{%
  \refstepcounter{solution}% Incrémenter le compteur
  \textbf{\textcolor{orange}{Solution \thesolution : }} \ignorespaces
}
%---------------------------------------
\definecolor{myorange}{rgb}{1.0, 0.8, 0.0}

% Définir un compteur pour les exercices d'application
\newcounter{exerciceapp}

% Définir la commande pour afficher un exercice d'application numéroté
\newcommand{\exerciceapp}{%
  \refstepcounter{exerciceapp}%
  \textbf{\textcolor{myorange}{Exercice d'application \theexerciceapp :}} \ignorespaces
}
%--------------------------------------
% Définir un compteur pour les exercices d'application
\newcounter{correction}

% Définir la commande pour afficher un correction exercice d'application numéroté
\newcommand{\correction}{%
  \refstepcounter{correction}%
  \textbf{\textcolor{myorange}{Correction \thecorrection :}} \ignorespaces
}
%--------------------------------------
% Définir un compteur pour les remarque d'application
\newcounter{remarque}

%----------------------------------------
\definecolor{myorange1}{rgb}{1.0, 1.5, 0}
% Définir la commande pour afficher un remarque numéroté
\newcommand{\remarque}{%
  \refstepcounter{remarque}%
  \textbf{\textcolor{myorange1}{Remarque \theremarque :}} \ignorespaces
}

% Commande pour ajouter du texte en arrière-plan
\AddToShipoutPicture{
    \AtTextCenter{%
        \makebox[0pt]{\rotatebox{45}{\textcolor[gray]{0.9}{\fontsize{5cm}{5cm}\selectfont Pathé Gobel BA}}}
    }
}

\begin{document}
\maketitle

\section*{\underline{\textbf{\textcolor{red}{I. Généralités}}}}
\subsection*{\underline{\textbf{\textcolor{red}{1. Définition}}}}
On appelle \textit{suite numérique} toute fonction définie de $\mathbb{N}$ ou d'une partie $E$ de $\mathbb{N}$ vers $\mathbb{R}$.

On note : $U : \mathbb{N} \rightarrow \mathbb{R}$ \\
\hspace*{3.3cm}$n \mapsto U_{n}$

Le réel $U_n$ est appelé  \textcolor{red}{\textit{terme général}} ou  \textcolor{red}{\textit{terme d'indice $n$}}.
L'ensemble des termes de la suite est noté $(U_n)$ et $n \in \mathbb{N}$ ou $(U_{n})n\in\mathbb{N}$.
\subsection*{\underline{\textbf{\textcolor{red}{2. Modes de définition d'une suite }}}}
\subsection*{\underline{\textbf{\textcolor{red}{2.1. Suite explicite}}}}
\subsection*{\underline{\textbf{\textcolor{red}{2.2.Définition}}}}
Lorsqu'une suite $(U_{n})$ est exprimée en fonction de $n$, alors on dit que la suite $(U_{n})$ est définie par une \textcolor{red}{\textit{formule explicite}} et on note $U_{n} = f(n)$.

\textbf{\underline{\exemple}}

Soit $(u_{n})$ et $(v_{n})$ des suites définies par : $u_{n}=2n^{2}+1$ $v_{n}=\dfrac{n^{2}-4}{2n} n\in\mathbb{N}^{*}$

Calculer 

$u_{0}$ ; $u_{10}$ ; $u_{50}$

$v_{1}$ ; $v_{10}$ ; $v_{50}$

\textbf{\underline{\solution}}

\subsection*{\underline{\textbf{\textcolor{red}{3.Suite définie par récurrence}}}}
\subsection*{\underline{\textbf{\textcolor{red}{3.1.Définition}}}}	
Lorsque la suite $(U_n)$ est définie par une relation entre $U_n$ et $U_{n+1}$, alors on dit que $(U_n)$ est définie par une \textcolor{red}{\textit{relation de récurrence}}.

	$$\left\lbrace\begin{array}{rcl} u_{0}& =& \alpha \\ u_{n+1}& =& 2u_{n}-3\end{array}\right.\;,\quad\left\lbrace\begin{array}{rcl} v_{1}& =& \alpha \\ v_{n+1}& =& f(v_{n})\end{array}\right.$$

\textbf{NB:} Par exemple, pour calculer $u_{p}$, il faudrait faire $p$ calculs successifs.	
	
\textbf{\underline{\exemple}}

\begin{equation*} 
    \begin{cases}
        u_{0}=2 \\
        U_{n + 1}=\frac{2U_n + 3}{U_n + 1}
    \end{cases}
\end{equation*}
Calculer les cinq premier termes de $u_{n}$.

\textbf{\underline{\solution}}
	
\textbf{\underline{\exerciceapp}}

Dans chacun des cas suivants, calculer les $6$ premiers termes de la suite $U_{n}$.
\begin{itemize}
\item[1.]$U_{n} = 7n^{2} - 5n + 2$, $n \in \mathbb{N}$.

\item[2.]$u_{n+1}=2u_{n}+3\text{ et }u_{0}=-1.$
\end{itemize}

\textbf{\underline{\correction}}

\subsection*{\underline{\textbf{\textcolor{red}{4. Sens de variation d'une suite}}}}
\subsection*{\underline{\textbf{\textcolor{red}{Définition :}}}}
	
Une suite $\left(u_{n}\right)$ est dite :
	
$-\ $Croissante si : $\forall\;n\;,\ :\ u_{n+1}\geq u_{n} .$ c'est-à-dire, $\ u_{n+1}- u_{n}>0$
	
$-\ $Décroissante si : $\forall\;n;,\ :\ u_{n+1}\leq u_{n}$. c'est-à-dire, $\ u_{n+1}- u_{n}<0$

$-\ $Monotone si elle est croissante ou décroissante.

$-\ $Constante si : $\forall\;n\;,\ :\ u_{n+1}=u_{n}.$

Étudier le sens de variation d'une suite $\left(u_{n}\right)$

C'est dire si elle est croissante ou décroissante ou constante.

\subsection*{\underline{\textbf{\textcolor{red}{Règle :}}}}

Pour étudier le sens de variation d'une suite $\left(u_{n}\right)$,on compare deux termes consécutifs, pour cela, on peut étudier le signe de leur différence, ou, s'il s'agit de nombres strictement positifs, comparer leur quotient à $1.$

\textbf{\underline{\exemple}}

Soit la suite $\left(u_{n}\right)$ définie par : $u_{n}=\dfrac{n+2}{2n+1}$

Alors :

$u_{n+1}=\dfrac{(n+1)+2}{2(n+1)+1}=\dfrac{n+3}{2n+3}$
	
$u_{n+1}-u_{n}=\dfrac{n+3}{2n+3}-\dfrac{n+2}{2n+1}=\dfrac{-3}{(2n+1)(2n+3})$
	
Pour tout entier naturel $n$, on a donc : $u_{n+1}-u_{n} < 0.$
	
La suite étudiée est par conséquent décroissante.

\textbf{\underline{\solution}}

\section*{\underline{\textbf{\textcolor{red}{II. Démonstration par récurrence}}}}

La démonstration par récurrence suit les étapes suivantes :  

\textbf{1. Initialisation}  

On montre que la propriété \( P(n) \) est vraie pour une première valeur \( n = n_0 \).  
\[
P(n_0) \quad \text{est vraie.}
\]

\textbf{2. Hypothèse de récurrence}  

On suppose que la propriété est vraie pour un entier \( k \geq n_0 \).  
C'est ce qu'on appelle l'hypothèse de récurrence :  
\[
P(k) \quad \text{est vraie.}
\]

\textbf{3. Hérédité}  

On démontre que si la propriété est vraie pour \( n = k \), alors elle est aussi vraie pour \( n = k+1 \).  
Autrement dit, on montre que :  
\[
P(k) \implies P(k+1).
\]

\textbf{4. Conclusion}  

Si les deux étapes précédentes sont vérifiées, alors, par le principe de récurrence, la propriété est vraie pour tout \( n \geq n_0 \).  

\textbf{\underline{\exemple}}

Montrer que Pour tout \( n \geq 1 \)
\[
S(n) = 1 + 2 + \dots + n = \frac{n(n+1)}{2}.
\]

\textbf{\underline{\solution}}

\textbf{Propriété à démontrer} : Pour tout \( n \geq 1 \), la somme des \( n \) premiers entiers naturels est donnée par :
\[
S(n) = 1 + 2 + \dots + n = \frac{n(n+1)}{2}.
\]

\textbf{Étape 1 : Initialisation}  

Pour \( n = 1 \), on a :  
\[
S(1) = 1.
\]  
D'autre part :  
\[
\frac{1(1+1)}{2} = 1.
\]  
Donc, la propriété est vraie pour \( n = 1 \).  

\textbf{Étape 2 : Hypothèse de récurrence}  

Supposons que la propriété est vraie pour un entier \( k \geq 1 \), c'est-à-dire :  
\[
S(k) = \frac{k(k+1)}{2}.
\]  

\textbf{Étape 3 : Hérédité}  

Montrons que la propriété est vraie pour \( k+1 \).  
En utilisant la définition de \( S(k+1) \), on a :  
\[
S(k+1) = S(k) + (k+1).
\]  
En remplaçant \( S(k) \) par l'hypothèse de récurrence, on obtient :  
\[
S(k+1) = \frac{k(k+1)}{2} + (k+1).
\]  
Factorisons \( (k+1) \) :  
\[
S(k+1) = \frac{k(k+1) + 2(k+1)}{2} = \frac{(k+1)(k+2)}{2}.
\]  
Ainsi, la propriété est vraie pour \( k+1 \).  

\textbf{Conclusion}  

Par le principe de récurrence, la propriété est vraie pour tout \( n \geq 1 \).  


\textbf{\underline{\exerciceapp}}

Soit \( (u_n)_{n \in \mathbb{N}} \) la suite définie par :
\[
\begin{cases} 
u_{n+1} = \frac{1}{3} u_n + \frac{5}{3}, & \forall n \in \mathbb{N}, \\ 
u_0 = 1. & 
\end{cases}
\]

Montrer que : \( \forall n \in \mathbb{N}, \; u_n < \frac{5}{2} \).


Soit \( (u_n)_{n \in \mathbb{N}} \) la suite définie par :
\[
\begin{cases} 
u_{n+1} = \sqrt{12 + u_n}, & \forall n \in \mathbb{N}, \\ 
u_0 = 0. & 
\end{cases}
\]

\begin{enumerate}
    \item Montrer que : \( \forall n \in \mathbb{N}^*, \; 3 \leq u_n \leq 4 \).
\end{enumerate}
\textbf{\underline{\correction}}

\section*{\underline{\textbf{\textcolor{red}{III. Suites Minorée-Majorée-Bornée-Convergente-Divergente-adjacents}}}}
\subsection*{\underline{\textbf{\textcolor{red}{1. Suites Minorées}}}}

Une suite $(u_n)$ est dite \textbf{minorée} s'il existe un réel $m \in \mathbb{R}$ tel que :
\[
\forall n \in \mathbb{N}, \quad u_n \geq m.
\]
Dans ce cas, le nombre $m$ est appelé une \textbf{borne inférieure} de la suite $(u_n)$.

\vspace{0.5cm}
\textbf{\underline{\exemple}}\\
Considérons la suite définie par :
\[
u_n = \frac{1}{n}, \quad \text{pour } n \in \mathbb{N}^*.
\]
Étudions si cette suite est minorée.

\vspace{0.5cm}
\textbf{\underline{\solution}}\\
Pour tout $n \in \mathbb{N}^*$, nous avons $u_n = \frac{1}{n} > 0$. Cela signifie que $u_n$ est strictement positive pour tout entier naturel $n$. Ainsi, la suite $(u_n)$ est minorée par 0, car :
\[
\forall n \in \mathbb{N}^*, \quad u_n \geq 0.
\]

\subsection*{\underline{\textbf{\textcolor{red}{2. Suites Majorées}}}}

Une suite $(u_n)$ est dite \textbf{majorée} s'il existe un réel $M \in \mathbb{R}$ tel que :
\[
\forall n \in \mathbb{N}, \quad u_n \leq M.
\]
Dans ce cas, le nombre $M$ est appelé une \textbf{borne supérieure} de la suite $(u_n)$.

\vspace{0.5cm}
\textbf{\underline{\exemple}}\\
Considérons la suite définie par :
\[
u_n = 1 - \frac{1}{n}, \quad \text{pour } n \in \mathbb{N}^*.
\]
Étudions si cette suite est majorée.

\vspace{0.5cm}
\textbf{\underline{\solution}}\\
Pour tout $n \in \mathbb{N}^*$, nous avons $u_n = 1 - \frac{1}{n}$. Observons les propriétés suivantes :
\begin{itemize}
    \item Lorsque $n$ augmente, $\frac{1}{n}$ diminue, ce qui implique que $u_n$ croît et se rapproche de 1 sans jamais dépasser cette valeur.
    \item Ainsi, nous avons $u_n \leq 1$ pour tout $n \in \mathbb{N}^*$. La suite $(u_n)$ est donc majorée par $M = 1$.
\end{itemize}

\subsection*{\underline{\textbf{\textcolor{red}{3. Suites Bornées}}}}

Une suite $(u_n)$ est dite \textbf{bornée} s'il existe deux réels $m, M \in \mathbb{R}$ tels que :
\[
\forall n \in \mathbb{N}, \quad m \leq u_n \leq M.
\]
Dans ce cas, $m$ est une borne inférieure et $M$ une borne supérieure de la suite $(u_n)$. En d'autres termes, une suite bornée est à la fois majorée et minorée.

\vspace{0.5cm}
\textbf{\underline{\exemple}}\\
Considérons la suite définie par :
\[
u_n = (-1)^n \cdot \frac{1}{n}, \quad \text{pour } n \in \mathbb{N}^*.
\]
Étudions si cette suite est bornée.

\vspace{0.5cm}
\textbf{\underline{\solution}}\\
Pour tout $n \in \mathbb{N}^*$, la suite $(u_n)$ alterne les signes (car $(-1)^n$ change de signe à chaque terme) et sa valeur absolue est donnée par $\lvert u_n \rvert = \frac{1}{n}$, qui décroît vers $0$ lorsque $n \to \infty$.\\

Ainsi, nous avons :
\[
-\frac{1}{n} \leq u_n \leq \frac{1}{n}.
\]
Cela signifie que la suite $(u_n)$ est minorée par $-\frac{1}{1} = -1$ et majorée par $\frac{1}{1} = 1$ pour tout $n \geq 1$. Par conséquent, $(u_n)$ est une suite bornée.

\subsection*{\underline{\textbf{\textcolor{red}{4. Suites Convergente}}}}
\begin{itemize}
    \item Si $(u_n)_n$ est croissante et majorée, c-à-d $(u_n \leq M)$, alors $(u_n)_n$ est convergente vers un $l \in \mathbb{R}$.
    \item Si $(u_n)_n$ est décroissante et minorée, c-à-d $(u_n \geq m)$, alors $(u_n)_n$ est convergente vers un $l \in \mathbb{R}$.
\end{itemize}

\textbf{Remarque} La suite u est dite convergente si et seulement si $\lim_{n \to \infty} u_n = \ell \in \mathbb{R}$. On dit que u converge vers $\ell$

\textbf{\underline{\exemple}}

D'après l'exemple 7 \[
u_n = \frac{1}{n}, \quad \text{pour } n \in \mathbb{N}^*.
\] est minorée.
De plus, cette suite est décroissante, car pour $n < m$, nous avons $u_n > u_m$. Donc, la suite $(u_n)$ est minorée et décroissante, ce qui implique qu'elle converge vers sa borne inférieure, à savoir :
\[
\lim_{n \to \infty} u_n = 0.
\]

D'après l'exemple 8 \[
u_n = 1 - \frac{1}{n}, \quad \text{pour } n \in \mathbb{N}^*.
\] est donc majorée par $M = 1$.
Par conséquent, la suite $(u_n)$ est majorée et croissante. Elle converge vers sa borne supérieure, à savoir :
\[
\lim_{n \to \infty} u_n = 1.
\]

D'après l'exemple 9

De plus, lorsque $n \to \infty$, $\lvert u_n \rvert \to 0$, donc la suite converge vers $0$. Nous avons donc :
\[
\lim_{n \to \infty} u_n = 0.
\]

\textbf{\underline{\solution}}
\subsection*{\underline{\textbf{\textcolor{red}{5. Suites Divergente}}}}
\begin{itemize}
    \item Si $(u_n)_n$ est croissante et non majorée, alors $\lim_{n \to \infty} u_n = +\infty$. donc elle diverge
    \item Si $(u_n)_n$ est décroissante et non minorée, alors $\lim_{n \to \infty} u_n = -\infty$.donc elle diverge
\end{itemize}
\textbf{\underline{\exemple}}

La suite suivante est-elle $(u_n)_n=n-3+\frac{1}{n+1}$ est convergente ?

\textbf{\underline{\solution}}

$$\lim_{n\to \infty}u_n=\lim_{n\to \infty}(n-3+\frac{1}{n+1})$$

\subsection*{\underline{\textbf{\textcolor{red}{6. Théorème de la convergence}}}}
\begin{itemize}
\item Toute suite croissante et majorée est convergente .
\item Toute suite décroissante et minorée est convergente .
\end{itemize}
\subsection*{\underline{\textbf{\textcolor{red}{Propriété}}}}
\begin{itemize}
\item Toute suite croissante et non majorée a pour limite $+\infty$.
\item Toute suite décroissante et non minorée a pour limite $-\infty$.
\end{itemize}
\subsection*{\underline{\textbf{\textcolor{red}{7. Suites adjacents}}}}
Deux suites $u$ et $v$ sont adjacentes si et seulement si , l’une est croissante, l’autre est décroissante et la
limite de leur différence est égale à 0 . C'est-à-dire
$$
\begin{cases}
u \textbf{ est croissante}\\
v \textbf{ est décroissante}\\
\lim_{n\to \infty}(u_{n}-v_{n})=0
\end{cases} \textbf{Ou}
\begin{cases}
v \textbf{ est croissante}\\
u \textbf{ est décroissante}\\
\lim_{n\to \infty}(u_{n}-v_{n})=0
\end{cases}
$$

\textbf{\underline{\exemple}}

Soient deux suites définies par :

\[
u_n = 1-\frac{1}{n}, \quad v_n = 1+\frac{1}{n}.
\]

Montrer que les deux suite sont adjacents.

\textbf{\underline{\solution}}

\textbf{\underline{\exerciceapp}}
Soient les deux suites suivantes définies par récurrence pour tout $n \in \mathbb{N}$ :
\[
\begin{cases}
u_0 = 1, & v_0 = 2, \\
u_{n+1} = \frac{u_n + v_n}{2}, & v_{n+1} = \sqrt{u_n \cdot v_n}.
\end{cases}
\]
\textbf{\underline{\correction}}

%\textbf{\underline{Étude des propriétés :}}\\
%1. \textbf{Monotonie} :\\
%- La suite $(u_n)$ est \textbf{croissante}. En effet, pour tout $n \geq 0$ :
%\[
%u_{n+1} = \frac{u_n + v_n}{2} \quad \text{et} \quad u_n \leq v_n \implies u_{n+1} \geq u_n.
%\]
%
%- La suite $(v_n)$ est \textbf{décroissante}. En effet, pour tout $n \geq 0$ :
%\[
%v_{n+1} = \sqrt{u_n \cdot v_n} \quad \text{et} \quad u_n \leq v_n \implies v_{n+1} \leq v_n.
%\]
%
%2. \textbf{Encadrement} :\\
%Pour tout $n \geq 0$, nous avons $u_n \leq v_n$. Cette propriété est conservée par les relations de récurrence.
%
%3. \textbf{Convergence des suites adjacentes} :\\
%- Les suites $(u_n)$ et $(v_n)$ sont \textbf{bornées}.
%
%- $(u_n)$ est croissante et bornée, donc convergente.
%
%- $(v_n)$ est décroissante et bornée, donc convergente.
%
%Soit $L$ la limite commune des deux suites, alors :
%\[
%\lim_{n \to \infty} u_n = \lim_{n \to \infty} v_n = L.
%\]
%En prenant la limite dans les relations de récurrence :
%\[
%\begin{aligned}
%\lim_{n \to \infty} u_{n+1} = \lim_{n \to \infty} \frac{u_n + v_n}{2} \implies L = \frac{L + L}{2} \implies L = L,\\
%\lim_{n \to \infty} v_{n+1} = \lim_{n \to \infty} \sqrt{u_n \cdot v_n} \implies L = \sqrt{L \cdot L} \implies L = L.
%\end{aligned}
%\]
%Ainsi, $L$ est la limite commune de $(u_n)$ et $(v_n)$.
%
%\textbf{\underline{Conclusion}} :\\
%Les deux suites $(u_n)$ et $(v_n)$ sont adjacentes car :
%\begin{itemize}
%    \item $(u_n)$ est croissante et $(v_n)$ est décroissante,
%    \item $(u_n)$ et $(v_n)$ convergent vers la même limite $L$.
%\end{itemize}
%
%\subsection*{\underline{\textbf{\textcolor{red}{Exemple de deux suites adjacentes non définies par récurrence}}}}
%
%\textbf{\underline{Définition des suites}}\\
%Considérons les deux suites suivantes définies de manière explicite pour tout $n \in \mathbb{N}^*$ :
%\[
%u_n = \frac{1}{n}, \quad v_n = \frac{1}{n} + \frac{1}{n^2}.
%\]
%
%\textbf{\underline{Étude des propriétés :}}\\
%1. \textbf{Monotonie} :\\
%- La suite $(u_n)$ est \textbf{décroissante}. En effet, pour tout $n \geq 1$, nous avons :
%\[
%u_{n+1} = \frac{1}{n+1} < \frac{1}{n} = u_n.
%\]
%- La suite $(v_n)$ est également \textbf{décroissante}. En effet, pour tout $n \geq 1$ :
%\[
%v_{n+1} = \frac{1}{n+1} + \frac{1}{(n+1)^2} < \frac{1}{n} + \frac{1}{n^2} = v_n.
%\]
%
%2. \textbf{Encadrement} :\\
%Pour tout $n \in \mathbb{N}^*$, on a :
%\[
%u_n = \frac{1}{n} < v_n = \frac{1}{n} + \frac{1}{n^2}.
%\]
%Cela montre que $(u_n)$ et $(v_n)$ sont encadrées.
%
%3. \textbf{Convergence des suites adjacentes} :\\
%- La suite $(u_n)$ est décroissante et bornée inférieurement par $0$, donc elle converge.
%- La suite $(v_n)$ est également décroissante et bornée inférieurement par $0$, donc elle converge.
%
%Lorsque $n \to \infty$, les deux suites $(u_n)$ et $(v_n)$ ont pour limite commune $0$, car :
%\[
%\lim_{n \to \infty} u_n = \lim_{n \to \infty} \frac{1}{n} = 0,
%\]
%\[
%\lim_{n \to \infty} v_n = \lim_{n \to \infty} \left(\frac{1}{n} + \frac{1}{n^2}\right) = 0.
%\]
%
%\textbf{\underline{Conclusion}} :\\
%Les suites $(u_n)$ et $(v_n)$ sont adjacentes car :
%\begin{itemize}
%    \item $(u_n)$ est décroissante et $(v_n)$ est également décroissante,
%    \item $(u_n)$ et $(v_n)$ convergent vers la même limite $0$,
%    \item À chaque étape, $(u_n)$ est strictement inférieure à $(v_n)$.
%\end{itemize}
%
%
%++++++++++++++++++++++++++++++++++++++++++++++++++++++++++++++++++++++++++++++++++++
%
%\textbf{\underline{\solution}}
%
%++++++++++++++++++++++++++++++++++++++++++++++++++++++++++++++++++++++++++++++++++++

%\subsection*{\underline{\textbf{\textcolor{red}{3. Déterminer le sens de variation d'une suite définie par } $u_{n+1} = f(u_n)$}}}
%
%\textbf{\underline{Méthode :}}\\
%Pour déterminer le sens de variation de la suite $(u_n)$ définie par $u_{n+1} = f(u_n)$ avec \( f \) continue :
%\begin{enumerate}
%    \item Vérifiez si \( f \) est une fonction \textbf{croissante} ou \textbf{décroissante} sur l'intervalle où \( u_n \) prend ses valeurs. Cela donne des indications sur le comportement global de la suite.
%    \item Calculez \( u_{n+1} - u_n = f(u_n) - u_n \) :
%    \begin{itemize}
%        \item Si \( f(u_n) - u_n \geq 0 \), alors la suite est croissante.
%        \item Si \( f(u_n) - u_n \leq 0 \), alors la suite est décroissante.
%    \end{itemize}
%    \item Vérifiez les conditions initiales pour confirmer le comportement de la suite dès ses premiers termes.
%\end{enumerate}
%
%---
%
%\subsection*{\underline{\textbf{\textcolor{red}{4. Déterminer la convergence d'une suite définie par } $u_{n+1} = f(u_n)$}}}
%
%\textbf{\underline{Méthode :}}\\
%Pour étudier la convergence d'une suite $(u_n)$ définie par $u_{n+1} = f(u_n)$ avec \( f \) continue :
%\begin{enumerate}
%    \item Montrez que la suite est \textbf{monotone} (croissante ou décroissante) en déterminant son sens de variation (voir point précédent).
%    \item Vérifiez si la suite est \textbf{bornée} :
%    \begin{itemize}
%        \item Si la suite est à la fois monotone et bornée, alors elle est convergente (théorème de la convergence des suites monotones).
%    \end{itemize}
%\end{enumerate}
%
%---
%
%\subsection*{\underline{\textbf{\textcolor{red}{5. Déterminer la limite d'une suite définie par } $u_{n+1} = f(u_n)$}}}
%
%\textbf{\underline{Méthode :}}\\
%Pour déterminer la limite \( L \) d'une suite convergente $(u_n)$ définie par $u_{n+1} = f(u_n)$ avec \( f \) continue :
%\begin{enumerate}
%    \item Si la suite converge vers une limite \( L \), alors :
%    \[
%    \lim_{n \to \infty} u_n = L \quad \text{implique} \quad L = f(L).
%    \]
%    Trouvez les solutions de l'équation \( L = f(L) \) (les points fixes de la fonction \( f \)).
%    \item Vérifiez si la suite converge vers l'une des solutions trouvées en étudiant la stabilité des points fixes :
%    \begin{itemize}
%        \item Si \( f'(L) \) (la dérivée de \( f \) au point \( L \)) satisfait \( |f'(L)| < 1 \), alors \( L \) est un point fixe attractif et la suite converge vers \( L \).
%        \item Si \( |f'(L)| > 1 \), alors \( L \) est un point fixe répulsif, et la suite ne peut pas converger vers \( L \).
%    \end{itemize}
%\end{enumerate}
%
%---
%
%\textbf{\underline{Exemple :}}\\
%Considérons la suite définie par \( u_{n+1} = \frac{1}{2} u_n + 1 \) avec \( u_0 = 2 \) :
%\begin{itemize}
%    \item Étudions son sens de variation :
%    \[
%    f(u) = \frac{1}{2}u + 1, \quad f(u) - u = -\frac{1}{2}u + 1.
%    \]
%    Si \( u > 2 \), alors \( f(u) - u < 0 \) (la suite est décroissante).
%    Si \( u < 2 \), alors \( f(u) - u > 0 \) (la suite est croissante).
%    \item Montrons que la suite est bornée et converge :
%    \( f(u) = \frac{1}{2}u + 1 \) admet un point fixe \( L \) tel que :
%    \[
%    L = \frac{1}{2}L + 1 \implies L = 2.
%    \]
%    La suite converge donc vers \( L = 2 \).
%\end{itemize}
%
%
%++++++++++++++++++++++++++++++++++++++++++++++++++++++++++++++++++++++++++++++++++++++++++++++++++++

\subsection*{\textcolor{red}{\underline{7.Théorème du point fixe } }}
\[
\begin{cases}
(u_{n}) \text{ est une suite qui converge vers un réel } \ell \\
f \text{ une fonction définie sur un intervalle } I \text{ et continue en } \ell \\
\text{ Pour tout entier naturel } n, u_{n}\in I
\end{cases}
\textbf{Alors} \lim_{x \to +\infty} f(u_{n})=f(\ell)
\]
\textbf{\underline{\exemple}}

$(u_n)$ est la suite définie par $u_0=0$ et pour tout entier naturel $n$,$u_n+1=\frac{1}{2}u_n+1$.

a)Montrer que pour tout entier naturel $n$,$u_n\leq u_{n+1}\leq 2$

b)En déduire que $(u_n)$ est convergente. On note $\ell$ sa limite.

c) Déterminer la valeur de $\ell$

\textbf{\underline{\solution}}
\section*{\underline{\textbf{\textcolor{red}{IV. Représentation graphique des termes d'une suite}}}}
\subsection*{\underline{\textbf{\textcolor{red}{1. Suite explicite}}}}
$-\ $ Si $u_{n}$ est définie de façon explicite ; $u_{n}=f(n)$ alors représenter graphiquement la suite $(u_{n})$ consiste à représenter dans un repère l'ensemble des points isolés $(n\;,\ u_{n}).$

$-\ $ On peut aussi représenter directement les valeurs des termes de la suite sur l'un des axes du repère.

\textbf{\underline{\exemple}}

Représenter les 6 premiers termes de la suite $(u_{n})$ définie par 

a) $u_{n}=\dfrac{n^{2}+1}{2n+3}$

\textbf{\underline{\solution}}\\
a)\\
\begin{tikzpicture}
    \begin{axis}[
        axis lines = middle,
        xlabel = \( n \),
        ylabel = \( u_n \),
        ymin = 0,
        ymax = 3,
        xmin = -0.5,
        xmax = 6.5,
        xtick = {0, 1, 2, 3, 4, 5},
        ytick = {0, 0.5, 1, 1.5, 2, 2.5, 3},
        grid = both,
        width=10cm,
        height=8cm
    ]
        % Points de la suite
        \addplot[only marks, mark=square*, blue] coordinates {
            (0, 1/3)
            (1, 2/5)
            (2, 5/7)
            (3, 10/9)
            (4, 17/11)
            (5, 2)
        };
    \end{axis}
\end{tikzpicture}

b)\\
La suite \( u_n = 2n - 7 \) est représentée par les points \( (n, u_n) \) pour \( n = 0, 1, 2, 3, 4, 5 \).

\begin{tikzpicture}
    \begin{axis}[
        axis lines = middle,
        xlabel = \( n \),
        ylabel = \( u_n \),
        ymin = -8,
        ymax = 4,
        xmin = -0.5,
        xmax = 5.5,
        xtick = {0, 1, 2, 3, 4, 5},
        ytick = {-8, -6, -4, -2, 0, 2, 4},
        grid = both,
        width=10cm,
        height=8cm
    ]
        % Points de la suite
        \addplot[only marks, mark=*, red] coordinates {
            (0, -7)
            (1, -5)
            (2, -3)
            (3, -1)
            (4, 1)
            (5, 3)
        };
    \end{axis}
\end{tikzpicture}

\subsection*{\underline{\textbf{\textcolor{red}{2. Suite définie par récurrence}}}} 

\[
\begin{cases}
u_0 = \alpha, \\
u_{n+1} = g(u_n),
\end{cases}
\]
où \( g \) est la fonction associée à la suite \((u_n)\).On trace \( C_g \), ainsi que la première bissectrice \( y = x \).  

Par une méthode graphique de projection, on construit les termes successifs de la suite \((u_n)\) en suivant les étapes suivantes :  

\begin{enumerate}
    \item On place le premier terme \( u_0 \) sur l’axe des abscisses.  
    \item On lit la valeur de \( u_1 = g(u_0) \) en projetant \( u_0 \) verticalement sur \( C_g \). Cette valeur se trouve sur l’axe des ordonnées.  
    \item On projette \( u_1 \) sur l’axe des abscisses à l’aide de la première bissectrice \( y = x \).  
    \item On utilise à nouveau la courbe \( C_g \) pour déterminer \( u_2 = g(u_1) \), en projetant \( u_1 \) verticalement sur \( C_g \).  
    \item On projette \( u_2 \) sur l’axe des abscisses via la première bissectrice.  
    \item On répète ce processus pour calculer les termes suivants \( u_3, u_4, \dots \).  
\end{enumerate}

Ce procédé, appelé *construction par itération graphique*, permet de visualiser l'évolution des termes de la suite \((u_n)\) et d'étudier son comportement (convergence, divergence ou oscillation).

\textbf{\underline{\exemple}}

Soit $(u_n)$ la suite définie par, $u_0=2$ et, pour tou $n\in\mathbb{N} \quad u_{n+1}=2\sqrt{u_{n}}$.

\textbf{\underline{\solution}}

Soit $f$ la fonction définie sur $[0;+\infty[$ par $f:x\mapsto 2\sqrt{x}$.
Ainsi, pour tout $n\in\mathbb{N}, u_{n+1}=f(u_{n})$ 

1.On trace $(\mathcal{C}_{f})$, la courbe représentation de la fonction $f$.

2.On trace la droite la d'équation $y=x$

\textbf{\underline{\exerciceapp}}

b) $u_{n}=2n-7$

Soit \( (u_n)_{n \in \mathbb{N}} \) la suite définie par :
\[
\begin{cases} 
u_{n+1} = \sqrt{5u_{n}} + 1 \\ 
u_0 = 1. & 
\end{cases}
\]

\textbf{\underline{\correction}}

\subsection*{\underline{\textbf{\textcolor{red}{V. Etude de suites définie par récurrence}}}}

\subsection*{\textcolor{red}{\underline{EXERCICE 3 } (BAC 2023) }}
\[
\text{On considère la suite numérique } (u_n)_{n \in \mathbb{N}} \text{ définie par :}
\begin{cases}
U_{0}=6\\
U_{n+1}=\frac{1}{U_{n}}+\frac{3}{4}U_{n}, n\in\mathbb{N}
\end{cases}
\]
\begin{enumerate}
\item[1)] Calculer \( u_1 \) et \( u_2 \).\hfill\textbf{(0.5pt)}

\item[2)] Démontrer par récurrence que: \( \forall n\in\mathbb{N},\quad u_n \geq \sqrt{3} \).\hfill\textbf{(01pt)}

\item[3)] Soit $f$ la fonction définie sur $]0, +\infty[$ par \( f(x) = \frac{1}{x} + \frac{3}{4}x \).
\begin{enumerate}
\item[a)] Etudier le sens de variations de $f$.\hfill\textbf{(01pt)}
\item[b)] En déduire par récurrence que $(u_n)_{n \in \mathbb{N}}$ est strictement décroissante.\hfill\textbf{(0.5pt)}
\end{enumerate}

\item[4)] Montrer que $(u_n)_{n \in \mathbb{N}}$ est convergente et déterminer sa limite.\hfill\textbf{(01pt)}

\end{enumerate}
\subsection*{\underline{\textbf{\textcolor{red}{VI. Suite arithmétiques-géométriques}}}}
\subsection*{\underline{\textbf{\textcolor{red}{1. Suite arithmétiques}}}}
Une suite $\left(u_{n}\right)$ est \textcolor{red}{arithmétique} si chaque terme s'obtient en ajoutant au précédent un même nombre \textcolor{red}{$r$} appelé raison : 
\textcolor{red}{$u_{n+1}=u_{n}+r.$} 
\subsection*{\underline{\textbf{\textcolor{red}{a.Expression du terme général}}}}		
$\bullet\ \text{Si }\left(u_{n}\right)$ est une suite arithmétique de premier terme $u_{0}$ et de raison $r$, alors :
	
$$u_{n}=u_{0}+nr$$
	
$\bullet\ $Si le premier terme est $u_{1}$, alors :
$$u_{n}=u_{1}+(n-1)r$$

$\bullet\ $ Si $(U_n)$ une suite arithmétique de raison $r$ et de premier terme $U_{p}$ alors on a : 
\begin{mdframed}[linecolor=red] % Définit la couleur du cadre
    \[
    \color{red} u_{n} = u_{p} + (n-p)r
    \]
\end{mdframed}
Avec $u_{p}$ le premier terme, p l'indice du premier terme et r la raison

\textbf{\underline{\exemple}}

Dans chacun des cas suivants, donner le terme général de la suite.
\begin{itemize}
\item[1.]$(u_n)$ est une suite arithmétique de raison $r = 3$ et de premier terme $u_{0}= 7$.
\item[2.]$(u_n)$ est une suite arithmétique de raison $r=-\frac{3}{4}$ et de premier terme 
$u_{1}=\frac{-7}{4}$.
\item[3.] $(U_n)$ est une suite arithmétique de raison $r$ et de premier terme $U_{0}$ tels que $U_{50} = 406$ et $U_{100} = 806$.
\end{itemize}
\textbf{\underline{\solution}}

\subsection*{\underline{\textbf{\textcolor{red}{A retenir 1 }}}}
Pour montrer qu'une suite $(U_n)$ est arithmétique de raison $r$, il suffit de montrer que :
\[ U_{n+1} - U_n = r \]

\textbf{\underline{\exemple}}

On considère la suite $(U_n)$ définie par :\[ U_n = 5n + 3 \]
Montrer que $(U_n)$ est une suite arithmétique dont on précisera la raison et le premier terme.

\textbf{\underline{\solution}}

\subsection*{\underline{\textbf{\textcolor{red}{A retenir 2}}}}
Pour montrer qu’une suite $(U_n)$ n’est pas arithmétique , il suffit de montrer que : 
$U_{1}$-$U_{0}\neq U_{2}$-$U_{1}$.

\textbf{\underline{\exemple}}

Soit la suite $(U_n)$ définie par : $U_{n}=\frac{n}{n+1}$

Montrer que $(U_n)$ n’est pas une suite arithmétique.

\textbf{\underline{\solution}}

\subsection*{\underline{\textbf{\textcolor{red}{b.Somme des premiers termes}}}}	
Soit (Un) une suite arithmétique.

Pour tous entiers naturels n et p tels que $p \leq n$, on a :

$\bullet\ $Si la suite a pour premier terme $u_{0}$, alors la somme $S_{n}=u_{0}+u_{1}+\ldots+u_{n}$ vaut :
	
$$S_{n}=\dfrac{(n+1)\left(u_{0}+u_{n}\right)}{2}$$

$\bullet\ $Si la suite a pour premier terme $u_{1}$, alors la somme $S_{n}=u_{1}+u_{1}+\ldots+u_{n}$ vaut :

$$S_{n}=\dfrac{n(u_{1}+u_{n})}{2}$$

$\bullet\ $De façon général si la suite a pour premier terme $u_{p}$, alors la somme $S_{n}=u_{p}+u_{p+1}+\ldots+u_{n}$ vaut :

\begin{mdframed}[linecolor=red] % Définit la couleur du cadre
    \[
    \color{red} S_{n}=\dfrac{(n-p+1)(u_{p}+u_{n})}{2}
    \]
\end{mdframed}

\textbf{NB} la Somme $S_{n}=u_{0}+u_{1}+\ldots+u_{n}$ peut etre notée par $S_{n}=\sum_{k=0}^{n}u_{k}$.

$$\sum_{k=0}^{4}2k+7 = \cdots$$

$$\sum_{k=0}^{10}5 = \cdots$$


\textbf{\underline{\exemple}}

Soit $\left(u_{n}\right)$ la suite définie par : $u_{n}=2n+7$
\begin{itemize}
\item[1.]Calculer $U_{1}$, $U_{2}$ et $U_{2024}.$
\item[2.] Montrer que (Un) est une suite arithmétique.
\item[3.] Calculer la somme : $S_{n} = U_{0} + U_{1} + \cdots+ U_{n}.$
\item[4.] En déduire le somme $S = U_{1} + U_{2} +\cdots+ U_{2024}.$
\end{itemize}

\textbf{\underline{\solution}}

\subsection*{\underline{\textbf{\textcolor{red}{2. Suites géométriques}}}}
Une suite $\left(u_{n}\right)$ est dite géométrie si chaque terme s'obtient en multipliant le précédent par un même nombre $q$ appelé raison : $u_{n+1}=u_{n}\times q.$
\subsection*{\underline{\textbf{\textcolor{red}{a. Expression du terme général}}}}
$\bullet\ $Si la suite géométrique $\left(u_{n}\right)$ a pour premier terme $u_{0}$ et pour raison $q$, alors : $$u_{n}=u_{0}\times q^{n}$$

$\bullet\ $Si le premier terme est $u_{1}$, alors :$$u_{n}=u_{1}\times q^{n-1}$$

$\bullet\ $ Si le premier terme est $u_{p}$, alors:
\begin{mdframed}[linecolor=red] % Définit la couleur du cadre
    \[
    \color{red} u_{n}=u_{p}\times q^{n-p}
    \]
\end{mdframed}

\subsection*{\underline{\textbf{\textcolor{red}{A retenir }}}}
Pour montrer qu’une suite $u_{n}$ est géométrique de raison q, il suffit de montrer que :
\begin{mdframed}[linecolor=red] % Définit la couleur du cadre
    \[
    \color{red} \dfrac{U_{n+1}}{U_{n}}=q
    \]
\end{mdframed}

\textbf{\underline{\exemple}}

montrer que $(v_{n})$ est une suite géométrique dont on précisera
la raison et le premier terme puis exprimer $(v_{n})$ en fonction de $n$.

\begin{equation*} 
    \begin{cases}
        U_{0}=5 \\
        U_{n + 1}=2U_{n}+3
    \end{cases}
    et\quad V_{n} = U_{n} + 3
\end{equation*}


\textbf{\underline{\exerciceapp}}

Dans chacun des cas suivants, montrer que $(v_{n})$ est une suite géométrique dont on précisera
la raison et le premier terme puis exprimer $(v_{n})$ en fonction de $n$.

\begin{itemize}

\item[1.]$v_{n} = 3 \times 5^{n}$

\item[2.]\begin{equation*} 
    \begin{cases}
        U_{1}=0 \\
        U_{n + 1}=-\frac{2}{3}U_{n}+1
    \end{cases}
    et\quad V_{n} = U_{n} - \frac{3}{5}
\end{equation*}

\end{itemize}

\textbf{\underline{\correction}}

\subsection*{\underline{\textbf{\textcolor{red}{b. Somme des premiers termes }}}}

Pour toute suite géométrique, de raison $q\neq 1$, on a:

$$S_{n}=u_{0}+u_{1}+\ldots+u_{n}=u_{0}\times\dfrac{1-q^{n+1}}{1-q}$$

$$S_{n}u_{1}+u_{2}+\ldots+u_{n}=u_{1}\times\dfrac{1-q^{n}}{1-q}$$

$$S_{n}=u_{p}+u_{p+1}+\ldots+u_{n}=u_{p}\times\dfrac{1-q^{n-p+1}}{1-q}$$

\begin{mdframed}[linecolor=red] % Définit la couleur du cadre
    \[
    \color{red} S_{n}=u_{p}\times\dfrac{1-q^{n-p+1}}{1-q}
    \]
\end{mdframed}
\section*{\underline{\textbf{\textcolor{red}{c. Sens de variation}}}}
    \begin{itemize}
        \item Si $0 < q < 1$, la suite $\left(u_{n}\right)$ est décroissante.
        \item Si $q > 1$, la suite $\left(u_{n}\right)$ est croissante.
        \item Si $q=1$, la suite $\left(u_{n}\right)$ est constante.
    \end{itemize}


\subsection*{\underline{\textbf{\textcolor{red}{Remarque :}}}}
Soit $q$ un nombre réel.
Soit \( q \) un nombre réel.

\begin{itemize}
    \item Si \( q > 1 \), alors
    \[
    \lim_{n \rightarrow +\infty} q^n = +\infty.
    \]
    
    \item Si \( -1 < q < 1 \), alors
    \[
    \lim_{n \rightarrow +\infty} q^n = 0.
    \]
\end{itemize}

\end{document}
