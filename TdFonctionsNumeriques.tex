\documentclass[a4paper,12pt]{article}
\usepackage{graphicx}
\usepackage[utf8]{inputenc}
\usepackage[T1]{fontenc}
\usepackage{geometry}
\usepackage{multicol}

\geometry{top=2cm, bottom=2cm, left=2cm, right=2cm}

\begin{document}

\noindent
\begin{minipage}[t]{0.48\textwidth}
\raggedright
\textbf{Ministère de l'Éducation Nationale}\\
Inspection Académique de Kédougou\\
Lycée Dindéfelo\\
Cellule de Mathématiques
\end{minipage}
\hfill
\begin{minipage}[t]{0.48\textwidth}
\raggedleft
\textbf{Année scolaire 2024-2025}\\
Date : 17/10/2024\\
Classe : Terminale S2\\
Professeur : M. BA
\end{minipage}

\vspace{1cm}
\section*{Exercice 1}

1. Déterminer, si elles existent, les limites suivantes :

\[
\lim_{x \to 0} \frac{\tan(3x)}{1 - \sqrt{x + 1}} \quad ; \quad \lim_{x \to 2} \frac{\sqrt{x+2} - \sqrt{3x+3}}{x-2} \quad ; \quad \lim_{x \to 0} \frac{\cos x - 1}{x^3 + x^2}
\]

\[
\lim_{x \to 0} \frac{3 \sin^2 x - \cos x + 1}{x^2} \quad ; \quad \lim_{x \to 0} \frac{\sin x - \tan x}{x^3}\quad ; \quad\lim_{x \to 0} \frac{1 - \cos x}{x \tan x}
\]

2. Déterminer, si elles existent, les limites suivantes :

\[
\lim_{x \to \frac{\pi}{2}} \frac{1 - \tan x}{1 + \tan x}\quad ; \quad \lim_{x \to 3} \frac{\sqrt{x+1} - 2}{\sqrt{x^2 - x - 6}}\quad ; \quad\lim_{x \to 1} \frac{\sqrt{x+3} - \sqrt{5-x}}{\sqrt{2x+7} - \sqrt{10-x}}
\]
\[
\lim_{x \to 0} \frac{\sqrt{1+\sin x -1}}{\sin 2x}\quad ; \quad\lim_{x \to \frac{\pi}{2}} \frac{\cos x}{1-\sin x}
\]

3. Déterminer, si elles existent, les limites suivantes :

\[
\lim_{x \to \pm \infty} \left(x - \sqrt{x^2 + 3x - 1}\right)\quad; \quad \lim_{x \to \pm \infty} \frac{2x + \sqrt{x^2 + 1}}{x}\quad; \quad
\lim_{x \to \pm \infty} \frac{x+1}{\sqrt{x+3}-2}
\]

\[
\lim_{x \to \pm \infty} \left(2x + \sqrt{x - 1}\right)\quad; \quad
\lim_{x \to \pm \infty} \left(x + \sqrt{2x^{2} - 5x + 1 }\right)\quad; \quad
\lim_{x \to \pm \infty} \left(x\sqrt{\frac{x+1}{x-1}}\right)
\]

\[
\lim_{x \to \pm \infty} \frac{x - \sqrt{|x|}}{3x + 2}
\]

4. Déterminer, si elles existent, les limites suivantes :
\[
\lim_{x \to \pm \infty} \frac{x^2 + \sin x}{x} \quad ; \quad \lim_{x \to \pm\infty} \frac{x + \sin x}{2 - \sin x} \quad ; \quad \lim_{x \to \pm \infty} \frac{x^3}{3 + 2 \sin x}
\]

5. Déterminer les réels \( a \) et \( b \) pour que la droite d'équation \( y = ax + b \) soit une asymptote à la courbe représentative de la fonction \( x \mapsto \sqrt{x^2 + 4} - \frac{x}{\sqrt{2}} \).


\end{document}
