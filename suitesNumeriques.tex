\documentclass{article}
\usepackage{stmaryrd}
\usepackage{graphicx} % Pour l'insertion d'images
\usepackage{float}    % Pour contrôler précisément le placement
\usepackage[utf8]{inputenc}

\usepackage[french]{babel}
\usepackage[T1]{fontenc}
\usepackage{hyperref}
\usepackage{verbatim}

\usepackage{color, soul}

\usepackage{pgfplots}
\pgfplotsset{compat=1.15}
\usepackage{mathrsfs}

\usepackage{array}
\usepackage{geometry}
\usepackage{multirow}
\geometry{a4paper, margin=1in}

\usepackage{amsmath}
\usepackage{amsfonts}
\usepackage{amssymb}
\usepackage{tkz-tab}
\usepackage{mdframed}
\author{Destiné aux élèves de Terminale S\\Lycée de Dindéfelo\\Présenté par M. BA}
\title{\textbf{Suites numériques}}
\date{\today}
\usepackage{tikz}
\usetikzlibrary{arrows, shapes.geometric, fit}

% Commande pour la couleur d'accentuation
\newcommand{\myul}[2][black]{\setulcolor{#1}\ul{#2}\setulcolor{black}}
\newcommand\tab[1][1cm]{\hspace*{#1}}

%\usepackage[margin=2cm]{geometry}
\usepackage{eso-pic}         % Pour ajouter des éléments en arrière-plan

\usepackage{enumitem}
%---------------------------------------
% Définir un compteur pour les exemples
\newcounter{exemple}

% Définir la commande \exemple pour afficher un exemple numéroté
\newcommand{\exemple}{%
  \refstepcounter{exemple}% Incrémenter le compteur
  \textbf{\textcolor{orange}{Exemple \theexemple : }} \ignorespaces
}
%---------------------------------------
\newcounter{solution}

% Définir la commande \solutione pour afficher un solution numéroté
\newcommand{\solution}{%
  \refstepcounter{solution}% Incrémenter le compteur
  \textbf{\textcolor{orange}{Solution \thesolution : }} \ignorespaces
}
%---------------------------------------
\definecolor{myorange}{rgb}{1.0, 0.8, 0.0}

% Définir un compteur pour les exercices d'application
\newcounter{exerciceapp}

% Définir la commande pour afficher un exercice d'application numéroté
\newcommand{\exerciceapp}{%
  \refstepcounter{exerciceapp}%
  \textbf{\textcolor{myorange}{Exercice d'application \theexerciceapp :}} \ignorespaces
}
%--------------------------------------
% Définir un compteur pour les exercices d'application
\newcounter{correction}

% Définir la commande pour afficher un correction exercice d'application numéroté
\newcommand{\correction}{%
  \refstepcounter{correction}%
  \textbf{\textcolor{myorange}{Correction \thecorrection :}} \ignorespaces
}
%--------------------------------------
% Définir un compteur pour les remarque d'application
\newcounter{remarque}

%----------------------------------------
\definecolor{myorange1}{rgb}{1.0, 1.5, 0}
% Définir la commande pour afficher un remarque numéroté
\newcommand{\remarque}{%
  \refstepcounter{remarque}%
  \textbf{\textcolor{myorange1}{Remarque \theremarque :}} \ignorespaces
}
% Commande pour ajouter du texte en arrière-plan
\AddToShipoutPicture{
    \AtTextCenter{%
        \makebox[0pt]{\rotatebox{45}{\textcolor[gray]{0.9}{\fontsize{5cm}{5cm}\selectfont Pathé Gobel BA}}}
    }
}

\begin{document}
\maketitle

\section*{\underline{\textbf{\textcolor{red}{I. Généralités}}}}
\subsection*{\underline{\textbf{\textcolor{red}{1. Définition}}}}
On appelle \textit{suite numérique} toute fonction définie de $\mathbb{N}$ ou d'une partie $E$ de $\mathbb{N}$ vers $\mathbb{R}$.

On note : $U : \mathbb{N} \rightarrow \mathbb{R}$ \\
\hspace*{3.3cm}$n \mapsto U_{n}$

Le réel $U_n$ est appelé  \textcolor{red}{\textit{terme général}} ou  \textcolor{red}{\textit{terme d'indice $n$}}.
L'ensemble des termes de la suite est noté $(U_n)$ et $n \in \mathbb{N}$ ou $(U_{n})n\in\mathbb{N}$.
\subsection*{\underline{\textbf{\textcolor{red}{2. Modes de définition d'une suite }}}}
\subsection*{\underline{\textbf{\textcolor{red}{2.1. Suite explicite}}}}
\subsection*{\underline{\textbf{\textcolor{red}{2.2.Définition}}}}
Lorsqu'une suite $(U_{n})$ est exprimée en fonction de $n$, alors on dit que la suite $(U_{n})$ est définie par une \textcolor{red}{\textit{formule explicite}} et on note $U_{n} = f(n)$.

\textbf{\underline{\exemple}}

Soit $(u_{n})$ et $(v_{n})$ des suites définies par : $u_{n}=2n^{2}+1$ $v_{n}=\dfrac{n^{2}-4}{2n} n\in\mathbb{N}^{*}$

Calculer 

$u_{0}$ ; $u_{10}$ ; $u_{50}$

$v_{1}$ ; $v_{10}$ ; $v_{50}$

\textbf{\underline{\solution}}

\subsection*{\underline{\textbf{\textcolor{red}{3.Suite définie par récurrence}}}}
\subsection*{\underline{\textbf{\textcolor{red}{3.1.Définition}}}}	
Lorsque la suite $(U_n)$ est définie par une relation entre $U_n$ et $U_{n+1}$, alors on dit que $(U_n)$ est définie par une \textcolor{red}{\textit{relation de récurrence}}.

	$$\left\lbrace\begin{array}{rcl} u_{0}& =& \alpha \\ u_{n+1}& =& 2u_{n}-3\end{array}\right.\;,\quad\left\lbrace\begin{array}{rcl} v_{1}& =& \alpha \\ v_{n+1}& =& f(v_{n})\end{array}\right.$$

\textbf{NB:} Par exemple, pour calculer $u_{p}$, il faudrait faire $p$ calculs successifs.	
	
\textbf{\underline{\exemple}}

\begin{equation*} 
    \begin{cases}
        u_{0}=2 \\
        U_{n + 1}=\frac{2U_n + 3}{U_n + 1}
    \end{cases}
\end{equation*}
Calculer les cinq premier termes de $u_{n}$.

\textbf{\underline{\solution}}
	
\textbf{\underline{\exerciceapp}}

Dans chacun des cas suivants, calculer les $6$ premiers termes de la suite $U_{n}$.
\begin{itemize}
\item[1.]$U_{n} = 7n^{2} - 5n + 2$, $n \in \mathbb{N}$.

\item[2.]$u_{n+1}=2u_{n}+3\text{ et }u_{0}=-1.$
\end{itemize}
\subsection*{\underline{\textbf{\textcolor{red}{4. Sens de variation d'une suite}}}}
\subsection*{\underline{\textbf{\textcolor{red}{Définition :}}}}
	
Une suite $\left(u_{n}\right)$ est dite :
	
$-\ $Croissante si : $\forall\;n\;,\ :\ u_{n+1}\geq u_{n} .$ c'est-à-dire, $\ u_{n+1}- u_{n}>0$
	
$-\ $Décroissante si : $\forall\;n;,\ :\ u_{n+1}\leq u_{n}$. c'est-à-dire, $\ u_{n+1}- u_{n}<0$

$-\ $Monotone si elle est croissante ou décroissante.

$-\ $Constante si : $\forall\;n\;,\ :\ u_{n+1}=u_{n}.$

Étudier le sens de variation d'une suite $\left(u_{n}\right)$

C'est dire si elle est croissante ou décroissante ou constante.

\subsection*{\underline{\textbf{\textcolor{red}{Règle :}}}}

Pour étudier le sens de variation d'une suite $\left(u_{n}\right)$,on compare deux termes consécutifs, pour cela, on peut étudier le signe de leur différence, ou, s'il s'agit de nombres strictement positifs, comparer leur quotient à $1.$

\textbf{\underline{\exemple}}

Soit la suite $\left(u_{n}\right)$ définie par : $u_{n}=\dfrac{n+2}{2n+1}$

Alors :

$u_{n+1}=\dfrac{(n+1)+2}{2(n+1)+1}=\dfrac{n+3}{2n+3}$
	
$u_{n+1}-u_{n}=\dfrac{n+3}{2n+3}-\dfrac{n+2}{2n+1}=\dfrac{-3}{(2n+1)(2n+3})$
	
Pour tout entier naturel $n$, on a donc : $u_{n+1}-u_{n} < 0.$
	
La suite étudiée est par conséquent décroissante.

\textbf{\underline{\solution}}

\section*{\underline{\textbf{\textcolor{red}{II. Démonstration par récurrence}}}}

La démonstration par récurrence suit les étapes suivantes :  

\textbf{1. Initialisation}  

On montre que la propriété \( P(n) \) est vraie pour une première valeur \( n = n_0 \).  
\[
P(n_0) \quad \text{est vraie.}
\]

\textbf{2. Hypothèse de récurrence}  

On suppose que la propriété est vraie pour un entier \( k \geq n_0 \).  
C'est ce qu'on appelle l'hypothèse de récurrence :  
\[
P(k) \quad \text{est vraie.}
\]

\textbf{3. Hérédité}  

On démontre que si la propriété est vraie pour \( n = k \), alors elle est aussi vraie pour \( n = k+1 \).  
Autrement dit, on montre que :  
\[
P(k) \implies P(k+1).
\]

\textbf{4. Conclusion}  

Si les deux étapes précédentes sont vérifiées, alors, par le principe de récurrence, la propriété est vraie pour tout \( n \geq n_0 \).  

\textbf{\underline{\exemple}}

Montrer que Pour tout \( n \geq 1 \)
\[
S(n) = 1 + 2 + \dots + n = \frac{n(n+1)}{2}.
\]

\textbf{\underline{\solution}}

\textbf{Propriété à démontrer} : Pour tout \( n \geq 1 \), la somme des \( n \) premiers entiers naturels est donnée par :
\[
S(n) = 1 + 2 + \dots + n = \frac{n(n+1)}{2}.
\]

\textbf{Étape 1 : Initialisation}  

Pour \( n = 1 \), on a :  
\[
S(1) = 1.
\]  
D'autre part :  
\[
\frac{1(1+1)}{2} = 1.
\]  
Donc, la propriété est vraie pour \( n = 1 \).  

\textbf{Étape 2 : Hypothèse de récurrence}  

Supposons que la propriété est vraie pour un entier \( k \geq 1 \), c'est-à-dire :  
\[
S(k) = \frac{k(k+1)}{2}.
\]  

\textbf{Étape 3 : Hérédité}  

Montrons que la propriété est vraie pour \( k+1 \).  
En utilisant la définition de \( S(k+1) \), on a :  
\[
S(k+1) = S(k) + (k+1).
\]  
En remplaçant \( S(k) \) par l'hypothèse de récurrence, on obtient :  
\[
S(k+1) = \frac{k(k+1)}{2} + (k+1).
\]  
Factorisons \( (k+1) \) :  
\[
S(k+1) = \frac{k(k+1) + 2(k+1)}{2} = \frac{(k+1)(k+2)}{2}.
\]  
Ainsi, la propriété est vraie pour \( k+1 \).  

\textbf{Conclusion}  

Par le principe de récurrence, la propriété est vraie pour tout \( n \geq 1 \).  


\textbf{\underline{\exerciceapp}}

Soit \( (u_n)_{n \in \mathbb{N}} \) la suite définie par :
\[
\begin{cases} 
u_{n+1} = \frac{1}{3} u_n + \frac{5}{3}, & \forall n \in \mathbb{N}, \\ 
u_0 = 1. & 
\end{cases}
\]

Montrer que : \( \forall n \in \mathbb{N}, \; u_n < \frac{5}{2} \).


Soit \( (u_n)_{n \in \mathbb{N}} \) la suite définie par :
\[
\begin{cases} 
u_{n+1} = \sqrt{12 + u_n}, & \forall n \in \mathbb{N}, \\ 
u_0 = 0. & 
\end{cases}
\]

\begin{enumerate}
    \item Montrer que : \( \forall n \in \mathbb{N}^*, \; 3 \leq u_n \leq 4 \).
\end{enumerate}

\section*{\underline{\textbf{\textcolor{red}{III. Représentation graphique des termes d'une suite}}}}
\subsection*{\underline{\textbf{\textcolor{red}{1. Suite explicite}}}}
$-\ $ Si $u_{n}$ est définie de façon explicite ; $u_{n}=f(n)$ alors représenter graphiquement la suite $(u_{n})$ consiste à représenter dans un repère l'ensemble des points isolés $(n\;,\ u_{n}).$

$-\ $ On peut aussi représenter directement les valeurs des termes de la suite sur l'un des axes du repère.

\textbf{\underline{\exemple}}

Représenter les 6 premiers termes de la suite $(u_{n})$ définie par 

a) $u_{n}=\dfrac{n^{2}+1}{2n+3}$

\textbf{\underline{\solution}}\\
a)\\
\begin{tikzpicture}
    \begin{axis}[
        axis lines = middle,
        xlabel = \( n \),
        ylabel = \( u_n \),
        ymin = 0,
        ymax = 3,
        xmin = -0.5,
        xmax = 6.5,
        xtick = {0, 1, 2, 3, 4, 5},
        ytick = {0, 0.5, 1, 1.5, 2, 2.5, 3},
        grid = both,
        width=10cm,
        height=8cm
    ]
        % Points de la suite
        \addplot[only marks, mark=square*, blue] coordinates {
            (0, 1/3)
            (1, 2/5)
            (2, 5/7)
            (3, 10/9)
            (4, 17/11)
            (5, 2)
        };
    \end{axis}
\end{tikzpicture}

b)\\
La suite \( u_n = 2n - 7 \) est représentée par les points \( (n, u_n) \) pour \( n = 0, 1, 2, 3, 4, 5 \).

\begin{tikzpicture}
    \begin{axis}[
        axis lines = middle,
        xlabel = \( n \),
        ylabel = \( u_n \),
        ymin = -8,
        ymax = 4,
        xmin = -0.5,
        xmax = 5.5,
        xtick = {0, 1, 2, 3, 4, 5},
        ytick = {-8, -6, -4, -2, 0, 2, 4},
        grid = both,
        width=10cm,
        height=8cm
    ]
        % Points de la suite
        \addplot[only marks, mark=*, red] coordinates {
            (0, -7)
            (1, -5)
            (2, -3)
            (3, -1)
            (4, 1)
            (5, 3)
        };
    \end{axis}
\end{tikzpicture}

\subsection*{\underline{\textbf{\textcolor{red}{2. Suite définie par récurrence}}}} 
\[
\begin{cases}
u_0 = \alpha \\
u_{n+1} = g(u_n)
\end{cases}
\]
$g$ étant la fonction associée à la suite $(u_n)$, alors on trace la courbe de $g$ et la première bissectrice (droite d’équation $y = x$). 

Par projection, on construit les termes de la suite $(u_n)$ en procédant comme suit : 

\begin{enumerate}
    \item On place le premier terme $u_0$ sur l’axe des abscisses ;
    \item On utilise $C_g$ pour construire $u_1 = g(u_0)$ sur l’axe des ordonnées ;
    \item On projette $u_1$ sur l’axe des abscisses à l’aide de la première bissectrice ;
    \item On répète la même procédure en utilisant à nouveau $C_g$ pour construire $u_2 = g(u_1)$ sur l’axe des ordonnées ;
    \item On projette ensuite $u_2$ sur l’axe des abscisses à l’aide de la première bissectrice ;
    \item etc.
\end{enumerate}

\[
\begin{cases}
u_0 = \alpha, \\
u_{n+1} = g(u_n),
\end{cases}
\]
où \( g \) est la fonction associée à la suite \((u_n)\).On trace \( C_g \), ainsi que la première bissectrice \( y = x \).  

Par une méthode graphique de projection, on construit les termes successifs de la suite \((u_n)\) en suivant les étapes suivantes :  

\begin{enumerate}
    \item On place le premier terme \( u_0 \) sur l’axe des abscisses.  
    \item On lit la valeur de \( u_1 = g(u_0) \) en projetant \( u_0 \) verticalement sur \( C_g \). Cette valeur se trouve sur l’axe des ordonnées.  
    \item On projette \( u_1 \) sur l’axe des abscisses à l’aide de la première bissectrice \( y = x \).  
    \item On utilise à nouveau la courbe \( C_g \) pour déterminer \( u_2 = g(u_1) \), en projetant \( u_1 \) verticalement sur \( C_g \).  
    \item On projette \( u_2 \) sur l’axe des abscisses via la première bissectrice.  
    \item On répète ce processus pour calculer les termes suivants \( u_3, u_4, \dots \).  
\end{enumerate}

Ce procédé, appelé *construction par itération graphique*, permet de visualiser l'évolution des termes de la suite \((u_n)\) et d'étudier son comportement (convergence, divergence ou oscillation).

\textbf{\underline{\exemple}}

Représenter les 6 premiers termes de la suite $(u_n)$ définie par
$u_{n+1}=2\sqrt{u_{n}}$ et $u_0=2$:

\textbf{\underline{\exerciceapp}}

b) $u_{n}=2n-7$

Soit \( (u_n)_{n \in \mathbb{N}} \) la suite définie par :
\[
\begin{cases} 
u_{n+1} = \sqrt{5u_{n}} + 1 \\ 
u_0 = 1. & 
\end{cases}
\]

\section*{\underline{\textbf{\textcolor{red}{IV. Suite arithmétiques}}}}
Une suite $\left(u_{n}\right)$ est \textcolor{red}{arithmétique} si chaque terme s'obtient en ajoutant au précédent un même nombre \textcolor{red}{$r$} appelé raison : 
\textcolor{red}{$u_{n+1}=u_{n}+r.$} 
\subsection*{\underline{\textbf{\textcolor{red}{1.Expression du terme général}}}}		
$\bullet\ \text{Si }\left(u_{n}\right)$ est une suite arithmétique de premier terme $u_{0}$ et de raison $r$, alors :
	
$$u_{n}=u_{0}+nr$$
	
$\bullet\ $Si le premier terme est $u_{1}$, alors :
$$u_{n}=u_{1}+(n-1)r$$

$\bullet\ $ Si $(U_n)$ une suite arithmétique de raison $r$ et de premier terme $U_{p}$ alors on a : 
\begin{mdframed}[linecolor=red] % Définit la couleur du cadre
    \[
    \color{red} u_{n} = u_{p} + (n-p)r
    \]
\end{mdframed}
Avec $u_{p}$ le premier terme, p l'indice du premier terme et r la raison
\subsection*{\underline{\textbf{\textcolor{red}{Exercie d'application 1 :}}}}
Dans chacun des cas suivants, donner le terme général de la suite.
\begin{itemize}
\item[1.]$(u_n)$ est une suite arithmétique de raison $r = 3$ et de premier terme $u_{0}= 7$.
\item[2.]$(u_n)$ est une suite arithmétique de raison $r=-\frac{3}{4}$ et de premier terme 
$u_{1}=\frac{-7}{4}$.
\item[3.] $(U_n)$ est une suite arithmétique de raison $r$ et de premier terme $U_{0}$ tels que $U_{50} = 406$ et $U_{100} = 806$.
\end{itemize}
\subsection*{\underline{\textbf{\textcolor{green}{Correction :}}}}
\subsection*{\underline{\textbf{\textcolor{red}{Exercie d'application 2 :}}}}
Le prix du transport augmente de $100^{F}$ chaque année dans une ville. En $2019$, le prix du transport était de $600FCFA$.
\begin{itemize}
\item[1.]Calculer le prix du transport en $2020$, $2021$ et en $2022.$
\item[2.]On note par $U_{0}$ le prix en $2019$ et $U_{n}$ le prix en $2019 + n.$
\begin{itemize}
	\item[a.]Que représente $U_{1}$, $U_{2}$ et $U_{3}$ ?
	\item[b.]Exprimer $U_{n+1}$ en fonction de $U_{n}.$
	\item[c.]Quelle est la nature de la suite $U_{n}.$
	\item[d.]Exprimer $U_{n}$ en fonction de $n$.
\end{itemize}
\end{itemize}
\subsection*{\underline{\textbf{\textcolor{red}{Correction :}}}}
\subsection*{\underline{\textbf{\textcolor{red}{A retenir 1 }}}}
Pour montrer qu'une suite $(U_n)$ est arithmétique de raison $r$, il suffit de montrer que :
\[ U_{n+1} - U_n = r \]
\textbf{\underline{\exemple}}
On considère la suite $(U_n)$ définie par :\[ U_n = 5n + 3 \]
Montrer que $(U_n)$ est une suite arithmétique dont on précisera la raison et le premier terme.
\subsection*{\underline{\textbf{\textcolor{red}{A retenir 2}}}}
Pour montrer qu’une suite $(U_n)$ n’est pas arithmétique , il suffit de montrer que : 
$U_{1}$-$U_{0}\neq U_{2}$-$U_{1}$.
\textbf{\underline{\exemple}}
Soit la suite $(U_n)$ définie par : $U_{n}=\frac{n}{n+1}$

Montrer que $(U_n)$ n’est pas une suite arithmétique.
\subsection*{\underline{\textbf{\textcolor{red}{2.Somme des premiers termes}}}}	
Soit (Un) une suite arithmétique.

Pour tous entiers naturels n et p tels que $p \leq n$, on a :

$\bullet\ $Si la suite a pour premier terme $u_{0}$, alors la somme $S_{n}=u_{0}+u_{1}+\ldots+u_{n}$ vaut :
	
$$S_{n}=\dfrac{(n+1)\left(u_{0}+u_{n}\right)}{2}$$

$\bullet\ $Si la suite a pour premier terme $u_{1}$, alors la somme $S_{n}=u_{1}+u_{1}+\ldots+u_{n}$ vaut :

$$S_{n}=\dfrac{n(u_{1}+u_{n})}{2}$$

$\bullet\ $De façon général si la suite a pour premier terme $u_{p}$, alors la somme $S_{n}=u_{p}+u_{p+1}+\ldots+u_{n}$ vaut :

\begin{mdframed}[linecolor=red] % Définit la couleur du cadre
    \[
    \color{red} S_{n}=\dfrac{(n-p+1)(u_{p}+u_{n})}{2}
    \]
\end{mdframed}
\subsection*{\underline{\textbf{\textcolor{red}{Exercie d'application :}}}}
Soit $\left(u_{n}\right)$ la suite définie par : $u_{n}=2n+7$
\begin{itemize}
\item[1.]Calculer $U_{1}$, $U_{2}$ et $U_{2024}.$
\item[2.] Montrer que (Un) est une suite arithmétique.
\item[3.] Calculer la somme : $S_{n} = U_{0} + U_{1} + \cdots+ U_{n}.$
\item[4.] En déduire le somme $S = U_{1} + U_{2} +\cdots+ U_{2024}.$
\end{itemize}
\section*{\underline{\textbf{\textcolor{red}{III. Suites géométriques}}}}
Une suite $\left(u_{n}\right)$ est dite géométrie si chaque terme s'obtient en multipliant le précédent par un même nombre $q$ appelé raison : $u_{n+1}=u_{n}\times q.$
\subsection*{\underline{\textbf{\textcolor{red}{1. Expression du terme général}}}}
$\bullet\ $Si la suite géométrique $\left(u_{n}\right)$ a pour premier terme $u_{0}$ et pour raison $q$, alors : $$u_{n}=u_{0}\times q^{n}$$

$\bullet\ $Si le premier terme est $u_{1}$, alors :$$u_{n}=u_{1}\times q^{n-1}$$

$\bullet\ $ Si le premier terme est $u_{p}$, alors:
\begin{mdframed}[linecolor=red] % Définit la couleur du cadre
    \[
    \color{red} u_{n}=u_{p}\times q^{n-p}
    \]
\end{mdframed}
\textbf{\underline{\exemple}}
Dans chacun des cas suivants, montrer que $(v_{n})$ est une suite géométrique dont on précisera
la raison et le premier terme puis exprimer $(v_{n})$ en fonction de $n$.
\begin{itemize}
\item[1.]$v_{n} = 3 \times 5^{n}$
\item[2.] \begin{equation*} 
    \begin{cases}
        U_{0}=5 \\
        U_{n + 1}=2U_{n}+3
    \end{cases}
    et\quad V_{n} = U_{n} + 3
\end{equation*}
\item[3.] \begin{equation*} 
    \begin{cases}
        U_{1}=0 \\
        U_{n + 1}=-\frac{2}{3}U_{n}+1
    \end{cases}
    et\quad V_{n} = U_{n} - \frac{3}{5}
\end{equation*}
\end{itemize}
\subsection*{\underline{\textbf{\textcolor{red}{A retenir }}}}
Pour montrer qu’une suite $u_{n}$ est géométrique de raison q, il suffit de montrer que :
\begin{mdframed}[linecolor=red] % Définit la couleur du cadre
    \[
    \color{red} \dfrac{U_{n+1}}{U_{n}}=q
    \]
\end{mdframed}
\subsection*{\underline{\textbf{\textcolor{red}{2. Somme des premiers termes }}}}

Pour toute suite géométrique, de raison $q\neq 1$, on a:

$$S_{n}=u_{0}+u_{1}+\ldots+u_{n}=u_{0}\times\dfrac{1-q^{n+1}}{1-q}$$

$$S_{n}u_{1}+u_{2}+\ldots+u_{n}=u_{1}\times\dfrac{1-q^{n}}{1-q}$$

$$S_{n}=u_{p}+u_{p+1}+\ldots+u_{n}=u_{p}\times\dfrac{1-q^{n-p+1}}{1-q}$$

\begin{mdframed}[linecolor=red] % Définit la couleur du cadre
    \[
    \color{red} S_{n}=u_{p}\times\dfrac{1-q^{n-p+1}}{1-q}
    \]
\end{mdframed}
\section*{\underline{\textbf{\textcolor{red}{3. Sens de variation}}}}
    \begin{itemize}
        \item Si $0 < q < 1$, la suite $\left(u_{n}\right)$ est décroissante.
        \item Si $q > 1$, la suite $\left(u_{n}\right)$ est croissante.
        \item Si $q=1$, la suite $\left(u_{n}\right)$ est constante.
    \end{itemize}
\textbf{\underline{\exemple}}
$1+\dfrac{1}{2}+\dfrac{1}{2^{2}}+\ldots\ldots+\dfrac{1}{2^{n}}=S_{n}$ est la somme des premiers termes d'une suite géométrique de raison $\dfrac{1}{2}.$ 

Donc : $1+\dfrac{1}{2}+\dfrac{1}{2^{2}}+\ldots\ldots+\dfrac{1}{2^{n}}=\dfrac{1-\left(\dfrac{1}{2}\right)^{n+1}}{1-\dfrac{1}{2}}=2-\dfrac{1}{2^{n}}$
\section*{\underline{\textbf{\textcolor{red}{IV. limites d'une suite}}}}
La notion de limite en $+\infty$, déjà rencontrée à propos des fonction, s'étend au cas des suites.
	
On a les résultats suivantes :
\subsection*{\underline{\textbf{\textcolor{red}{Théorème 1 :}}}}
a. $\lim\limits_{n\;\longrightarrow\;+\infty}\sqrt{n}=+\infty\ ;\ \lim\limits_{n\;\longrightarrow\;+\infty}n^{2}=+\infty\ ;\ \lim\limits_{n\;\longrightarrow\;+\infty}n^{3}=+\infty.$

b. $\lim\limits_{n\;\longrightarrow\;+\infty}\dfrac{1}{\sqrt{n}}=0\ ;\ \lim\limits_{n\;\longrightarrow\;+\infty}\dfrac{1}{n}=0\ ;\ \lim\limits_{n\;\longrightarrow\;+\infty}\dfrac{1}{n^{2}}=0\ ;\ \lim\limits_{n\;\longrightarrow\;+\infty}\dfrac{1}{n^{3}}=0.$
\subsection*{\underline{\textbf{\textcolor{red}{Théorème 2 :}}}}
Soit $q$ un nombre réel.
Soit \( q \) un nombre réel.

\begin{itemize}
    \item Si \( q > 1 \), alors
    \[
    \lim_{n \rightarrow +\infty} q^n = +\infty.
    \]
    
    \item Si \( -1 < q < 1 \), alors
    \[
    \lim_{n \rightarrow +\infty} q^n = 0.
    \]
\end{itemize}
\subsection*{\underline{\textbf{\textcolor{red}{Théorème 3 :}}}}
Les résultats concernant les opérations sur les limites de fonctions s'étendent aux limites de suites.
\textbf{\underline{\exemple}}

1) Soit la suite \( (u_{n}) \) définie par : \( u_{n} = \dfrac{3n^{3} - 5n^{2} + 1}{2n^{3} + 1} \)

Alors \( \lim_{n \rightarrow +\infty} u_{n} = \lim_{n \rightarrow +\infty} \dfrac{3n^{3}}{2n^{3}} = \dfrac{3}{2} \).

2) Soit la suite \( (v_{n}) \) définie par : \( v_{n} = 1 + \dfrac{1}{3} + \left(\dfrac{1}{3}\right)^{2} + \ldots + \left(\dfrac{1}{3}\right)^{n+1} \).

On a d'après le paragraphe III :

\[
v_{n} = \dfrac{1 - \left(\dfrac{1}{3}\right)^{''}}{1 - \dfrac{1}{3}}
\]

car \( v_{n} \) est la somme des termes consécutifs d'une suite géométrique de raison \( \dfrac{1}{3} \), or comme \( -1 < \dfrac{1}{3} < 1 \), \( \lim_{n \rightarrow +\infty} \left(\dfrac{1}{3}\right)^{n} = 0 \).

D'où : \( \lim_{n \rightarrow +\infty} \dfrac{1}{1 - \dfrac{1}{3}} = \dfrac{3}{2} \).
\end{document}