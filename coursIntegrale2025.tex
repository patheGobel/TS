\documentclass[12pt]{article}
\usepackage{stmaryrd}
\usepackage{graphicx}
\usepackage[utf8]{inputenc}

\usepackage[french]{babel}
\usepackage[T1]{fontenc}
\usepackage{hyperref}
\usepackage{verbatim}

\usepackage{color, soul}

\usepackage{pgfplots}
\pgfplotsset{compat=1.15}
\usepackage{mathrsfs}

\usepackage{amsmath}
\usepackage{amsfonts}
\usepackage{amssymb}
\usepackage{tkz-tab}
\author{Destiné aux élèves de Terminale S\\Lycée de Dindéfelo\\Présenté par M. BA}
\title{\textbf{Calcul Intégral Ts2}}
\date{\today}
\usepackage{tikz}
\usetikzlibrary{arrows, shapes.geometric, fit}

% Commande pour la couleur d'accentuation
\newcommand{\myul}[2][black]{\setulcolor{#1}\ul{#2}\setulcolor{black}}
\newcommand\tab[1][1cm]{\hspace*{#1}}

\begin{document}
\maketitle
\newpage

\section*{\underline{\textbf{\textcolor{red}{I. Intégrale d'une fonction continue: }}}}
\subsection*{\underline{\textbf{\textcolor{red}{1.Introduction :}}}}

Soit \( f \) une fonction continue sur un intervalle \( I \), et \( a \in I \), \( b \in I \), \( F \) et \( G \) 2 primitives de \( f \) sur \( I \)

Pour tout \( x \in I \) ou \( G(x) = F(x) + c \) \( (c\in \mathbb{R}) \)

\underline{\( 
\begin{cases}
a \in I,\quad  G(a) = F(a) + c\\
b \in I,\quad G(b) = F(b) + c
\end{cases}
\)}

\(
\quad\quad=G(b) - G(a) = F(b) - F(a)
\)

Le nombre réel \( F(b) - F(a) \) est indépendant de la primitive \( F \) choisie sur \( I \).

\subsection*{\underline{\textbf{\textcolor{red}{2.Définition :}}}}

Soit \( f \) une fonction continue sur un intervalle \( I \).
En appelle intégrale de \( a \) à \( b \) de \( f \) le nombre réel \( F(b) - F(a) \) où \( F \) est une primitive de \( f \) sur \( I \) on note :

\[
F(b) - F(a) = \int_{a}^{b} f(x) \, dx = \left[ F(x) \right]_{a}^{b}
\]

\begin{itemize}
    \item \( \int_a^b f(x) \, dx \) se lit ``intégrale ou somme  de $a$ à $b$ de $f(x)dx$''.
    \item \( \left[ F(x) \right]_a^b \) se lit ``\( F(x) \) pris entre \( a \) et \( b \)''.
    \item \( a \) et \( b \) sont les bornes de l'intégrale \( \int_a^b f(x) \, dx \)
        \item Dans l'écriture \( \int_a^b f(x) \, dx \) On peut remplaçer \( x \) par toute autre lettre (sauf \( a \) et \( b \)) on écrit \( \int_a^b f(t) \, dt \) ou \( \int_a^b f(s) \, ds \) ; \( x \) est appelé variable muette.
\end{itemize}
\newpage
\section*{\textbf{\textcolor{red}{Exemple 1}}}

\[
f(x) = x^2, \quad F(x) = \frac{1}{3} x^3
\]

\[
\begin{aligned}
\int_1^2 f(x) \, dx &= \left[ \frac{1}{3} x^3 \right]_1^2 \\
&= \frac{1}{3} \left[ 2^3 - 1^3 \right] \\
&= \frac{1}{3} \left[ 8 - 1 \right] \\
&= \frac{7}{3}
\end{aligned}
\]

\[\boxed{\int_1^2 f(x) \, dx=\frac{7}{3}} \]

\section*{\textbf{\textcolor{red}{Exemple 2}}}

\[
f(x) = \cos x, \quad F(x) = \sin x
\]

\[
\begin{aligned}
\int_0^{\frac{\pi}{4}} \cos x \, dx &= \left[ \sin x \right]_0^{\frac{\pi}{4}} \\
&= \sin \left( \frac{\pi}{4} \right) - \sin(0) \\
&= \frac{\sqrt{2}}{2} - 0 \\
&= \frac{\sqrt{2}}{2}
\end{aligned}
\]

\[\boxed{\int_0^{\frac{\pi}{4}} \cos x \, dx = \frac{\sqrt{2}}{2}} \]

\section*{\textbf{\textcolor{red}{Propriété 1 }}}
Soit \( f \) une fonction continue sur un intervalle \( I \), \( a \) et \( b \) deux élément de \( I \). On a
\[
\int_a^a f(x) \, dx = 0 \quad;\quad \quad \int_a^b f(x) \, dx = - \int_b^a f(x) \, dx
\]

En effet :

\[
\int_a^b f(x) \, dx = \left[ F(x) \right]_a^b = F(b) - F(a) = 0
\]

\[
\int_a^b f(x) \, dx = \left[ P(x) \right]_a^b = F(b) - F(a) = - \left( P(b) - P(a) \right)=-\int_b^a f(x) \, dx
\]

\subsection*{\underline{\textbf{\textcolor{red}{3. Intégrale et Primitive :}}}}

Soit \( f \) une fonction continue sur \( I \) (intervalle), \( a \in I \), \( F \) primitive de \( f \) sur \( I \), telle que \( F'(x) = f(x) \), et \( F(a) = 0 \). \(F(x) = \int_a^x f(t) \, dt \)

En effet:

\[
\int_a^x f(t) \, dt = F(x) - F(a) = F(x) - 0
\]

\subsection*{\underline{\textbf{\textcolor{red}{4. Interprétation graphique de l'intégrale :}}}}

Soit \( f \) une fonction continue et \textcolor{red}{positive} sur un intervalle \( I \), \( (Cf) \) sa courbe représentative, \( a \) et \( b \in I \) avec (\( a < b \)).


\[
\int_a^b f(x) \, dx .u.A  : \text{ aire du domaine (noté } A(D) \text{)}
\]

dellimité par la courbe \( (Cf) \), l'axe des abscisse (OI) et les droites d'équation \( x = a \), \( x = b \)

\[
A(D) = \{M\begin{pmatrix} x,y \end{pmatrix} / \quad a \leq x \leq b \quad \text{et} \quad 0 \leq y \leq f(x)\}
\]

++++++++++++++++++++

\subsection*{\underline{\textbf{\textcolor{red}{4. Interprétation graphique de l'intégrale :}}}}

Soit \( f \) une fonction continue et \textcolor{red}{positive} sur un intervalle \( I \), \( (Cf) \) sa courbe représentative, \( a \) et \( b \in I \) avec \( a < b \).

\[
\int_a^b f(x) \, dx \text{ est l'aire (en unité d'aire) du domaine D}
\]

délimité par la courbe \( (Cf) \), l'axe des abscisses \( (OI) \) et les droites d'équation \( x = a \), \( x = b \).

\[
A(D) = \left\{M\begin{pmatrix} x \\ y \end{pmatrix} \mid a \leq x \leq b \quad \text{et} \quad 0 \leq y \leq f(x) \right\}
\]


++++++++++++++++++

\begin{center}
   \includegraphics[scale=0.5]{Screenshot from 2025-04-26 23-30-27.png}
\end{center}

\subsection*{\underline{\textbf{\textcolor{red}{Exerice d'application}}}}

\[
\textcolor{red}{\text{Valeur moyenne de l'intégrale:}} \quad u = \frac{1}{b-a} \int_a^b f(x) \, dx
\]

\section*{\textbf{\textcolor{red}{Remarque}}}

\begin{itemize}
\item Si \( f \) \textcolor{blue}{une fonction continue et négative sur} \([a,b]\). \textcolor{blue}{La symétrie orthogonale d'axe} \( (ox) \) \textcolor{blue}{transforme la courbe} \(\mathcal{C}_f\) \textcolor{blue}{en celle de} \( \mathcal{C}_{-f} \) de f.
\[
A(D)=\int_a^b -f(x) \, dx .u.A 
\]
\begin{center}
   \includegraphics[scale=0.5]{Screenshot from 2025-04-27 00-00-46.png}
\end{center}
\item Si  \( f \) \textcolor{red}{continue et négative sur } \([a,b]\) et \( f \) \textcolor{red}{continue et positive} sur \([b,c]\)

\[
\int_a^c f(x) \, dx = \int_a^b f(x) \, dx + \int_b^c f(x) \, dx
\]

\[
A(D) = \int_a^b -f(x) \, dx.u.A + \int_b^c f(x) \, dx.u.A
\]

\begin{center}
   \includegraphics[scale=0.5]{Screenshot from 2025-04-27 08-29-47.png}
\end{center} 
\end{itemize}  

\section*{\textbf{\textcolor{red}{Exemple 3}}}

$f(x)=x^{2}-1$

          \begin{tikzpicture}[node style/.style={fill opacity=0,text opacity=1}]
              \tkzTabInit[espcl=1.75]{$x$/.5,$x^{2}-1$/.7}{$-\infty$,$-1$,$1$,$+\infty$}
              \tkzTabLine{,+,z,-,z,+}
          \end{tikzpicture}

\begin{center}
   \includegraphics[scale=0.5]{Screenshot from 2025-04-27 08-33-05.png}
\end{center}
\(
\begin{aligned}
A(D) &= \left( \int_0^1 -f(x) \, dx + \int_1^2 f(x) \, dx \right) u.A\\
		 &= \left( \int_0^1 -(x^{2}-1) \, dx + \int_1^2 (x^{2}-1) \, dx \right) u.A\\
		 &=\cdots\\
		 &=2cm^{2}
\end{aligned}
\)  

\[\boxed{A(D)=2cm^{2}}\]   
  
\subsection*{\underline{\textbf{\textcolor{red}{5.Propriétés:}}}}

\subsubsection*{\underline{\textbf{\textcolor{red}{a.Relation de châles:}}}}

Soit \( f \) une fonction continue sur un intervalle \( I \) \(a,b,c \in I \).

on a:

\[
\int_{a}^{b} f(x) \, dx = \int_{a}^{c} f(x) \, dx + \int_{c}^{b} f(x) \, dx
\]

\subsubsection*{\underline{\textbf{\textcolor{red}{Exemples}}}}

\(
\begin{aligned}
\int_{-\frac{\pi}{2}}^{\frac{\pi}{2}} \left| \sin(t) \right| \, dt &= \int_{-\frac{\pi}{2}}^{0} -\sin(t) \, dt + \int_{0}^{\frac{\pi}{2}} \sin(t) \, dt\\
&= -\left[\cos(t) \right]_{-\frac{\pi}{2}}^{0} + \left[ \cos(t) \right]_{0}^{\frac{\pi}{2}}\\
&=1+1\\
&=2
\end{aligned}
\)

\( \int_{-\frac{\pi}{2}}^{\frac{\pi}{2}} \left| \sin(t) \right| \, dt = 2 \)

\subsubsection*{\underline{\textbf{\textcolor{red}{b.Linéarité de l'intégrale}}}}

Soit \( f \) et \( g \) deux fonctions continues sur \( I \). 

\( \alpha, \beta \) deux réels,  \( a, b \in I \).

On a:

\[
\boxed{\int_a^b \left( \alpha f + \beta g \right)(x) \, dx = \alpha \int_a^b f(x) \, dx + \beta \int_a^b g(x) \, dx}
\]

\section*{\underline{\textbf{\textcolor{red}{Exemples}}}}
Calculer les intégrales suivantes
\[ \int_0^{\frac{\pi}{4}}\cos^{2}(t)dt\]
\section*{\underline{\textbf{\textcolor{red}{Solution}}}}
\( 
\begin{aligned} 
\int_0^{\frac{\pi}{4}}\cos^{2}(t)dt &=\int_0^{\frac{\pi}{4}}\frac{1+\cos(2t)}{2}dt \\
&=\frac{1}{2}\int_0^{\frac{\pi}{4}}1+\cos(2t)dt \\
&=\frac{1}{2}\int_0^{\frac{\pi}{4}}1+\frac{1}{2}\int_0^{\frac{\pi}{4}}\cos(2t)dt \\
\end{aligned}
\)
\subsection*{\underline{\textbf{\textcolor{red}{6.Signe de l'intégrale:}}}}
i)Soit $f$ une fonction continue sur $I$, $a$ et $b\in I$  $(a<b)$
\begin{itemize}
\item si $f>0$ sur $[a,b]$ alors $\int_a^b f(x)dx \geq 0$
\item si $f<0$ sur $[a,b]$ alors $\int_a^b f(x)dx \leq 0$
\end{itemize}
ii)Soit $f$ et $g$ deux fonctions continues sur $I$, $a$ et $b\in I$  $(a<b)$
\begin{itemize}
\item $ f \leq 0 $ $x\in[a,b]$ alors $\int_a^b f(x)dx \geq 0$
\item Si $f \geq	 g$ $\forall x \in[a,b]$ alors $ \int_a^b f(x)dx \geq \int_a^b g(x)dx$
\end{itemize}
\section*{\underline{\textbf{\textcolor{red}{Exemples}}}}
\end{document}