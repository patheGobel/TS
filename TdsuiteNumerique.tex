\documentclass[12pt]{article}
\usepackage{lmodern} % Pour une police plus nette
\usepackage{stmaryrd}
\usepackage{graphicx} % Pour l'insertion d'images
\usepackage{float}    % Pour contrôler précisément le placement
\usepackage[utf8]{inputenc}
\usepackage[french]{babel}
\usepackage[T1]{fontenc}
\usepackage{hyperref}
\usepackage{verbatim}
\usepackage{color, soul}
\usepackage{pgfplots}
\pgfplotsset{compat=1.18} % Version plus récente de pgfplots
\usepackage{mathrsfs}
\usepackage{amsmath}
\usepackage{amsfonts}
\usepackage{amssymb}
\usepackage{tkz-tab}
%\author{Destiné aux élèves de Terminale S\\Lycée de Dindéfelo\\Présenté par M. BA}
%\title{\textbf{Rappels et compléments sur les fonctions numériques}}
%\date{\today}
\usepackage{tikz}
\usetikzlibrary{arrows, shapes.geometric, fit}
% Commande pour la couleur d'accentuation
\newcommand{\myul}[2][black]{\setulcolor{#1}\ul{#2}\setulcolor{black}}
\newcommand\tab[1][1cm]{\hspace*{#1}}
\usepackage[margin=2.5cm]{geometry} % Ajustement des marges
\usepackage{eso-pic} % Pour ajouter des éléments en arrière-plan

% Commande pour ajouter du texte en arrière-plan, centré au milieu de chaque page
\AddToShipoutPicture{
    \AtPageCenter{%
        \makebox(0,0)[c]{\rotatebox{60}{\textcolor[gray]{0.9}{\fontsize{2cm}{2cm}\selectfont Pathé Gobel BA}}}
    }
}

\begin{document}

\noindent
\begin{minipage}[t]{0.48\textwidth}
\raggedright
\textbf{Ministère de l'Éducation Nationale}\\
Inspection Académique de Kédougou\\
Lycée Dindéfelo\\
Cellule de Mathématiques
\end{minipage}
\hfill
\begin{minipage}[t]{0.48\textwidth}
\raggedleft
\textbf{Année scolaire 2024-2025}\\
Date : 02/12/2024\\
Classe : Terminale S2\\
Professeur : M. BA
\end{minipage}
\begin{center}
\underline{\textbf{Suites numériques}}
\end{center}
\section*{Exercice 1 \quad Échauffement}

Déterminer toutes les primitives des fonctions suivantes sur \( I \) :

\begin{enumerate}
    \item[\textbf{a)}] \( f(x) = \cos x \sin^5 x, \quad I = \mathbb{R}; \)
    \item[\textbf{b)}] \( f(x) = \frac{7}{(x+2)^{5}}, \quad I = ]-\infty; -2[; \)
    \item[\textbf{c)}] \( f(x) = \frac{2x^{2}+3x-1}{x}, \quad I = ]0; +\infty[; \)
    \item[\textbf{d)}] \( f(x) = \frac{x^{2}}{1+x^{3}}, \quad I = ]0; +\infty[; \)
    \item[\textbf{e)}] \( f(x) = \frac{x}{\sqrt{3x^2 + 2}}, \quad I = \mathbb{R}; \)
    \item[\textbf{f)}] \( f(x) = \left( \frac{1}{2}x-1 \right)^6, \quad I = \mathbb{R}. \)
\end{enumerate}
\section*{Exercice 2 \quad Recherche de primitives}

Déterminer une primitive, sur un intervalle à préciser (le plus grand possible), des fonctions suivantes :

\begin{enumerate}
    \item $f(x) = 2x^2 -7x + 3;$
    \item $f(x) = x^5 - 3x^2 + 5x;$
    \item $f(x) = 6x^3 - \frac{4}{3}x^2 + \frac{1}{2}x-1;$
    \item $f(x) = x^5 -3x^{3}+ 5x;$
    \item $f(x) = 3x^{3}-\frac{1}{2x^2} + \frac{3}{x^3};$
    \item $f(x) = \frac{3}{x^2};$
    \item $f(x) = -\frac{2}{x^6};$
    \item $f(x) = \frac{1}{x^2} + \frac{2}{\sqrt{x}};$
    \item $f(x) = -\frac{3}{x^2} + \frac{1}{2\sqrt{x}} + 2\sqrt{x};$
    \item $f(x) = 3x + \frac{2}{x^2} - \frac{5}{x^3};$
    \item $f(x) = \frac{7x^3 - 4x^2 + 3x - 1}{5};$
    \item $f(x) = -\frac{5x^4 + 2x^3 - 3x + 6}{x^3};$
    \item $f(x)=\frac{4(x-1)^{3}-1}{(x-1)^{2}}.$
\end{enumerate}
\end{document}