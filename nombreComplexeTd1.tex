\documentclass[12pt]{article}
\usepackage{lmodern} % Pour une police plus nette
\usepackage{stmaryrd}
\usepackage{graphicx} % Pour l'insertion d'images
\usepackage{float}    % Pour contrôler précisément le placement
\usepackage[utf8]{inputenc}
\usepackage[french]{babel}
\usepackage[T1]{fontenc}
\usepackage{hyperref}
\usepackage{verbatim}
\usepackage{color, soul}
\usepackage{pgfplots}
\pgfplotsset{compat=1.18} % Version plus récente de pgfplots
\usepackage{mathrsfs}
\usepackage{amsmath}
\usepackage{amsfonts}
\usepackage{amssymb}
\usepackage{tkz-tab}
%\author{Destiné aux élèves de Terminale S\\Lycée de Dindéfelo\\Présenté par M. BA}
%\title{\textbf{Rappels et compléments sur les fonctions numériques}}
%\date{\today}
\usepackage{tikz}
\usetikzlibrary{arrows, shapes.geometric, fit}
% Commande pour la couleur d'accentuation
\newcommand{\myul}[2][black]{\setulcolor{#1}\ul{#2}\setulcolor{black}}
\newcommand\tab[1][1cm]{\hspace*{#1}}
\usepackage[margin=2.5cm]{geometry} % Ajustement des marges
\usepackage{eso-pic} % Pour ajouter des éléments en arrière-plan

% Commande pour ajouter du texte en arrière-plan, centré au milieu de chaque page
\AddToShipoutPicture{
    \AtPageCenter{%
        \makebox(0,0)[c]{\rotatebox{60}{\textcolor[gray]{0.9}{\fontsize{2cm}{2cm}\selectfont PGB}}}
    }
}

\begin{document}

\noindent
\begin{minipage}[t]{0.48\textwidth}
\raggedright
\textbf{Ministère de l'Éducation Nationale}\\
Inspection Académique de Kédougou\\
Lycée Dindéfelo\\
Cellule de Mathématiques
\end{minipage}
\hfill
\begin{minipage}[t]{0.48\textwidth}
\raggedleft
\textbf{Année scolaire 2024-2025}\\
Date : 03/01/2025\\
Classe : Terminale S2\\
Professeur : M. BA
\end{minipage}
\vspace{1cm}
\begin{center}
\textbf{\textcolor{red}{Nombres Complexes}}
\end{center}
\vspace{1cm}
\section*{\textcolor{black}{Exercice 1 :}}
\begin{enumerate}
    \item Déterminer le module et un argument des nombres complexes suivants :
    \begin{enumerate}
        \item $2i$ ;
        \item $\sqrt{3} + 3i$ ;
        \item $\sqrt{6} + i\sqrt{2}$ ;
        \item $5$ ;
        \item $\left(\frac{1}{2}-i\frac{\sqrt{3}}{2}\right)^3$ ;
        \item $\left(\frac{1}{2}+i\frac{\sqrt{3}}{2}\right)\left(-\frac{1}{2}+i\frac{\sqrt{3}}{2}\right)$ ;
        \item $\frac{-1+i\sqrt{3}}{\sqrt{3} + i}$.
    \end{enumerate}
    \item Écrire sous forme trigonométrique les nombres complexes suivants :
    \begin{enumerate}
        \item $(2+2i)(1-i)$ ;
        \item $\frac{-1+i\sqrt{3}}{1+i}$ ;
        \item $\frac{\sqrt{2}}{1+i}$ ;
        \item $\frac{-2i}{1+i\sqrt{3}}$ ;
        \item $(-1-i)^4$ ;
        \item $\left(\frac{1+i\sqrt{3}}{1-i}\right)^2$.
    \end{enumerate}
    \item Soit $z_1 = 1+i$ et $z_2 = 1+i\sqrt{3}$.
    \begin{enumerate}
        \item Déterminer le module et un argument de $z_1$ et $z_2$.
        \item Écrire sous forme algébrique et sous forme trigonométrique le produit $z_1z_2$.
        \item En déduire les valeurs de $\cos\frac{7\pi}{12}$ et $\sin\frac{7\pi}{12}$.
    \end{enumerate}
\end{enumerate}

\section*{\textcolor{black}{Exercice 2 :}}
\begin{enumerate}
    \item Mettre sous forme alg\'ebrique les nombres complexes suivants :
    \begin{enumerate}
        \item $z_1 = (1 - i)(5 + i)$
        \item $z_2 = (2 - 3i)^2$
        \item $z_3 = \frac{1}{3 + 2i}$
        \item $z_4 = \frac{4 - 5i}{3 + 2i}$
    \end{enumerate}
    \item \`Ecrire en fonction de $\overline{z}$ les conjugu\'es des nombres complexes suivants :
    \begin{enumerate}
        \item $z_1 = 1 + iz$
        \item $z_2 = i(z + 3)$
        \item $z_3 = \frac{1 - z}{1 + iz}$
        \item $z_4 = \frac{1 + 3z}{i + 2z}$
    \end{enumerate}
    \item D\'eterminer un argument de $z$ dans chacun des cas suivants :
    \begin{enumerate}
        \item $z = -1 + i$
        \item $z = \sqrt{6} + i\sqrt{6}$
        \item $z = \frac{1}{2} - i\frac{\sqrt{3}}{2}$
        \item $z = (2 + 2i)(1 - i)$
        \item $z = \frac{-1 + i\sqrt{3}}{1 + i}$
        \item $z = (-1 - i)^4$
    \end{enumerate}
\end{enumerate}

\section*{\textcolor{black}{Exercice 3 :}}
Le plan est muni d'un rep\`ere orthonorm\'e direct.
\begin{enumerate}
    \item D\'eterminer puis construire l'ensemble des points $M$ du plan d'affixe $z$ v\'erifiant :
    \begin{enumerate}
        \item $|z - 3| = |z + i|$
        \item $|iz + 3| = |z + 4 + i|$
        \item $|\overline{z} + \frac{1}{3}i| = 3$
        \item $|z - \overline{z} + i| = 2$
        \item $|\overline{z} - 2 + i| = |z + 5 - 2i|$
        \item $|\overline{z} - 2 + i| = |z + 5 - 2i|$
    \end{enumerate}
    \item Pour tout nombre complexe $z \neq -1 + 2i$, on pose $Z = \frac{z - 2 + 4i}{z + 1 - 2i}$.

    Déterminer l'ensemble des points $M$ du plan tels que :

\begin{enumerate}
        \item $|Z| = 1$
        \item $|Z| = 2$
        \item $Z$ soit un réel.
        \item $Z$ soit un imaginaire pur.
\end{enumerate}
    \item Pour tout complexe $z \neq i$, on pose
    \(
    U = \frac{z + i}{z - i}.
    \)
    
    Déterminer l’ensemble des points $M$ d’affixe $z$ tels que :
    \begin{enumerate}
        \item $U \in \mathbb{R}^*_{-}$
        \item $U \in \mathbb{R}^*_{+}$
        \item $U \in i\mathbb{R}$
    \end{enumerate}
\end{enumerate}

\end{document}