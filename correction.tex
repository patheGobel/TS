\documentclass[12pt]{article}
\usepackage{stmaryrd}
\usepackage{graphicx}
\usepackage[utf8]{inputenc}

\usepackage[french]{babel}
\usepackage[T1]{fontenc}
\usepackage{hyperref}
\usepackage{verbatim}

\usepackage{color, soul}

\usepackage{pgfplots}
\pgfplotsset{compat=1.15}
\usepackage{mathrsfs}

\usepackage{amsmath}
\usepackage{amsfonts}
\usepackage{amssymb}
\usepackage{tkz-tab}
\author{\\Lycée de Dindéfelo\\Mr BA}
\title{\textbf{Correction}}
\date{\today}
\usepackage{tikz}
\usetikzlibrary{arrows, shapes.geometric, fit}

% Commande pour la couleur d'accentuation
\newcommand{\myul}[2][black]{\setulcolor{#1}\ul{#2}\setulcolor{black}}
\newcommand\tab[1][1cm]{\hspace*{#1}}

\begin{document}
\maketitle
\newpage
\section*{\underline{\textbf{\textcolor{red}{Exercice 1}}}}
Combien de menus différents peut-on composer si on a le choix entre 3 entrées, 2 plats et 4 desserts ?
\section*{\underline{\textbf{\textcolor{red}{Correction}}}}
Pour calculer le nombre de menus différents, nous utilisons le principe de multiplication.

Si nous avons 3 choix d'entrées, 2 choix de plats et 4 choix de desserts, le nombre total de menus différents est le produit de ces choix.

Nombre de menus différents = (nombre d'entrées) × (nombre de plats) ×
	 (nombre de desserts)

En substituant les valeurs, nous avons :

Nombre de menus différents = 3 × 2 × 4

Calculons le produit :

Nombre de menus différents = 24

Donc, il est possible de composer 24 menus différents en choisissant parmi 3 entrées, 2 plats et 4 desserts.
\section*{\underline{\textbf{\textcolor{red}{Exercice 2}}}}
Une femme a dans sa garde-robe 4 jupes, 5 chemisiers et 3 vestes. Elle choisit au hasard une jupe, un chemisier et une veste. De combien de façons différentes peut-elle s’habiller ?
\section*{\underline{\textbf{\textcolor{red}{Correction}}}}
Pour calculer le nombre de façons différentes qu'elle peut s'habiller, nous utilisons également le principe de multiplication.

Elle a le choix entre 4 jupes, 5 chemisiers et 3 vestes. Le nombre total de façons différentes qu'elle peut s'habiller est le produit de ces choix.

Nombre de façons différentes = (nombre de jupes) × (nombre de chemisiers) × (nombre de vestes)

En substituant les valeurs, nous avons :

Nombre de façons différentes = 4 × 5 × 3

Calculons le produit :

Nombre de façons différentes = 60

Donc, elle peut s'habiller de 60 façons différentes en choisissant une jupe, un chemisier et une veste au hasard dans sa garde-robe.
\section*{\underline{\textbf{\textcolor{red}{Exercice 3}}}}
Deux équipes de hockeys de 12 et 15 joueurs échangent une poignée de main à la fin d’un match : chaque joueur d’une équipe serre la main de chaque joueur de l’autre équipe. Combien de poignées de main ont été échangées ?
\section*{\underline{\textbf{\textcolor{red}{Correction}}}}
Pour résoudre ce problème, je t'invite à utiliser le principe multiplicatif. Celui-ci te permettra de déterminer le nombre total de façons différentes en multipliant le nombre d'options disponibles pour chaque élément que tu dois choisir.
\section*{\underline{\textbf{\textcolor{red}{Exercice 10}}}}
1) Un tel choix est donné par un 6-uplet (sextuplé) de 6 chiffres, chacun choisi entre 1 et 6. Pour connaître le nombre de choix, on effectue le produit cartésien de l’ensemble $\left\lbrace 1; 2;3; 4;5; 6\right\rbrace $ six fois par lui-même. Il y donc $6^{6} = 46656$ choix
possibles.

2) Si les six chiffres doivent être distincs, un tel choix sera donné par un arrangement de 6 chiffres choisis parmi 6, c’est-à-dire une permutation des 6 chiffres. Il aura donc $6 !=720$ choix possibles
\section*{\underline{\textbf{\textcolor{red}{Exercice 12}}}}
1) Un code est un élément du produit cartésien entre un élément de l’ensemble $\left\lbrace A ;B ;C\right\rbrace $, de cardinal 3, et de l’ensemble
des 3-listes d’éléments de $\left\lbrace 1 ;2 ;3 ;4 ;5 ;6\right\rbrace $, de cardinal 63 = 216
Il y a donc $3 \times 63 = 3 \times 216 = 648$ codes possibles.

2) Si le code ne doit pas contenir de chiffre 1, alors les 3-listes sont constituées d’éléments de $\left\lbrace 2 ;3 ;4 ;5 ;6\right\rbrace $. Il y en a donc $53 = 125$ , et le nombre de codes vaut alors $3 \times 53 = 3 \times 125 = 375$

3) Le contraire de « le code contient au moins une fois le chiffre 1 » est « le code ne contient aucun chiffre 1 »

Le nombre de codes contenant au moins une fois le chiffre 1 est donc égal au nombre total de codes diminué du nombre
de codes ne contenant pas le chiffre 1. Ces deux nombres ayant été calculés dans les deux questions précédentes, on
conclut que le nombre de codes contenant au moins une fois le chiffre 1 est égal à 648-375=273

4) un code comportant des chiffres distincts sera un élément du produit cartésien entre un élément de l’ensemble $\left\lbrace A ;B ;C\right\rbrace $, de cardinal 3, et de l’ensemble des arrangements de 3 éléments pris parmi $\left\lbrace 1 ;2 ;3 ;4 ;5 ;6\right\rbrace $. Ces arrangements sont au nombre de 
$A_{6}^{3}=\frac{6!}{(6-3)!}=6 \times 5 \times 4 = 120$ Il y a donc $3A_{6}^{3}=3\times120=360$ codes possibles.

5) Le contraire de « le code contient au moins deux chiffres identiques» étant « le code ne contient que des chiffres
distincts », le nombre de codes contenant au moins deux chiffres identiques est égal au nombre total de codes diminué du
nombre de codes ne contenant que des chiffres distincts, soit 648-360=288 codes possibles.
\section*{\underline{\textbf{\textcolor{red}{Exercice 17}}}}
Une anagramme du mot TABLEAU est une permutation des $7$ lettres de ce mot. Il y en a donc, a priori, $7 !$

Mais si au sein de ces anagrammes, on « permute » les deux lettres A, on retombe sur le même mot.

Autrement dit, au sein des $7 !$ anagrammes, sont comptées deux fois les mots où se permutent les deux lettres A

Pour éviter de compter ces anagrammes deux fois, on doit diviser $7 !$ par le nombre de permutations possibles des deux
lettres A, soit $2 !=2$

Le nombre d’anagrammes différentes du mot TABLEAU est donc égal à $\frac{7!}{2!}=2520$
\section*{\underline{\textbf{\textcolor{red}{Exercice 18}}}}
Cet exercice est une généralisation de l'exercice: \textbf{\textcolor{red}{Exercice 17}}

\textbf{\textcolor{blue}{Now, you have to think !!!}}
\end{document}