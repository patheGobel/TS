\documentclass[12pt]{article}
\usepackage{lmodern} % Pour une police plus nette
\usepackage{stmaryrd}
\usepackage{graphicx} % Pour l'insertion d'images
\usepackage{float}    % Pour contrôler précisément le placement
\usepackage[utf8]{inputenc}
\usepackage[french]{babel}
\usepackage[T1]{fontenc}
\usepackage{hyperref}
\usepackage{verbatim}
\usepackage{color, soul}
\usepackage{pgfplots}
\pgfplotsset{compat=1.18} % Version plus récente de pgfplots
\usepackage{mathrsfs}
\usepackage{amsmath}
\usepackage{amsfonts}
\usepackage{amssymb}
\usepackage{tkz-tab}
%\author{Destiné aux élèves de Terminale S\\Lycée de Dindéfelo\\Présenté par M. BA}
%\title{\textbf{Rappels et compléments sur les fonctions numériques}}
%\date{\today}
\usepackage{tikz}
\usetikzlibrary{arrows, shapes.geometric, fit}
% Commande pour la couleur d'accentuation
\newcommand{\myul}[2][black]{\setulcolor{#1}\ul{#2}\setulcolor{black}}
\newcommand\tab[1][1cm]{\hspace*{#1}}
\usepackage[margin=2.5cm]{geometry} % Ajustement des marges
\usepackage{eso-pic} % Pour ajouter des éléments en arrière-plan

% Commande pour ajouter du texte en arrière-plan, centré au milieu de chaque page
\AddToShipoutPicture{
    \AtPageCenter{%
        \makebox(0,0)[c]{\rotatebox{60}{\textcolor[gray]{0.9}{\fontsize{2cm}{2cm}\selectfont PGB}}}
    }
}

\begin{document}

\noindent
\begin{minipage}[t]{0.48\textwidth}
\raggedright
\textbf{Ministère de l'Éducation Nationale}\\
Inspection Académique de Kédougou\\
Lycée Dindéfelo\\
Cellule de Mathématiques
\end{minipage}
\hfill
\begin{minipage}[t]{0.48\textwidth}
\raggedleft
\textbf{Année scolaire 2024-2025}\\
Date : 03/01/2025\\
Classe : Terminale S2\\
Professeur : M. BA
\end{minipage}
\vspace{1cm}
\begin{center}
\textbf{\textcolor{red}{Nombres Complexes}}
\end{center}
\vspace{1cm}
\section*{\textcolor{black}{Exercice 1 :}}
\begin{enumerate}
    \item Déterminer le module et un argument des nombres complexes suivants :
    \begin{enumerate}
        \item $2i$ ;
        \item $\sqrt{3} + 3i$ ;
        \item $\sqrt{6} + i\sqrt{2}$ ;
        \item $5$ ;
        \item $\left(\frac{1}{2}-i\frac{\sqrt{3}}{2}\right)^3$ ;
        \item $\left(\frac{1}{2}+i\frac{\sqrt{3}}{2}\right)\left(-\frac{1}{2}+i\frac{\sqrt{3}}{2}\right)$ ;
        \item $\frac{-1+i\sqrt{3}}{\sqrt{3} + i}$.
    \end{enumerate}
    \item Écrire sous forme trigonométrique les nombres complexes suivants :
    \begin{enumerate}
        \item $(2+2i)(1-i)$ ;
        \item $\frac{-1+i\sqrt{3}}{1+i}$ ;
        \item $\frac{\sqrt{2}}{1+i}$ ;
        \item $\frac{-2i}{1+i\sqrt{3}}$ ;
        \item $(-1-i)^4$ ;
        \item $\left(\frac{1+i\sqrt{3}}{1-i}\right)^2$.
    \end{enumerate}
    \item Soit $z_1 = 1+i$ et $z_2 = 1+i\sqrt{3}$.
    \begin{enumerate}
        \item Déterminer le module et un argument de $z_1$ et $z_2$.
        \item Écrire sous forme algébrique et sous forme trigonométrique le produit $z_1z_2$.
        \item En déduire les valeurs de $\cos\frac{7\pi}{12}$ et $\sin\frac{7\pi}{12}$.
    \end{enumerate}
\end{enumerate}

\section*{\textcolor{black}{Exercice 2 :}}
\begin{enumerate}
    \item Mettre sous forme alg\'ebrique les nombres complexes suivants :
    \begin{enumerate}
        \item $z_1 = (1 - i)(5 + i)$
        \item $z_2 = (2 - 3i)^2$
        \item $z_3 = \frac{1}{3 + 2i}$
        \item $z_4 = \frac{4 - 5i}{3 + 2i}$
    \end{enumerate}
    \item \`Ecrire en fonction de $\overline{z}$ les conjugu\'es des nombres complexes suivants :
    \begin{enumerate}
        \item $z_1 = 1 + iz$
        \item $z_2 = i(z + 3)$
        \item $z_3 = \frac{1 - z}{1 + iz}$
        \item $z_4 = \frac{1 + 3z}{i + 2z}$
    \end{enumerate}
    \item D\'eterminer un argument de $z$ dans chacun des cas suivants :
    \begin{enumerate}
        \item $z = -1 + i$
        \item $z = \sqrt{6} + i\sqrt{6}$
        \item $z = \frac{1}{2} - i\frac{\sqrt{3}}{2}$
        \item $z = (2 + 2i)(1 - i)$
        \item $z = \frac{-1 + i\sqrt{3}}{1 + i}$
        \item $z = (-1 - i)^4$
    \end{enumerate}
\end{enumerate}

\section*{\textcolor{black}{Exercice 3 :}}
Le plan est muni d'un rep\`ere orthonorm\'e direct.
\begin{enumerate}
    \item D\'eterminer puis construire l'ensemble des points $M$ du plan d'affixe $z$ v\'erifiant :
    \begin{enumerate}
        \item $|z - 3| = |z + i|$
        \item $|iz + 3| = |z + 4 + i|$
        \item $|\overline{z} + \frac{1}{3}i| = 3$
        \item $|z - \overline{z} + i| = 2$
        \item $|\overline{z} - 2 + i| = |z + 5 - 2i|$
        \item $|\overline{z} - 2 + i| = |z + 5 - 2i|$
    \end{enumerate}
    \item Pour tout nombre complexe $z \neq -1 + 2i$, on pose $Z = \frac{z - 2 + 4i}{z + 1 - 2i}$.

    Déterminer l'ensemble des points $M$ du plan tels que :

\begin{enumerate}
        \item $|Z| = 1$
        \item $|Z| = 2$
        \item $Z$ soit un réel.
        \item $Z$ soit un imaginaire pur.
\end{enumerate}
    \item Pour tout complexe $z \neq i$, on pose
    \(
    U = \frac{z + i}{z - i}.
    \)
    
    Déterminer l’ensemble des points $M$ d’affixe $z$ tels que :
    \begin{enumerate}
        \item $U \in \mathbb{R}^*_{-}$
        \item $U \in \mathbb{R}^*_{+}$
        \item $U \in i\mathbb{R}$
    \end{enumerate}
\end{enumerate}

\section*{\textcolor{black}{Exercice 4 :}}
Soit le nombre complexe 
\(
z = \frac{2(-1 + i\sqrt{3})}{1 + i\sqrt{3}}.
\)

\begin{enumerate}
    \item Déterminer $Re(z)$ et $Im(z)$.
    \item Déterminer le module et un argument de $z$.
    \item En déduire le module et un argument de :
    \(
    \frac{1}{z}, \quad \frac{i}{z} \quad \text{et} \quad \frac{1 + i}{z}.
    \)
\end{enumerate}

\section*{\textcolor{black}{Exercice 5 :}}
\begin{enumerate}
    \item On pose $z_1 = \frac{\sqrt{6} + i\sqrt{2}}{2}$ ; $z_2 = 1 - i$ et $z_3 = \frac{z_1}{z_2}$.
    \begin{enumerate}
        \item Déterminer un argument de $z_1$, $z_2$ et $z_3$.
        \item En déduire les valeurs exactes de $\cos\left(\frac{5\pi}{12}\right)$ et $\sin\left(\frac{5\pi}{12}\right)$.
    \end{enumerate}
    \item On considère les nombres complexes : $a = 1 - i$, $b = 1 - i\sqrt{3}$ et $Z = \frac{a^5}{b^4}$.
    \begin{enumerate}
        \item Déterminer une écriture trigonométrique de $Z$.
        \item Déterminer une écriture cartésienne de $Z$. \\
        En déduire les valeurs de $\cos\left(\frac{\pi}{12}\right)$ et de $\sin\left(\frac{\pi}{12}\right)$.
        \item Calculer $Z^{12}$ et $Z^{2024}$.
        \item Pour quelles valeurs de l’entier naturel $n$ :
        
        $Z^{n}$ est un réel.\\
        $Z^{n}$ est un imaginaire pur.\\
    \end{enumerate}
\end{enumerate}
\section*{\textcolor{black}{Exercice 6 :}}
On donne $u = \sqrt{2 - \sqrt{2}} + i \sqrt{2 + \sqrt{2}}.$

\begin{enumerate}
    \item Calculer $u^2$ et $u^4$ sous forme algébrique.
    \item En déduire le module et un argument de $u$.
    \item Soit $M$ le point d’affixe $z \in \mathbb{C}$.\\ Déterminer l’ensemble des points $M$ tels que $|uz| = 8$.
\end{enumerate}
\section*{\textcolor{black}{Exercice 7 :}}

Soit dans le plan complexe les points $A$, $B$ et $C$ d’affixes respectives $a$, $b$ et $c$ avec $a = -1 - i$, $b = 2 + i$ et $c = 4i$.

\begin{enumerate}
    \item Représenter les points $A$, $B$ et $C$.
    \item Déterminer la forme trigonométrique de $\frac{a - b}{c - b}$.\\  
    En déduire que le triangle $ABC$ est rectangle isocèle.
    \item Déterminer l’affixe du point $D$ tel que $ABCD$ soit un carré.
\end{enumerate}

\section*{\textcolor{black}{Exercice 8 :}}

\begin{enumerate}
    \item Résoudre dans $\mathbb{C}$ les équations suivantes :
    \begin{enumerate}
        \item $\overline{z} = i - z$
        \item $2z + \overline{z} = i - z$
        \item $2z^2 - 6z + 5 = 0$
        \item $z^2 + z + 1 = 0$
        \item $z^2 - (1 + 2i)z + i - 1 = 0$
        \item $z - \sqrt{3}z - i = 0$
        \item $z^2 - (3 + 4i)z - 1 + 5i = 0$
        \item $4z^2 - 2z + 1 = 0$
    \end{enumerate}

    \item Résoudre dans $\mathbb{C}$ chacune des équations suivantes sachant qu’elles admettent une solution réelle :
    \begin{enumerate}
        \item $z^3 - (1 + i)z^2 - 2(1 + i)z + 8 = 0$
        \item $iz^3 + (3 - 5i)z^2 + (16 - 2i)z + 30i = 0$
    \end{enumerate}

    \item Résoudre chacune des équations suivantes sachant qu’elles admettent une solution imaginaire pure :
    \begin{enumerate}
        \item $z^3 - (3 + 4i)z^2 - 6(3 - 2i)z + 72i = 0$
        \item $z^3 + (5i - 1)z^2 - (4i + 7)z + 3 - 3i = 0$
    \end{enumerate}
\end{enumerate}

\section*{\textcolor{black}{Exercice 9 :}}

\begin{enumerate}
    \item
    \begin{enumerate}
        \item Résoudre dans $\mathbb{C}$ l'équation 
        \[
        Z^2 - 6Z + 13 = 0.
        \]

        \item En déduire les solutions de l'équation :
        \[
        \left( \frac{z - 3i}{z + 2i} \right)^2 - 6 \left( \frac{z - 3i}{z + 2i} \right) + 13 = 0.
        \]
    \end{enumerate}

    \item Soit $\theta \in [0; \pi]$, donner sous forme trigonométrique les solutions dans $\mathbb{C}$ des équations suivantes :
    \begin{enumerate}
        \item $Z^2 + 2(1 - \cos \theta)Z + 2(1 - \cos \theta) = 0.$
        \item $Z^2 - (2^{\theta+1} \cos \theta)Z + 2^{2\theta} = 0.$
    \end{enumerate}
\end{enumerate}

\section*{\textcolor{black}{Exercice 10 :}}

\begin{enumerate}
    \item Résoudre dans $\mathbb{C}$, $z^3 = 1$.

    \item
    \begin{enumerate}
        \item Développer $(\sqrt{2} - i\sqrt{2})^3$.

        \item Soit l'équation $(E) : z^3 = 4\sqrt{2}(-1 - i)$.

        En posant $u = \frac{z}{\sqrt{2} - i\sqrt{2}}$, déterminer sous forme algébrique puis sous forme trigonométrique les racines de l'équation $(E)$.
    \end{enumerate}

    \item En déduire les valeurs exactes de $\cos\frac{5\pi}{12}$ et $\sin\frac{5\pi}{12}$.
\end{enumerate}

\section*{\textcolor{black}{Exercice 11 :}}

\begin{enumerate}
    \item Montrer que $(1 + i)^6 = -8i$.

    \item On considère l'équation $(E) : z^2 = -8i$.
    \begin{enumerate}
        \item Déduire de la question 1 une résolution de l'équation $(E)$.
        \item Donner sous forme algébrique les solutions de $(E)$.
    \end{enumerate}

    \item
    \begin{enumerate}
        \item Déduire également de 1 une solution notée $t$ de l'équation $(E') : z^3 = -8i$.
        \item On pose $j = e^{i\left(\frac{2\pi}{3}\right)}$. \\ 
        Montrer que $jt$ et $j^2t$ sont aussi des solutions de $(E')$.
    \end{enumerate}
\end{enumerate}
\section*{\textcolor{black}{Exercice 12 :}}

\begin{enumerate}
    \item Résoudre dans l’ensemble $\mathbb{C}$ des nombres complexes les équations suivantes :
    \begin{enumerate}
        \item $z^2 - 2z + 5 = 0$
        \item $z^2 - 2(1 + \sqrt{3})z + 5 + 2\sqrt{3} = 0$
    \end{enumerate}

    \item On considère dans le plan complexe rapporté à un repère orthonormal $(O; \vec{u}, \vec{v})$ d’unité graphique $2\ \text{cm}$, les points $A$, $B$, $C$ et $D$ d’affixes respectives $1 + 2i$, $1 + \sqrt{3} + i$,\\ $1 + \sqrt{3} - i$ et $1 - 2i$.
    \begin{enumerate}
        
        \item Placer les points $A$, $B$, $C$ et $D$ et préciser la nature du quadrilatère $ABCD$.

        \item Vérifier que 
        \[
        \frac{Z_D - Z_B}{Z_A - Z_B} = i\sqrt{3}.
        \]
        Que peut-on en déduire pour les droites $(AB)$ et $(BD)$ ?

        \item Prouver que les points $A$, $B$, $C$ et $D$ appartiennent à un même cercle $\Gamma$ dont on précisera le rayon et le centre. Tracer $\Gamma$.
    \end{enumerate}

    \item On considère l’équation :
    \[
    z^2 - 2(1 + 2\cos\theta)z + 5 + 4\cos\theta = 0,
    \]
    où $\theta$ est un nombre réel quelconque.
    \begin{enumerate}
        \item Résoudre l’équation dans $\mathbb{C}$.

        \item Montrer que les points ayant pour affixe les solutions de l’équation appartiennent à $\Gamma$.
    \end{enumerate}
\end{enumerate}

\section*{\textcolor{black}{Exercice 13 :}}
\section*{\textcolor{black}{Exercice 14 :}}

\end{document}
