\documentclass[12pt]{article}
\usepackage{stmaryrd}
\usepackage{graphicx}
\usepackage[utf8]{inputenc}
\usepackage[french]{babel}
\usepackage[T1]{fontenc}
\usepackage{hyperref}
\usepackage{verbatim}
\usepackage{color,soul}
\usepackage{amsmath}
\usepackage[table]{xcolor}
\usepackage{amsfonts}
\usepackage{amssymb}
\usepackage{systeme}
\usepackage{tkz-tab}
\author{Destiné à la TerminaleS2\\Au Lycée de Dindéfelo}
\title{\textbf{Dénombrement}}
\date{\today}
\usepackage{tikz}
\usetikzlibrary{arrows}
\usepackage[a4paper,left=20mm,right=20mm,top=15mm,bottom=15mm]{geometry}
\usepackage{mathtools}
\usepackage{systeme}

\usepackage{pgfplots}
\pgfplotsset{compat=1.15}
\usepackage{mathrsfs}

\usetikzlibrary{arrows}
\pagestyle{empty}

\DecimalMathComma

\begin{document}

\maketitle
\newpage
\section*{\underline{\textbf{\textcolor{red}{I.Introduction}}}}
\subsection*{\underline{\textbf{\textcolor{red}{1.Définition des objets :}}}}
Dénombrer, c’est compter des objets.

Ces objets sont créés à partir d’un ensemble E, formé d’éléments.
A partir des éléments de cet ensemble, les objets que l’on peut former sont soit des 
\textbf{\textcolor{red}{listes}}
d’éléments de E soit des \textbf{\textcolor{red}{sous-ensembles}} de E.

A la différence des sous-ensembles,
les listes peuvent utiliser plusieurs fois un même élément, et surtout, possèdent un ordre.\\
Dans les exercices, les objets obtenus sont le résultat d’une expérience aléatoire, c’est à dire que le hasard intervient dans leur formation.\\

%Le dénombrement de ces objets, lui, n’a rien d’aléatoire,
%il est un décompte organisé de tous les résultats possibles d’une telle expérience.

%Nous allons maintenant définir avec plus de précisions les différents objets que l’on peut rencontrer et pour ce, nous allons prendre comme exemple
%l’ensemble suivant :
\subsection*{\underline{\textbf{\textcolor{red}{2.Cardinale d'un ensemble}}}}
Soit E un ensemble à n éléments.
Le nombre d’éléments de E est appelé cardinal de E on note card E = n.\\
\textbf{\textcolor{red}{Exemple}}\\
Soit E un ensemble à 5 éléments : E = $\lbrace a ; b ; c ; d ; e \rbrace$
le nombre d’éléments de E est appelé cardinal de E et noté card E = 5
\subsection*{\underline{\textbf{\textcolor{red}{3.Dénombrement de parites d'ensemles finis}}}}
\subsection*{\underline{\textbf{\textcolor{red}{3.1.Cardinal de la réunion de deux parties}}}}
%$A$ et $B$ étant deux parties non vides d'un ensemble fini, $A \setminus B$, $B %%\setminus A$ et $A \cap B \neq 0$\\
%sont des ensemble disjoints deux à deux, dont la réunion est égale à $A \cup B$.\\
%Donc $card(A \cup B)=card(A \setminus B)+card(B \setminus A)+card(A \cup B)$\\
%			or
%		$card(A)=card(A \setminus B)+card(A \cap B)$\\
%		$card(B)=card(B \setminus A)+card(A \cap B)$\\
si $A \cap B \neq \emptyset $ alors $card(A \cup B)=card(A)+card(B)-card(A \cap B)$\\
\begin{center}
\begin{tikzpicture}
% Cercle A
\draw (0,0) circle (2cm);
\node at (0,0) [left] {$A$};
% Cercle B
\draw (2,0) circle (2cm);
\node at (2.5,0) [right] {$B$};
\node at (0.5,0) [right] {$A\cap B$};
\end{tikzpicture}
\end{center}
\subsection*{\underline{\textbf{\textcolor{red}{3.2.Cardinal de la réunion de deux parties disjointes}}}}
si $A \cap B=\emptyset $ alors $card(A \cap B)=card(A)+card(B)$ 
\begin{center}
\begin{tikzpicture}
% Cercle A
\draw (0,0) circle (1.5cm);
\node at (0,0) [left] {$A$};
% Cercle B
\draw (5,0) circle (1.5cm);
\node at (4.5,0) [right] {$B$};
\end{tikzpicture}
\end{center}
\textbf{\textcolor{red}{Exercice:}}\\
Montrer que $card(A \cup B)=card(A)+card(B)-card(A \cap B)$\\
\textbf{\textcolor{red}{Exemple1:}}\\
Les 50 élèves d'une classe de TS disposent de deux options sportives, l'athlétisme et la natation.\\27 élèves pratiquent l'athlétisme:\\ 29 élèves pratiquent la natation et 5 élèves ne pratiquent aucun des deux sports.\\
Trouver le nombre d'élèves qui pratiquent uniquement l'athlétisme, ceux qui pratiquent uniquement la nation  et ceux qui pratiquent les deux sports.\\
\textbf{\textcolor{red}{Solution:}}\\
\textbf{\textcolor{red}{Traduction des données en termes d'ensembles:}}\\
E est l'ensemble des élèves de la classe de TS;\\
A l'ensemble des élèves pratiquant l'athlétisme;\\
B l'ensemble des élèves pratiquant la natation.\\
Donc:\\
$A\cup B$ est l'ensemble des élèves pratiquant l'athlétisme ou la natation;\\
$A\cap B$ est l'ensemble des élèves pratiquant l'athlétisme et la natation;\\
$A\setminus B$ est l'ensemble des élèves pratiquant uniquement l'athlétisme;\\
$B\setminus A$ est l'ensemble des élèves pratiquant uniquement la natation;\\
$C_{E}(A\cup B)$ est l'ensemble des élèves ne pratiquant aucun des deux sports.\\
On a: card(E)=50 ; card(A)=27 ; card(B)=29 ; card($C_{E}(A\cup B)$)=5\\
\textbf{\textcolor{red}{Résolution du problème:}}\\
card($A\cup B$)=50-5\\
card($A\cap B$)=card(A)+card(B)-card($A \cup B$)\\
card($A\cap B$)=27+29-45=11\\
card($A\setminus B$)=card(A)-card($A \cap B$)=27-11=16\\
card($B\setminus A$)=card(B)-card($A \cap B$)=29-11=18\\
\textbf{\textcolor{red}{Conclusion en langage courant:}}\\
11 élèves pratiquent les deux sports.\\
16 élèves pratiquement uniquement l'athlétisme.\\
18 élèves pratiquent uniquement la natation.\\
\textbf{\textcolor{red}{Exemple2:}}\\
Une enquete sur la lecture des 3 journaux $\mathcal{A}$,$\mathcal{B}$,$\mathcal{C}$ d'une ville a donné les résultats suivants.\\
Sur 1500 personnes intérrogées: \\
900 lisent $\mathcal{A}$ ; \\
750 lisent $\mathcal{B}$ ; \\
750 lisent $\mathcal{C}$ ; \\
300 lisent $\mathcal{B}$ et $\mathcal{C}$ ; \\
450 lisent $\mathcal{C}$ et $\mathcal{A}$ ; \\
450 lisent $\mathcal{A}$ et $\mathcal{B}$ ; \\
100 lisent $\mathcal{A}$ ; $\mathcal{B}$ et $\mathcal{C}$.\\
1.Combien de personnes lisent deux journaux et deux seulement?\\
2.Combien de personne ne lisent aucun de ces journaux.\\
\textbf{\textcolor{red}{Solution:}}\\
\textbf{\textcolor{red}{Traduction des données en termes d'ensembles:}}\\
E l'ensemble des personnes interrogées;\\
A l'ensemble des personnes qui lisent le journal $A$;\\
B l'ensemble des personnes qui lisent le journal $B$;\\
B l'ensemble des personnes qui lisent le journal $C$.\\
Donc:\\ A$\cup$B$\cup$C est l'ensemble des personnes qui lisent au moins un des trois journaux.
\\
A$\cap$B est l'enseble des personnes qui lisent les journaux $A$ et $B$,\\
B$\cap$C est l'enseble des personnes qui lisent les journaux $B$ et $C$;\\
A$\cap$C est l'enseble des personnes qui lisent les journaux $A$ et $C$;\\
A$\cap$B$\cap$C est l'enseble des personnes qui lisent les trois journaux $A$,$B$ et $C$;\\
On a :\\
card(E)=1500 ; card(A)=900 ; card(B)=750 ; card(C)=750 ; card($A\cap B$)=450 ; 
card($A\cap C$)=450 ; card($B\cap C$)=300 ; card($A\cap B\cap C$)=100\\
\textbf{\textcolor{red}{Résolution du problème:}}\\
\textbf{\textcolor{red}{Question 1:}}\\
Nombre de personnes qui lisent $A$ et $B$ mais ne lisent pas $C$:\\
\quad\quad card($A\cap B$)-card($A\cap B\cap C$)=450-100=350\\
Nombre de personnes qui lisent $A$ et $C$ mais ne lisent pas $B$;\\
\quad\quad card($A\cap C$)-card($A\cap B\cap C$)=450-100=350\\
Nombre de personnes qui lisent $B$ et $C$ mais ne lisent pas $A$;\\
\quad\quad card($B\cap C$)-card($A\cap B\cap C$)=300-100=200\\
Nombre de personnes qui lisent seulement deux journaux.\\
			350+350+200=900\\
\textbf{\textcolor{red}{Question 2:}}\\
Nombre de personnes qui lisent uniquement le journal $A$:\\
\quad\quad 900-(350+350+100)=100\\
Nombre de personnes qui lisent uniquement le journal $B$:\\
\quad\quad 750-(350+200+100)=100\\
Nombre de personnes qui lisent uniquement le journal $C$:\\
\quad\quad 750-(50+200+100)=100\\
Nombre de personnes qui lisent au moins un de ces deux journaux:\\
100+100+100+100+350+350+200=1300\\
Nombre de personnes qui ne lisent aucun de ces deux journaux:\\
1500-1300=200
\subsection*{\underline{\textbf{\textcolor{red}{2.Cardinale d'un ensemble}}}}
\textbf{\textcolor{red}{Expérience aléatoire:}}\\
Prenons 4 jetons indiscernables au toucher.\\
Inscrivons sur le premier : a, sur le deuxième : b, sur le troisième : c, sur le dernier : d et plaçons-les dans un sac.
\subsection*{\underline{\textbf{\textcolor{red}{2.P-uplets(P-liste)}}}}
\textbf{\textcolor{red}{Expérience n°1 :Tirages successifs avec remise.}}\\
* On pioche un premier jeton, par exemple : b et on le remet dans le sac.\\
* On pioche un deuxième jeton, par exemple : a et on le remet dans le sac.\\
* On pioche un troisième jeton, par exemple : a et on le remet dans le sac.\\
L’expérience est terminée. \\
Le résultat est noté à l’aide de \textbf{\textcolor{blue}{parenthèses}} : ( b ; a ; a )\\
Les résultats possibles de cette expérience sont des \textbf{\textcolor{blue}{listes}} de 3 éléments de E,  \textbf{\textcolor{blue}{avec répétition}} d’éléments possible. \\
- Une liste de 3 éléments est appelée un triplet.\\
- Plus généralement, une liste de p éléments \textbf{\textcolor{blue}{appelée un un p-uplet}} ou une \textbf{\textcolor{blue}{p-liste}}.\\
\textbf{\textcolor{blue}{Attention!}}\\
Une liste respecte un ordre : ( b ; a ; a ) $\neq$ ( a ; a ; b )
Dans une liste, il y a un premier élément, un deuxième élément, etc… 
\subsection*{\underline{\textbf{\textcolor{red}{3.Arrangements}}}}
\textbf{\textcolor{red}{Expérience n°2 :Tirages successifs sans remise}}\\
* On pioche un premier jeton, par exemple : b que l’on ne remet pas dans le sac.\\
* On pioche un deuxième jeton, par exemple : a que l’on ne remet pas dans le sac.\\
* On pioche un troisième jeton, par exemple : c que l’on ne remet pas dans le sac.\\
L’expérience est terminée.\\

Le résultat est noté à l’aide de \textbf{\textcolor{red}{parenthèses :}} ( b ; a ; c )\\
Les résultats possibles de cette expérience sont des \textbf{\textcolor{blue}{listes}} de 3 éléments de E,
\textbf{\textcolor{blue}{sans répétition}} d’éléments possible.\\

- Une liste de 3 éléments sans répétition possible est appelée un arrangement de 3 éléments.\\
- Plus généralement, une liste de p éléments sans répétition possible est appelée\\ \textbf{\textcolor{blue}{un arrangement de p éléments de E.}}\\
Remarques :\\
1) Cette dénomination a pour avantage de bien marquer l’importance de l’ordre dans une telle liste.\\

2) Un arrangement de p éléments de E est un cas particulier de p-uplet d’éléments de E.\\
\subsection*{\underline{\textbf{\textcolor{red}{4.Permutations(Cas particulier d’arrangement:)}}}}
\textbf{\textcolor{red}{Expérience n°3 :Tirages successifs sans remise}}\\
Si l’on réalise autant de pioches sans remise qu’il y a de jetons dans le sac, on obtient alors une liste de tous les éléments de E rangés dans un certain ordre. \\
Une telle liste est appelée une permutation des éléments de E.\\

Par exemple : ( d ; b ; c ; a ) est une permutation des éléments de E.\\
Et : ( c ; a ; d ; b ) en est une autre.\\
Plus généralement : un arrangement de n éléments d’un ensemble E à n éléments est appelé permutation des éléments de E.\\
\subsection*{\underline{\textbf{\textcolor{red}{5.Combinaisons}}}}
\textbf{\textcolor{red}{Expérience n°4 :Tirages simultanés.}}\\
On pioche trois jetons en une seule fois, par exemple : a, d et c.\\
Le résultat est noté à l’aide d’accolades : $\lbrace a ; d ; c \rbrace$\\
Les résultats possibles de cette expérience sont des sous-ensembles de E ou parties de E, possédant 3 éléments.\\

Un sous-ensemble de E comportant 3 éléments est appelé une combinaison de 3 éléments de E.\\

Plus généralement, une partie de E possédant p éléments est appelée une combinaison de p éléments de E.\\

- L’ensemble $\lbrace d ; a ; c \rbrace$ possède les mêmes éléments que l’ensemble $\lbrace a ; d ; c \rbrace$ ils sont donc égaux.\\
Par conséquent, contrairement aux listes, \textbf{\textcolor{blue}{l’ordre d’écriture des éléments d’une combinaison n'est pas important.}}\\

Cette absence d’importance de l’ordre est marquée par l’utilisation de 
\textbf{\textcolor{blue}{l’écriture avec accolades, écriture réservée aux ensembles.}}
\section*{\underline{\textbf{\textcolor{red}{III.Dénombrement}}}}
Nous savons ce qu’est, par exemple, un arrangement de 3 éléments de E, mais le problème est maintenant de trouver combien on peut former de listes de ce type.\\
Deux grandes techniques de dénombrement existent. 
\subsection*{\underline{\textbf{\textcolor{red}{1.Arrangements}}}}
Nous savons ce qu’est, par exemple, un arrangement de 3 éléments de E,
mais le problème est maintenant de trouver combien on peut former de listes de ce type.\\

Deux grandes techniques de dénombrement existent.\\

Voici la première de ces techniques, appliquée au dénombrement des arrangements de 3 éléments
de l’ensemble E, défini plus haut : \\
\subsection*{\underline{\textbf{\textcolor{red}{Technique de l’arbre :}}}}
Pour compter le nombre d'arrangements possibles de 3 éléments d'un ensemble E, nous utilisons une méthode appelée la "technique de l'arbre".\\

    1.Pour le premier élément de la liste, nous avons 4 choix possibles car nous choisissons parmi les 4 éléments de E.\\

    2.Pour le deuxième élément, une fois que le premier élément est choisi, il reste 3 éléments dans E parmi lesquels choisir, car nous ne permettons pas de répétition dans la liste.\\

    3.Enfin, pour le troisième élément, une fois que les deux premiers sont choisis, il reste 2 éléments dans E  parmi lesquels choisir, toujours sans répétition.\\

En multipliant ces nombres de choix ensemble (4 pour le premier élément, 3 pour le deuxième et 2 pour le troisième), nous obtenons le nombre total d'arrangements possibles, qui est\\ 4x3x2 = 24.\\
\begin{tikzpicture}[level distance=3cm,
  level 1/.style={sibling distance=3cm},%Ecarte les branches des 1eme ramifications
  level 2/.style={sibling distance=1cm},%Ecarte les branches des  2eme ramifications
  level 3/.style={sibling distance=0.5cm}]%Ecarte les branches des 3eme ramifications
  \node {}
    child {node {a}
      child {node {b}
		child {node {c}}
        child {node {d}}      
      }
      child {node {c}
		child {node {b}}
        child {node {d}}      
      }
      child {node {d}
		child {node {b}}
        child {node {c}}       
      }
    }% 1ere branche 
    child {node {b}
      child {node {a}
      	child {node {c}}
        child {node {d}}
      }
      child {node {c}
		child {node {a}}
        child {node {d}}      
      }
      child {node {d}
		child {node {a}}
        child {node {c}}      
      }
    }   
    child {node {c}
      child {node {a}
		child {node {b}}
        child {node {d}}      
      }
      child {node {b}
		child {node {a}}
        child {node {d}}      
      }
      child {node {d}
		child {node {a}}
        child {node {b}}      
      }
    }
    child {node {d}
      child {node {a}
		child {node {b}}
        child {node {c}}      
      }
      child {node {b}
		child {node {a}}
        child {node {c}}      
      }
      child {node {c}
		child {node {a}}
        child {node {b}}      
      }
    };
\end{tikzpicture}\\
Cette méthode utilise une approche systématique pour compter toutes les possibilités, en organisant les choix en étapes successives pour garantir que chaque arrangement unique est pris en compte.
\subsection*{\underline{\textbf{\textcolor{red}{Technique des cases :Principe Multiplicatif}}}}
« Fabriquer » un arrangement de 3 éléments de E, équivaut à remplir les 3 cases suivantes avec des éléments 2 à 2 distincts :\\
Il y a 4 choix possibles pour le premier élément.\\

Puis le choix du premier élément étant fait, il reste 3 choix possibles pour le deuxième.\\

Et enfin, le choix des deux premiers éléments étant fait, il reste 2 choix possibles pour le dernier.\\
\begin{tikzpicture}
  % Case 1
  \draw (0,0) rectangle (1,1);
  \node at (0.5,1.5) {Case 1};
  \node at (0.5,-0.5) {1ère élément de la liste};
  \node at (0.5,0.5) {a};
  
  % Case 2
  \draw (5,0) rectangle (6,1);
  \node at (5.5,1.5) {Case 2};
  \node at (5.5,-0.5) {2ème élément de la liste};
  \node at (5.5,0.5) {b};
  
  % Case 3
  \draw (10,0) rectangle (11,1);
  \node at (10.5,1.5) {Case 3};
  \node at (10.5,-0.5) {3ème élément de la liste};
  \node at (10.5,0.5) {c};
\end{tikzpicture}\\
Il y a 4 choix possibles pour le premier élément.\\

Puis le choix du premier élément étant fait, il reste 3 choix possibles pour le deuxième.\\

Et enfin, le choix des deux premiers éléments étant fait, il reste 2 choix possibles pour le dernier.\\
\begin{tikzpicture}
  % Case 1
  \draw (0,0) rectangle (1.5,1.5);
  \node at (0.5,1.9) {Case 1};
  \node at (0.5,-0.5) {1ère élément de la liste};
  \node at (0.8,0.8) {4 choix};
  \node at (3,0.8) {X};
  \node at (-1.6,0.8) {nombre de liste=};
  
  % Case 2
  \draw (5,0) rectangle (6.5,1.5);
  \node at (5.7,1.8) {Case 2};
  \node at (5.7,0.8) {3 choix};
  \node at (7.9,0.8) {X};
  \node at (5.5,-0.5) {2ème élément de la liste};
  
  % Case 3
  \draw (10,0) rectangle (11.5,1.5);
  \node at (10.9,1.9) {Case 3};
  \node at (10.5,-0.5) {3ème élément de la liste};
  \node at (10.7,0.8) {2 choix};
  \node at (11.9,0.8) {=24};
\end{tikzpicture}\\ 
\underline{\textbf{\textcolor{red}{Remarque:}}}\\
cette technique équivalente à celle de l’arbre, est parfois plus pratique quand par exemple un élément de la liste est connu ainsi que sa position.\\

\underline{\textbf{\textcolor{red}{Généralisation:}}}\\
soit un entier naturel $n > 1$, et soit p entier naturel tel que : $1 < p < n$.\\
Le nombre d’arrangements de p éléments d’un ensemble E à n éléments est noté : $A_{n}^{p}$\\
Et en généralisant le raisonnement tenu sur le cas particulier, on a : \\
$A_{n}^{p}=n\times(n-1)\times(n-2)\times...\times[n-(n-p)]$\\
$A_{n}^{p}=\frac{n!}{(n-p)!}$\\
\underline{\textbf{\textcolor{red}{Exemple:}}}\\
Dans un parking de 5 places, combien y a-t-il de facon de garer trois voiture?\\
\underline{\textbf{\textcolor{red}{Solution:}}}\\
\subsection*{\underline{\textbf{\textcolor{red}{2.Permutations :}}}}
* Si p = n, on dénombre alors les permutations d’éléments de E.\\
Sur notre cas particulier, en utilisant par exemple la technique des cases, on trouve qu’il existe : 4x3x2x1 permutations des éléments de E.\\
Soit : 24 permutations des 4 éléments de E.\\
Plus généralement, une permutation étant un arrangement de n éléments de E, il en existe :
$n\times(n-1)\times(n-2)\times...\times[n-(n-1)]$\\
Soit :$n\times(n-1)\times(n-2)\times...\times2\times1$\\
\underline{\textbf{\textcolor{red}{Généralisation:}}}\\
pour tout entier $n \geq 1$,\\
Le nombre de permutations d’un ensemble à n éléments est noté : n! (se lit "factoriel n")\\
Avec: n!=$n\times(n-1)\times(n-2)\times...\times2\times1$\\
Par convention : 0! = 1\\
\underline{\textbf{\textcolor{red}{Exemple:}}}\\
De combien de facon differentes peut-on disposer 4 personnes sur 4 chaises numérotés ?\\
\underline{\textbf{\textcolor{red}{Solution:}}}\\
\subsection*{\underline{\textbf{\textcolor{red}{3.p-uplets :}}}}
Toujours avec notre exemple, en dénombrant à l’aide de la technique des cases et en tenant compte du fait que la répétition d’un même élément est possible :\\

on trouve qu’il existe : 4x4x4 triplets possibles, soit 64 triplets formés avec les éléments de E.\\
\underline{\textbf{\textcolor{red}{Généralisation:}}}\\
Soit un entier naturel $n > 1$, et soit p entier naturel tel que : $p > 1$
Le nombre de p-uplets d’un ensemble à n éléments est : $n^{p}$\\
\underline{\textbf{\textcolor{red}{Remarque :}}}\\
Comme il peut y avoir répétition des éléments de E, p peut être strictement plus grand que n.\\
\underline{\textbf{\textcolor{red}{Exemple:}}}\\ 
\underline{\textbf{\textcolor{red}{Solution:}}}\\
\subsection*{\underline{\textbf{\textcolor{red}{4.combinaisons :}}}}
 Considérons la combinaison de 3 éléments de E : $\lbrace a ; b ; c \rbrace$\\
En permutant ses éléments, il est possible de former des arrangements de 3 éléments de E.\\

Et le nombre de permutations d’un ensemble de 3 éléments étant : 3!, il est donc possible à partir de cette combinaison de former 6 arrangements de 3 éléments de E.\\
On peut évidemment faire de même avec les autres combinaisons de 3 éléments de E, obtenant ainsi tous les arrangements de 3 éléments de E.\\

De plus, deux combinaisons différentes ne peuvent générer deux arrangements identiques.\\
Donc, si nous notons $C^{3}_{4}$ le nombre de combinaisons de 3 éléments de E, par analogie avec la notation $A^{3}_{4}$ des arrangements de 3 éléments de E, on a alors : $A^{3}_{4}=6 \times C^{3}_{4}$ or 
$A^{3}_{4}=4\times3\times2=24$\\ Donc $C^{3}_{4}=4$\\
En effet, les combinaisons possibles sont : $\lbrace a ; b ; c \rbrace$,$\lbrace a ; b ; d \rbrace$,
$\lbrace a ; c ; d \rbrace$, $\lbrace b ; c ; d \rbrace$\\
\underline{\textbf{\textcolor{red}{Généralisation:}}}\\
Généralisons ce raisonnement au cas d’une combinaison de p éléments
d’un ensemble E à n éléments.\\

Chaque combinaison de p éléments, par permutations, génère p! arrangements
de p éléments de E.\\
Donc, avec les notations utilisées précédemment :$ A^{p}_{n}=p! \times C^{p}_{n}$ or, 
$A_{n}^{p}=n\times(n-1)\times(n-2)\times...\times[n-(p-1)]$\\
Donc $C^{p}_{n}=\frac{n\times(n-1)\times(n-2)\times...\times[n-(p-1)]}{p!}=\frac{A^{p}_{n}}{p!}
=\frac{\frac{n!}{(n-p)!}}{p!}=\frac{n!}{p!(n-p)!}$\\
D'où $C^{p}_{n}=\frac{n!}{p!(n-p)!}$.\\ $C^{p}_{n}$ se lit:"p parmi n" \\
\underline{\textbf{\textcolor{red}{Que dirais cette formule si elle pouvait parler?}}}\\
Le nombre de combinaisons de p éléments d'un ensemble de E à n éléments est noté:$C^{p}_{n}$\\
D'un point de vu pratique, n correspond au nombre de façon de choisir p élément parmi n, l'ordre de choix ne comptant pas.\\
\underline{\textbf{\textcolor{red}{Cas particlier sur la formule:}}}\\
\textbf{\textcolor{red}{Cas p = 0 :}} \\
Nous avons démontré cette égalité pour $p \geq 1$ mais non pour p = 0 car alors la notion de liste n’a plus de sens.\\
Par contre, un sous-ensemble de E possédant 0 éléments existe, il est unique et il s’agit de l’ensemble vide.
On a donc :$C^{0}_{n}=1$ \\
\textbf{\textcolor{red}{Cas p = 1 :}} \\
Il existe autant de sous-ensembles de E à 1 élément que d’éléments de E donc :$C^{1}_{n}=1$\\
\textbf{\textcolor{red}{Cas p = n :}} \\
Il n’existe qu’un sous-ensemble de E à n éléments c’est E donc : $C^{n}_{n}=1$\\
\textbf{\textcolor{red}{Exemple1}} \\
Calculons le nombre de combinaisons de 3 éléments d’un ensemble de cardinal 8.\\
nombre de listes de 3 éléments, sans répétitions. \\
\underline{\textbf{\textcolor{red}{Solution:}}}\\
$C^{3}_{8}=\frac{8!}{3!5!}$\\
\textbf{\textcolor{red}{Exemple2}} \\
L'éclairage d'une salle est assurée par cinq ampoules commandées chacune par un interrupteur.\\
De combien de manière peut-on éclarer cette salle en allunment exactement trois lampes?
\underline{\textbf{\textcolor{red}{Solution:}}}\\
Le nombre de manière que l'on éclarer cette salle en allunment exactement trois lampes:\\
Soient $L_{1}$,$L_{2}$,$L_{3}$,$L_{4}$et$L_{5}$ les differentes lampes.\\
Pas d'ordre\\
Pas de répétition\\
Donc $C^{3}_{5}=\frac{5!}{3!(5-3)!}=10$
\subsection*{\underline{\textbf{\textcolor{red}{5.Annagramme :}}}}
\section*{\underline{\textbf{\textcolor{red}{III.Grosse Résumé:}}}}
Récapitulons les différentes questions que l'on doit se poser, confronté a un problème de dénombrement. Cela nous permettra de choisir le concepte à utiliser en fonction de la situation.\\
\begin{tikzpicture}[level distance=3cm,
  level 1/.style={sibling distance=5cm},%Ecarte les branches des 1eme ramifications
  level 2/.style={sibling distance=9cm},%Ecarte les branches des  2eme ramifications
  level 3/.style={sibling distance=2cm}]%Ecarte les branches des 3eme ramifications
    every node/.style={text width=2cm, align=center}]%Permet de spécifier une largeur pour chaque nœud
  \node {résultat ordonné?}
    child {node {Répétition?}
     child {node {p-uplet $n^{p}$}    
      }
      child {node {}
        child {node {$n!$}} 
      	child {node {$A^{p}_{n}$}}     
      }
    }% 1ere branche      
    child {node {$C^{p}_{n}$}
    };
\node at (0.5,-5) [right] {$Non$};
\node at (0.8,-1) [right] {$Non$};
\node at (-2,-1) [right] {$Oui$};
\node at (-7,-5) [right] {$Oui$};
\end{tikzpicture}\\
\subsection*{\underline{\textbf{\textcolor{red}{Exemples:}}}}
\underline{\textbf{\textcolor{red}{Exemples1:}}}\\
Combien de mots composés de 3 lettres de l'alphabet peut-on former?\\
\underline{\textbf{\textcolor{red}{Exemples2:}}}\\
On dispose de 26 jetons marqués des 26 lettres de l'alphabet. On tire successivement et sans remise 3 jetons.\\
Combien de mots de 3 lettres peut-on former?\\
\underline{\textbf{\textcolor{red}{Exemples3:}}}\\
Quel est le nombre d'anagramme du mot "MDR".\\
\underline{\textbf{\textcolor{red}{Exemples4:}}}\\
On dispose de 6 jetons marqués des 6 couleurs différentes. On tire simultanément 3 jetons.\\
Combien de possibilités existe-t-il ?
\subsection*{\underline{\textbf{\textcolor{red}{6.Formules :}}}}
$C^{n-p}_{n}=C^{p}_{n}$\\
$C^{p-1}_{n-1}+C^{p}_{n-1}=C^{p}_{n}$\\
\underline{\textbf{\textcolor{red}{Triangle de Pascal:}}}\\
ce qui peut être utile pour la formule qui suit : \\
\begin{equation*}
\begin{array}{c|ccccccccc}
n \backslash p & 0 & 1 & 2 & 3 & 4 & 5 & 6 & 7 & 8 \\
\hline
0 & 1 \\
1 & 1 & 1 \\
2 & 1 & 2 & 1 \\
3 & 1 & 3 & 3 & 1 \\
4 & 1 & 4 & 6 & 4 & 1 \\
5 & 1 & 5 & 10 & 10 & 5 & 1 \\
6 & 1 & 6 & 15 & 20 & 15 & 6 & 1 \\
7 & 1 & 7 & 21 & 35 & 35 & 21 & 7 & 1 \\
8 & 1 & 8 & 28 & 56 & 70 & 56 & 28 & 8 & 1 \\
\end{array}
\end{equation*}
\underline{\textbf{\textcolor{red}{Formule du binôme de Newton :}}}\\
Pour tout a et b, et $n\geq 1$:\\
\[(a+b)^{n}=\sum_{k=0}^{n}C_{n}^{p}a^{n-k}b^{k}\]
\underline{\textbf{\textcolor{red}{Exemple:}}}\\
Calculons $(1+x)^{5}$\\
$(1+x)^{5}=\sum_{k=0}^{5}C_{5}^{k}1^{5-k}x^{k}$=$C_{5}^{0}1^{5-0}x^{0}+C_{5}^{1}1^{5-1}x^{1}+C_{5}^{2}1^{5-2}x^{2}+C_{5}^{3}1^{5-3}x^{3}+C_{5}^{4}1^{5-4}x^{4}+C_{5}^{5}1^{5-5}x^{5}$\\
\underline{\textbf{\textcolor{red}{Remarque:}}}\\
Les coefficients binomiaux pouvait également être trouvés en remplissant le triangle de Pascal jusqu’à la ligne 5. 
\begin{equation*}
\begin{array}{c|cccccc}
n \backslash p & 0 & 1 & 2 & 3 & 4 & 5 \\
\hline
0 & 1 \\
1 & 1 & 1 \\
2 & 1 & 2 & 1 \\
3 & 1 & 3 & 3 & 1 \\
4 & 1 & 4 & 6 & 4 & 1 \\
5 & \cellcolor{red!25}1 & \cellcolor{red!25}5 & \cellcolor{red!25}10 & \cellcolor{red!25}10 & \cellcolor{red!25}5 & \cellcolor{red!25}1 \\
\end{array}
\end{equation*}
\end{document}