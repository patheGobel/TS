\documentclass[12pt]{article}
\usepackage{stmaryrd}
\usepackage{graphicx}
\usepackage[utf8]{inputenc}

\usepackage[french]{babel}
\usepackage[T1]{fontenc}
\usepackage{hyperref}
\usepackage{verbatim}

\usepackage{color,soul}

\usepackage{amsmath}
\usepackage{amsfonts}
\usepackage{amssymb}
\usepackage{systeme}
\usepackage{tkz-tab}
\author{Destiné à la TerminaleS2\\Au Lycée de Dindéferlo}
\title{\textbf{Nombres Complexes}}
\date{\today}
\usepackage{tikz}
\usetikzlibrary{arrows}

\usepackage[a4paper,left=20mm,right=20mm,top=15mm,bottom=15mm]{geometry}
\usepackage{mathtools}
\usepackage{systeme}
\usepackage{hyperref} % Pour avoir des références colorées si nécessaire
\usepackage{eso-pic}         % Pour ajouter des éléments en arrière-plan
\newcounter{exemple} % Compteur pour les questions

% Définir la commande pour afficher une question numérotée
\newcommand{\exemple}{%
  \refstepcounter{exemple}%
  \textbf{\textcolor{green}{Exemple \theexemple :}} \ignorespaces
}
%---------------------------------------
\definecolor{myorange}{rgb}{1.0, 0.8, 0.0}

% Définir un compteur pour les exercices d'application
\newcounter{exerciceapp}

% Définir la commande pour afficher un exercice d'application numéroté
\newcommand{\exerciceapp}{%
  \refstepcounter{exerciceapp}%
  \textbf{\textcolor{myorange}{Exercice d'application \theexerciceapp :}} \ignorespaces
}
%--------------------------------------
% Définir un compteur pour les exercices d'application
\newcounter{correction}

% Définir la commande pour afficher un correction exercice d'application numéroté
\newcommand{\correction}{%
  \refstepcounter{correction}%
  \textbf{\textcolor{myorange}{Correction \thecorrection :}} \ignorespaces
}
%--------------------------------------
% Commande pour ajouter du texte en arrière-plan
\AddToShipoutPicture{
    \AtTextCenter{%
        \makebox[0pt]{\rotatebox{80}{\textcolor[gray]{0.9}{\fontsize{10cm}{10cm}\selectfont Pathé Gobel BA}}}
    }
}
%This command takes a colour as an optional argument; the default colour is black.
\usetikzlibrary{shapes.geometric,fit}
\newcommand{\myul}[2][black]{\setulcolor{#1}\ul{#2}\setulcolor{black}}
\newcommand\tab[1][1cm]{\hspace*{#1}}

\DecimalMathComma
\begin{document}
\maketitle
\newpage
\section*{Introduction aux Nombres Complexes}

Les nombres complexes, que nous allons explorer dans ce cours, ont une origine intrigante liée à la résolution des équations polynomiales. Au cours de l'histoire des mathématiques, des énigmes telles que la résolution de l'équation cubique $x^3 + ax^2 + bx + c = 0$ ont captivé l'attention des mathématiciens.

Au XVIe siècle, l'équation cubique a été résolue par le mathématicien italien Niccolò Fontana, également connu sous le nom de Tartaglia. Cependant, il est intéressant de noter que la tentative de résolution de certaines équations cubiques conduisit à l'introduction des nombres complexes.

L'équation $x^2 + 1 = 0$, apparemment insoluble dans l'ensemble des nombres réels, était une pierre d'achoppement dans la résolution des équations cubiques. Pourtant, pour surmonter cette difficulté, les mathématiciens du XVIe siècle, tels que Gerolamo Cardano et Rafael Bombelli, ont introduit une unité imaginaire, notée $i$, qui était la solution de l'équation $i^2 = -1$. Cette innovation a été la première étape vers la création de l'ensemble des nombres complexes.

Ainsi, les nombres complexes ne sont pas seulement une extension abstraite des nombres réels, mais ils ont émergé de la nécessité de résoudre des problèmes concrets en algèbre. Dans ce cours, nous explorerons les propriétés fascinantes et les applications puissantes des nombres complexes.

Préparez-vous à plonger dans le monde fascinant des nombres complexes, où l'imagination des mathématiciens a transcendé les limites des nombres réels pour créer un outil puissant et élégant.

\section*{\underline{\textbf{\textcolor{red}{I.Définition et notations }}}}

On appelle ensemble des nombres complexes l'ensemble noté $\mathbb{C}$ qui :

\begin{itemize}
  \item[$\blacktriangleright$] contient l'ensemble des réels $\mathbb{R}$ : $\mathbb{R} \subset \mathbb{C}$
  \item[$\blacktriangleright$] est muni de l'addition et de la multiplication, possédant les mêmes propriétés que $\mathbb{R}$ (commutativité, associativité, élément neutre, élément symétrique ou opposé, distributivité, etc.)
  \item[$\blacktriangleright$] contient un élément noté $i$ vérifiant $i^{2}=-1$
\end{itemize}

$\bullet$ Tout nombre complexe $z \in \mathbb{C}$ s'écrit, de façon algébrique, sous la forme $z=a+ib$, où $a$ et $b$ sont des nombres réels. 

$a$ est appelé la partie réelle de $z$ notée $\mathcal{R}e(z)=a$, et $b$ est appelé la partie imaginaire de $z$ notée $\mathcal{I}m(z)=b$.

$\bullet$ Le nombre complexe $z=a+ib$ est réel si, et seulement si, $b=0$ (c'est-à-dire $\mathcal{I}m(z)=0$).

$\bullet$ Le nombre complexe $z=a+ib$ est imaginaire pur si, et seulement si, $a=0$ (c'est-à-dire $\mathcal{R}e(z)=0$).

$\bullet$ Le nombre complexe $z=a+ib$ est nul si, et seulement si,
\[
  \systeme{a=0, b=0}
  \iff
  \systeme{\mathcal{R}e(z)=0, \mathcal{I}m(z)=0}
\]

$\bullet$ Soient $z$ et $z'$ deux nombres complexes tels que $z=a+ib$ et $z'=a'+ib'$, alors $z=z'$ si, et seulement si,
\[
  \systeme{a=a', b=b'}
  \iff
  \systeme{\mathcal{R}e(z)=\mathcal{R}e(z'), \mathcal{I}m(z)=\mathcal{I}m(z')}
\]

\underline{\exemple}\\
Soit le nombre complexe \(z = 3 - 4i\).

\begin{itemize}
  \item[$\bullet$] La partie réelle de \(z\) est \(\mathcal{R}e(z) = 3\).
  \item[$\bullet$] La partie imaginaire de \(z\) est \(\mathcal{I}m(z) = -4\).
  \item[$\bullet$] \(z\) n'est pas un réel car \(\mathcal{I}m(z) \neq 0\).
  \item[$\bullet$] \(z\) n'est pas imaginaire pur car \(\mathcal{R}e(z) \neq 0\).
  \item[$\bullet$] Le conjugué de \(z\) est \(\overline{z} = 3 + 4i\).
\end{itemize}

\underline{\textbf{\textcolor{red}{Remarque }}}\\
Dans $\mathbb{C}$, où $a$ et $b$ sont des nombres réels. Les identités remarquables deviennent:
\[
  \begin{aligned}
    (a+ib)^{2} &= a^{2}+2iab-b^{2}, \\
    (a-ib)^{2} &= a^{2}-2iab-b^{2}.
  \end{aligned}
\]

\underline{\exemple}\\
Considérons le nombre complexe $z=2+3i$. Appliquons les identités remarquables :
\[
  \begin{aligned}
    (2+3i)^{2} &= (2)^{2} + 2 \cdot 2 \cdot 3i + (3i)^{2} \\
               &= 4 + 12i - 9 \\
               &= -5 + 12i.
  \end{aligned}
\]

De même, pour $z$, $\overline{z} = 2-3i$, appliquons l'identité remarquable correspondante :
\[
  \begin{aligned}
    (2-3i)^{2} &= (2)^{2} - 2 \cdot 2 \cdot 3i + (3i)^{2} \\
               &= 4 - 12i - 9 \\
               &= -5 - 12i.
  \end{aligned}
\]

\section*{\underline{\textbf{\textcolor{red}{II. Conjugué d'un Nombre Complexe }}}}

\underline{\textbf{\textcolor{red}{1.Définition}}}

Soit $z = a + ib$ un nombre complexe, où $a$ et $b$ sont des nombres réels, le conjugué de $z$, noté $\overline{z}$, est obtenu en changeant le signe de la partie imaginaire, c'est-à-dire $\overline{z} = a - ib$.

\underline{\exemple}

Le conjugué de $w = 3 + 2i$ est $\overline{w}=3-2i$

\underline{\textbf{\textcolor{red}{2.Propriétés}}}\
\begin{itemize}
\item[$\bullet$] Le conjugué du conjugué : $\overline{\overline{z}} = z$.
\item[$\bullet$] La somme des conjugués est égale à la conjugaison d'une somme : $\overline{z_1 + z_2} = \overline{z_1} + \overline{z_2}$.
\item[$\bullet$] Le conjugué du produit est égal au produit des conjugués : $\overline{z_1 \cdot z_2} = \overline{z_1} \cdot \overline{z_2}$.
\item[$\bullet$] Le conjugué d'un quotient est égal au quotient des conjugués : $\overline{\left(\frac{z_1}{z_2}\right)} = \frac{\overline{z_1}}{\overline{z_2}}$, pour $z_2 \neq 0$.
\item[$\bullet$] $p\in\mathbb{Z}$ $\overline{z^{p}} = \overline{z}^{p}$ 
\item[$\bullet$] $z$ est réel si, et seulement si, $ \overline{z} = z $.
\item[$\bullet$] $z$ est imaginaire pur si, et seulement si, $\overline{z} = -z$.
\item[$\bullet$] $\mathcal{R}e(z)=\frac{z+\overline{z}}{2}$
\item[$\bullet$] $\mathcal{I}m(z)=\frac{z-\overline{z}}{2i}$
\end{itemize}

\underline{\exemple}

\begin{enumerate}
\item Considérons deux nombres complexes $z_1 = 2 + 3i$ et $z_2 = 1 - 2i$. 

Calculons les conjugués et vérifions les propriétés :

\begin{itemize}
\item[$\bullet$] Le conjugué du conjugué : $\overline{\overline{z_1}} = \overline{\overline{2 + 3i}} = 2 + 3i = z_1$.
\item[$\bullet$] La somme des conjugués : $\overline{z_1 + z_2} = \overline{3 + i} = 3 - i = \overline{z_1} + \overline{z_2}$.
\item[$\bullet$] Le produit des conjugués : $\overline{z_1 \cdot z_2} = \overline{8 + 5i} = 8 - 5i = \overline{z_1} \cdot \overline{z_2}$.
\item[$\bullet$] Le conjugué du produit : $\overline{z_1 \cdot z_2} = \overline{-4 - 7i} = -4 + 7i = \overline{z_1} \cdot \overline{z_2}$.
\item[$\bullet$] Le conjugué d'un rapport : $\overline{(\frac{z_1}{z_2})} = \frac{\overline{2 + 3i}}{\overline{1 - 2i}} = \frac{2 - 3i}{1 + 2i}$.

\end{itemize}
\item Soit \(z = 1 + i\) et \(p = 3\). 

Vérifions que  \(\overline{z^3} = \overline{z}^3 \)

On a :
\(
z^3 = (1 + i)^3 = -2 + 2i.
\)

Donc :
\(
\overline{z^3} = \overline{-2 + 2i} = -2 - 2i.
\)

D'autre part :
\(
\overline{z} = 1 - i, \quad \overline{z}^3 = (1 - i)^3 = (1 - i)(1 - i)(1 - i) = -2 - 2i.
\)

Ainsi, \(\overline{z^3} = \overline{z}^3\).
\end{enumerate}

\section*{\underline{\textbf{\textcolor{red}{III. Représentation graphique d'un Nombre Complexe }}}}

Les nombres complexes peuvent être vus comme des points du plan appelé.

Chaque nombre complexe est représenté par un point unique dans ce plan, où l'axe horizontal correspond à la partie réelle et l'axe vertical à la partie imaginaire.

\underline{\textbf{\textcolor{red}{1.Plan Complexes}}}

Le plan complexe est un plan muni d'un repère cartésien. 

Chaque nombre complexe $z = a + ib$ peut être associé à un point $(a, b)$ dans ce plan.

Le nombre complexe $z = a + i b$ est appelé l'affixe du point $M$. 

On peut donc noter sans ambiguïté $M(z)$ le point $M$ d'affixe $z$.

$z$ est l'affixe de $M$

$M$ est l'image de $z$

On note $M(z)$
\begin{tikzpicture}[scale=1.5]
  % Axes
  \draw[->] (-2,0) -- (2,0) node[right] {$\Re$};
  \draw[->] (0,-2) -- (0,2) node[above] {$\Im$};
  
  % Points
  \fill (1,1) circle[radius=2pt] node[above right] {$z = a + bi$};
  
  % Lignes auxiliaires
  \draw[dashed] (1,0) -- (1,1) node[midway, right] {$b$};
  \draw[dashed] (0,1) -- (1,1) node[midway, above] {$a$};
\end{tikzpicture}
\begin{tikzpicture}[scale=2]
  % Axes
  \draw[->] (-2,0) -- (2,0) node[right] {$\Re$};
  \draw[->] (0,-2) -- (0,2) node[above] {$\Im$};
  
  % Repère gradué
  \foreach \x in {-2,-1.5,...,2}
    \draw (\x,-0.05) -- (\x,0.05) node[below=2pt] {\tiny $\x$};
  \foreach \y in {-2,-1.5,...,2}
    \draw (-0.05,\y) -- (0.05,\y) node[left=2pt] {\tiny $\y$};
  
  % Points
  \fill (1,1) circle[radius=2pt] node[above right] {$z = a + bi$};
  
  % Lignes auxiliaires
  \draw[dashed] (1,0) -- (1,1) node[midway, right] {$b$};
  \draw[dashed] (0,1) -- (1,1) node[midway, above] {$a$};
\end{tikzpicture}
\\
\begin{tikzpicture}[scale=2]
  % Axes
  \draw[->] (-2,0) -- (2,0) node[right] {$\Re$};
  \draw[->] (0,-2) -- (0,2) node[above] {$\Im$};
  
  % Repère gradué
  \foreach \x in {-2,-1.5,...,2}
    \draw (\x,-0.05) -- (\x,0.05) node[below=2pt] {\tiny $\x$};
  \foreach \y in {-2,-1.5,...,2}
    \draw (-0.05,\y) -- (0.05,\y) node[left=2pt] {\tiny $\y i$};
  
  % Points
  \fill (1,1) circle[radius=2pt] node[above right] {$z = a + bi$};
  
  % Lignes auxiliaires
  \draw[dashed] (1,0) -- (1,1) node[midway, right] {$b$};
  \draw[dashed] (0,1) -- (1,1) node[midway, above] {$a$};
\end{tikzpicture}
\\	
\textbf{\textcolor{yellow}{Cette équivalence permet de considérer le plan orienté muni d'un repère orthonormé direct comme une « représentation » de l'ensemble des nombres complexes}}

\underline{\textbf{\textcolor{red}{2.Représentation d'un Nombre Complexe}}}

Soit $z = a + ib$ un nombre complexe. Le point associé dans le plan complexe est $(a, b)$.\\
\textbf{\textcolor{yellow}{On peut également utiliser la forme polaire pour représenter un nombre complexe en utilisant sa distance au point d'origine et l'angle qu'il forme avec l'axe des abscisses}}.

\underline{\exemple}

Considérons le nombre complexe $z = 3 + 4i$. 

Le point associé dans le plan complexe est $(3, 4)$.\\
Représenter $z$ et $\overline{z}$\\

++++++++++++++++++++++++++++++++++++++++++++++++++++++

\begin{tikzpicture}[>=stealth]
    % Axes
    \draw[->] (-1, 0) -- (4, 0) node[below] {\quad Re};
    \draw[->] (0, -1) -- (0, 4) node[left] {Im};
    % Point z
    \draw[->, thick, blue] (0, 0) -- (3, 2) node[midway, above left] {$z$};
    \fill (3, 2) circle (2pt) node[above right] {$(3, 2)$};
    % Grille
    \draw[dashed, gray] (3, 0) -- (3, 2) -- (0, 2);
    \node[below] at (3, 0) {3};
    \node[left] at (0, 2) {2};
\end{tikzpicture}


++++++++++++++++++++++++++++++++++++++++++++++++++++++

\section*{\underline{\textbf{\textcolor{red}{IV. Module d'un Nombre Complexe}}}}
\subsection*{\underline{\textbf{\textcolor{red}{1.Définition  }}}} 
Le module de $z$, noté $|z|$, est défini comme la distance entre le point représentant $z$ et l'origine $(0,0)$ dans le plan complexe.

Le module d'un nombre complexe $z = a + ib$ est défini comme :
\[ |z| = \sqrt{a^2 + b^2} \]

\subsection*{\underline{\textbf{\textcolor{red}{2.Propriétés}}}}

Le module d'un nombre complexe possède plusieurs propriétés importantes :

\begin{itemize}
\item[$\bullet$] Le module d'un conjugué : $|\overline{z}| = |z|$.
\item[$\bullet$] Le module d'un produit : $|z_1 \cdot z_2| = |z_1| \cdot |z_2|$.
\item[$\bullet$] Le module d'une puissance : $|z^n| = |z|^n$.
\item[$\bullet$] Le module d'un conjugué : $z\overline{z} = |z|=\sqrt{a^2 + b^2}$.
\item[$\bullet$] Inégalité triangulaire : Pour tous \(z_1, z_2 \in \mathbb{C}\),\(|z_1 + z_2| \leq |z_1| + |z_2|\)
\item[$\bullet$] Pour deux points \(A\) et \(B\) d’affixes respectives \(z_A\) et \(z_B\), la distance \(AB\) entre ces points est donnée par : $|z_A - z_B| =AB$.
\end{itemize}

\underline{\exemple}

Soit $z = 3 + 4i$. Calculons son module $|z|$.puis $|z^{5}|$.

\[ |z| = \sqrt{3^2 + 4^2} = \sqrt{9 + 16} = \sqrt{25} = 5 \]

Le module de $z$ est donc $5$.\\
\[ |z^{5}| =|z|^{5}=(\sqrt{3^2 + 4^2})^{5} = (\sqrt{9 + 16})^{5} = (\sqrt{25})^{5} = (\sqrt{5})^{4+1} = 25\sqrt{5} \]
%Les propriétés du module sont essentielles dans de nombreux calculs et démonstrations impliquant des nombres complexes. Elles fournissent une mesure de la taille d'un nombre complexe dans le plan complexe.\\

\underline{\textbf{\textcolor{green}{Exerice 1.a 1.b 1.c 1.d CIAM SE PAGE 2010}}}

\underline{\textbf{\textcolor{green}{Exerice 1.e 1.f 1.g CIAM SE PAGE 2012}}}

\textbf{\exerciceapp}

$z_{1}=-\sqrt{3}+i$ ; $z_{2}=2(-\sqrt{3}+i)$ ; $z_{3}=(-\sqrt{3}+i)(1+i)^{2}$ ; 
$z_{4}=\frac{(-\sqrt{3}+i)^{3}}{(1+i)^{2}}$ ; $z_{4}=\frac{(1+i)^{3}}{3+2i}$

1.Calcule le module de ces nombres complexe

2.Le plan est muni d'un repère orthonormal $(O,\vec{u},\vec{v})$.

Soit A, B et C les affixes respectives $1-3i$ ; $4+5i$ ; $-3+2i$.

Calculer: AB ; AC ; BC.

\section*{\underline{\textbf{\textcolor{red}{V. Argument d'un nombre complexe non nul }}}}
\subsection*{\underline{\textbf{\textcolor{red}{1.Définition}}}}
Soit z un nombre complexe non nul sur le repère othonormal $(O,\vec{u},\vec{v})$. L'argument du nombre complexe $z = a + ib$ non nul est l'angle en radian que le vecteur correspondant à $z$ forme avec l'axe des abscisses dans le plan complexe.\\ On note généralement $arg(z)=mes(\widehat{\vec{u},\vec{OM}})$\\
En notant $\theta$ on a : $\cos(\theta)=\frac{\mathcal{R}e(z)}{|z|}$ et 
$\sin(\theta)=\frac{\mathcal{I}m(z)}{|z|}$\\
\underline{\textbf{\textcolor{red}{Rappelle}}}\\
Tableau trigonométrique\\
Cercles trigonométrie\\
\subsection*{\underline{\textbf{\textcolor{red}{2.Propiétés}}}}
Soit z et z' deux nombrs complexes ,et $n\in \mathbb{N}$\\
\begin{itemize}
\item[$\bullet$] $\text{{arg}}(z_1 \cdot z_2) = \text{{arg}}(z_1) + \text{{arg}}(z_2)[2\pi]$\\
\item[$\bullet$] $\text{{arg}}(\frac{z_1}{z_2}) = \text{{arg}}(z_1) - \text{{arg}}(z_2)[2\pi]$\\
\item[$\bullet$] $\text{{arg}}(\frac{1}{z_2}) = - \text{{arg}}(z_2)[2\pi]$\\
\item[$\bullet$] $\text{{arg}}((z_2)^{n}) = n\text{{arg}}(z_2)[2\pi]$\\
\item[$\bullet$] $\text{{arg}}(-z_2) = \text{{arg}}(z_2)+\pi[2\pi]$\\
\item[$\bullet$] $\text{{arg}}(\overline{z}) = -\text{{arg}}(z)[2\pi]$\\
\item[$\bullet$] soit $z_A$ ; $z_B$ ; $z_C$ les affixes respectives des points A, B et C les image de ces complexes dans le repère orthonormal $(O,\vec{u},\vec{v})$. tel que $z_A \neq z_B$ ; 
$z_C \neq z_A$ $arg(z_A-z_B)=(\vec{u},\vec{AB})$
\item[$\bullet$] $arg(\frac{z_C-z_A}{z_B-z_A})=(\widehat{\vec{AB},\vec{AC}})$
\end{itemize}

\textbf{\exerciceapp}

Déterminer l'argument des nombres complexes suivnats\\
$z_{1}=-\sqrt{3}+i$ ; $z_{2}=2(-\sqrt{3}+i)$ ; $z_{3}=(-\sqrt{3}+i)(1+i)^{2}$ ; 
$z_{4}=\frac{(-\sqrt{3}+i)^{3}}{(1+i)^{2}}$ ; $z_{4}=\frac{(1+i)^{3}}{3+2i}$\\
\textbf{\correction}\\
\underline{\textbf{\textcolor{red}{Remarque}}}\\
\begin{itemize}
\item[$\bullet$]Si $arg(z)=0$ alors z est un réel positif
\item[$\bullet$]Si z est une réel négatif alors $arg(z)=\pi$
\item[$\bullet$]Si $z=ib$ (imaginaire pur) alors :\\
Si $b>0$ alors $arg(z)=\frac{\pi}{2}$\\
Si $b<0$ alors $arg(z)=\frac{-\pi}{2}$
\end{itemize}

\section*{\underline{\textbf{\textcolor{red}{VI. Forme d'un nombre complexe }}}}
\subsection*{\underline{\textbf{\textcolor{red}{1. Forme algébrique d'un nombre Complexe}}}}
Soit $a$ et $b$ des réels, l'écriture $z=a+ib$ est appélé forme algébrique du nombre complexe.

\underline{\exemple}\\
Donner la forme algébrique dans chaque cas
\subsection*{\underline{\textbf{\textcolor{red}{2. Forme trigonométrique d'un nombre Complexe}}}}
Soit z un nombre complexe non nul d'argument $\theta$. On appelle forme trigonométrique d'un nombre complexe l'écriture suivante:\\
$z=|z|(cos(\theta)+isin(\theta))$\\
\underline{\exemple}\\
Donner la forme trigonométrique des nombres complexes suivantes:\\
$z_{1}=-\sqrt{3}+i$ ; $z_{2}=1+i$ ; $\frac{-\sqrt{2}}{1-i}$\\
\underline{\textbf{\textcolor{red}{Solution}}}\\
\subsection*{\underline{\textbf{\textcolor{red}{3. Forme exponentielle d'un nombre complexe non nul}}}}
\subsection*{\underline{\textbf{\textcolor{blue}{a. Définition }}}}
 Soit z un nombre complexe non nul d'argument $\theta$.
 
On appelle forme exponentielle d'un nombre complexe l'écriture : $z=|z|e^{i\theta}$\\
\subsection*{\underline{\textbf{\textcolor{blue}{b. Propriétés }}}}
Soit $\theta$,$\theta_{1}$ et $\theta_{2}$ deux nombres réels et $n\in\mathbb{Z}$\\%, $z\in\mathbb{C}$ et $|z|=1$\\
$z=e^{i\theta_{1}} \times e^{i\theta_{2}} = e^{i(\theta_{1}+\theta_{2})}$\\
$z=\frac{e^{i\theta_{1}}}{e^{i\theta_{2}}} = e^{i(\theta_{1}-\theta_{2})}$\\
$z=\frac{1}{e^{i\theta}} = e^{-i\theta}$\\
$z^{n}=(e^{i\theta_{1}})^{n}=e^{in\theta_{1}}$\\
\textbf{\exerciceapp}\\
Donne la forme exponentielle des nombres complexes suivants:\\
$z_{1}=-\sqrt{3}+i$ ; $z_{2}=-\sqrt{6}+i\sqrt{2}$ et $z_{3}=1+i\sqrt{3}$\\
\textbf{\correction}\\
\section*{\underline{\textbf{\textcolor{red}{VII.Formule de Moivre-Formule d'Euler }}}}
\subsection*{\underline{\textbf{\textcolor{red}{1.Formule de Moivre }}}}
A partir de la forme trigonométrique d'un nombre complexe on peut déduire la formule de Moivre.\\
$\forall x \in \mathbb{R}$ et $n \in \mathbb{Z}$,\\
$(cos(x)+isin(x))^{n}=cos(nx)+isin(nx)$\\
$(cos(x)-isin(x))^{n}=cos(nx)-isin(nx)$\\
\underline{\exemple}

\subsection*{\underline{\textbf{\textcolor{red}{2.Formule d'Euler}}}}
A partir de la forme exponentielle d'un nombre complexe on peut déduire les formules\\ d'Euler.\\
$cos(x)$=$\frac{e^{ix}+e^{-ix}}{2}$\\
$sin(x)$=$\frac{e^{ix}-e^{-ix}}{2i}$\\
\subsection*{\underline{\textbf{\textcolor{red}{3.Linéariser des expressions trigonométriques:}}}}
Formules de linéarisations\\
$cos^{n}(x)=(\frac{e^{ix}+e^{-ix}}{2})^{n}$\\
$sin^{n}(x)=(\frac{e^{ix}-e^{-ix}}{2i})^{n}$\\
\underline{\exemple}\\
Linéariser $sin^{3}(x)$ et $cos^{3}(x)$\\
\underline{\textbf{\textcolor{red}{Solution}}}\\
$cos^{3}(x)=\frac{1}{4}cos(3x)+\frac{3}{4}cos(x)$\\
$sin^{3}(x)=\frac{-1}{4}sin(3x)+\frac{3}{4}sin(x)$
\subsection*{\underline{\textbf{\textcolor{red}{4.Expression de cosnx et sinnx en fonction de cosx ou sinx }}}}
\section*{\underline{\textbf{\textcolor{red}{VIII.Résolution d'équations dans $\mathbb{C}$}}}}
\subsection*{\underline{\textbf{\textcolor{red}{1.Déterminer les racines nièmes $(n\geq2)$ d'un complexe:}}}}
\underline{\textbf{\textcolor{blue}{a. Racine n-ième de l’unité}}}
Les \textbf{racines n-ièmes de l’unité} sont les solutions complexes de l’équation suivante :
\[
z^n = 1,
\]
où \(n \in \mathbb{N}^*\) est un entier naturel non nul.

Elles sont appelées ainsi car leur module est toujours égal à \(1\), ce qui signifie qu’elles appartiennent au cercle unité dans le plan complexe.

+++++++++++++++++++++++++++++++++++++++++++++++++

\paragraph{\underline{\textbf{\textcolor{blue}{Formule générale :}}}}
Les \(n\) racines n-ièmes de l’unité sont données par :
\[
z_k = e^{i\frac{2k\pi}{n}} = \cos\left(\frac{2k\pi}{n}\right) + i\sin\left(\frac{2k\pi}{n}\right),
\]
où \(k \in \{0, 1, 2, \ldots, n-1\}\).

\textbf{Interprétation :} Chaque \(z_k\) est une rotation par un angle \(\frac{2k\pi}{n}\) autour de l’origine, dans le sens trigonométrique (sens anti-horaire). 

+++++++++++++++++++++++++++++++++++++++++++++++++

\underline{\textbf{\textcolor{blue}{b.Racine nième d'un nombre complexe}}}\\
Soit $U \in \mathbb{C}$ et n un entier naturel différent de zéro.\\
L'équation $z^{n}=U$, $\theta=arg(z)$, a n solutions de la forme.\\
$z_{k}=\sqrt[n]{|U|}e^{i(\frac{\theta+2k\pi}{n})}$ avec $ 0\leq k \leq n-1 $.\\
\underline{\exemple}\\
Résoudre l'equation $z^{4}=\sqrt{2}+\sqrt{2}i$\\
$z=\sqrt{2}+\sqrt{2}i \Longrightarrow z_{k}=\sqrt[4]{|z|}e^{i(\frac{\theta+2k\pi)}{4}}$ avec 
$\theta=\frac{\pi}{4}$\\
\underline{\textbf{\textcolor{red}{6.1.3 Propiétés}}}\\
Soit $z_{k}=\sqrt[n]{|z|}e^{i(\frac{\theta+2k\pi)}{n}}$\\
Si $M_{0}$ ; $M_{1}$ ; $M_{2}$ ; $M_{3}$ ; $M_{4}$ ; $M_{5}$ ; ... ; $M_{k-1}$ sont les images de $z_{0}$ ; $z_{1}$ ; $z_{2}$ ; $z_{3}$ ; $z_{4}$ ; $z_{5}$ ; ... ; $z_{k-1}$
\subsection*{\underline{\textbf{\textcolor{red}{6.6 Résoudre dans $\mathbb{C}$ les équations du $2^{nd}$ degré à coefficients complexes:}}}}
++++++++++++++++++++++++++++++++++++++++++++++++++++++++++
\subsection*{\underline{\textbf{\textcolor{red}{Interprétation de $|\frac{d-c}{b-a}|$}}}}
Soit a,b,c et d des nombres complexes qui ont comme image A; B; C; D tel que $a \neq b$ et $c \neq d$\\
$|\frac{d-c}{b-a}|=\frac{CD}{AB}$\\
$arg(\frac{d-c}{b-a})=(\vec{AB},\vec{CD})$\\
\underline{\textbf{\textcolor{red}{Réel}}}\\
Si $\frac{d-c}{b-a} \in \mathbb{R}^{*}_{+}$ alors $\widehat{(\vec{AB},\vec{CD})}=0 [2\pi]$\\ 
Si $\frac{d-c}{b-a} \in \mathbb{R}^{*}_{-}$ alors $\widehat{(\vec{AB},\vec{CD})}=\pi [2\pi]$\\
\underline{\textbf{\textcolor{red}{imaginaire pur}}}\\
Si $\frac{d-c}{b-a} \in i\mathbb{R}^{*}_{+}$ alors $\widehat{(\vec{AB},\vec{CD})}=\frac{\pi}{2} [2\pi]$\\ 
Si $\frac{d-c}{b-a} \in i\mathbb{R}^{*}_{-}$ alors $\widehat{(\vec{AB},\vec{CD})}=-\frac{\pi}{2} [2\pi]$\\
\underline{\exemple}\\
Soit $Z=\frac{iz+2}{z+i}$ un nombre complexe\\
Détermine l'ensemble des points M tel que: $|Z'|=1$ puis $|Z'|=2$\\
\underline{\textbf{\textcolor{red}{Solution}}}\\
$|Z'|=|\frac{i(z-2i)}{z+i}|$\\
Soit $M(z)$ ; $A(2i)$ et $B(-i)$ alos on a:\\
$|Z'|=|\frac{i(z_{M}-z_{B})}{z_{M}-z_{A}}|=\frac{|i|\times|(z_{M}-z_{B})|}{|z_{M}-z_{A}|}=|\frac{MA}{MB}$\\
Si $|Z'|=1$ alors $\frac{MA}{MB}=1 \Longrightarrow MA=MB$\\
L'ensemble des points $M$ tel que  $|Z'|=1$ est la médiatrice $[AB]$\\
\end{document}