\documentclass{article}
\usepackage{lmodern} % Pour une police plus nette
\usepackage{stmaryrd}
\usepackage{graphicx} % Pour l'insertion d'images
\usepackage{float}    % Pour contrôler précisément le placement
\usepackage[utf8]{inputenc}
\usepackage[french]{babel}
\usepackage[T1]{fontenc}
\usepackage{hyperref}
\usepackage{verbatim}
\usepackage{color, soul}
\usepackage{pgfplots}
\pgfplotsset{compat=1.18} % Version plus récente de pgfplots
\usepackage{mathrsfs}
\usepackage{amsmath}
\usepackage{amsfonts}
\usepackage{amssymb}
\usepackage{tkz-tab}

\begin{document}

\begin{table}[h!]
    \centering
    \renewcommand{\arraystretch}{1.5}
    \begin{tabular}{|c|c|c|c|}
        \hline
        \textbf{\( I = \) intervalle de} & \textbf{Remarques ou} & \textbf{Fonction \( f \)} & \textbf{Primitive \( F \) où \( k \) est une} \\
        \textbf{définition de \( f \)} & \textbf{restrictions} & & \textbf{constante réelle} \\
        \hline
        \( I \subset \mathbb{R} \) & & \( x \mapsto 0 \) & \( x \mapsto k \) \\
        \hline
        \( I \subset \mathbb{R} \) & \( a \in \mathbb{R} \) & \( x \mapsto a \) & \( x \mapsto ax + k \) \\
        \hline
        \( I \subset \mathbb{R} \) & \( n \) entier naturel & \( x \mapsto x^n \) & \( x \mapsto \frac{x^{n+1}}{n+1} + k \) \\
        \hline
        \( I \subset \mathbb{R}^* \) & \( n \) entier relatif, \( n \neq -1 \) & \( x \mapsto x^n \) & \( x \mapsto \frac{x^{n+1}}{n+1} + k \) \\
        \hline
        \( I \subset \mathbb{R}^+ \) & \( n \) réel \( n \neq -1 \) & \( x \mapsto x^n \) & \( x \mapsto \frac{x^{n+1}}{n+1} + k \) \\
        \hline
        \( I \subset \mathbb{R}^* \) & & \( x \mapsto \frac{1}{x^2} \) & \( x \mapsto -\frac{1}{x} + k \) \\
        \hline
        \( I \subset \mathbb{R}^+ \) & & \( x \mapsto \frac{1}{\sqrt{x}} \) & \( x \mapsto 2\sqrt{x} + k \) \\
        \hline
        \( I \subset \mathbb{R} \) & & \( x \mapsto \sin x \) & \( x \mapsto -\cos x + k \) \\
        \hline
        \( I \subset \mathbb{R} \) & & \( x \mapsto \cos x \) & \( x \mapsto \sin x + k \) \\
        \hline
        \( I \subset \mathbb{R} \) & & \( x \mapsto \sin(ax+b) \) & \( x \mapsto -\frac{1}{a} \cos(ax+b) + k \) \\
        \hline
        \( I \subset \mathbb{R} \) & & \( x \mapsto \cos(ax+b) \) & \( x \mapsto \frac{1}{a} \sin(ax+b) + k \) \\
        \hline
        \( I \subset \mathbb{R} \setminus \left\{ \frac{\pi}{2} + k\pi \right\} \) & & \( x \mapsto \frac{1}{\cos^2 x} \) & \( x \mapsto \tan x + k \) \\
        \hline
        \( I \subset \mathbb{R} \) & & \( x \mapsto 1 + \tan^2 x \) & \( x \mapsto \tan x + k \) \\
        \hline
        \( I \subset \mathbb{R}^+ \) & \( x \) positif & \( x \mapsto \sqrt{x} \) & \( x \mapsto \frac{2}{3} x^{3/2} + k \) \\
        \hline
        \( I \subset \mathbb{R} \) & & \( x \mapsto (ax+b)^n \) & \( x \mapsto \frac{1}{a} \frac{(ax+b)^{n+1}}{n+1} + k \) \\
        \hline
        \( I \subset \mathbb{R} \) & Avec les mêmes & & \\
        & contraintes sur \( n \) et sur \( (ax+b) \) & & \\
        \hline
    \end{tabular}
    \caption{Tableau des primitives}
\end{table}

\end{document}

